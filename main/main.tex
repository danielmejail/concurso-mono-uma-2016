%\documentclass[a4paper,12pt,notitlepage]{report}
\documentclass[a4paper,12pt]{article}
%\documentclass[a4paper,10pt]{scrartcl}
%\usepackage{ucs}
\usepackage[utf8x]{inputenc}
\usepackage{graphicx}
\usepackage{tabularx}
\usepackage{amsthm}
\usepackage{amsmath}
\usepackage{amssymb}
\usepackage{amsfonts}
\usepackage{latexsym}
\usepackage{fullpage}
\usepackage{mathrsfs}

%\usepackage{makeidx}
\usepackage{import}
%\usepackage{theoremref}
%\usepackage{hyperref}
%\usepackage{fancyref}

\usepackage{tikz}
\usepackage{tikz-cd}
\usepackage{verbatim}
\usetikzlibrary{%
  matrix,%
  calc%
}

\title{El teorema de Stark-Heegner v\'{\i}a curvas modulares %
	asociadas a subgrupos de Cartan \textit{non-split} %
	de $\GL{2}(\bb{Z}/N\bb{Z})$}
\author{Daniel Mejail}
\date{26/09/2016}

\pdfinfo{%
  /Title    ()
  /Author   ()
  /Creator  ()
  /Producer ()
  /Subject  ()
  /Keywords ()
}

%elements of style
\usepackage{sectsty}
%\usepackage{titlesec}
%
\pagestyle{plain}
\allsectionsfont{\rmfamily \mdseries \scshape \centering}
%changing names
\renewcommand{\refname}{Bibliograf\'{\i}a}
%{\rmfamily \mdseries \scshape \centering Bibliograf\'{\i}a}
%this is not needed since we're using the ``\allsectionsfont''
%command

%commands and theorems

\renewcommand{\rm}[1]{\mathrm{#1}}
\renewcommand{\cal}[1]{\mathcal{#1}}
\newcommand{\bb}[1]{\mathbb{#1}}
\renewcommand{\frak}[1]{\mathfrak{#1}}
%
\newcommand{\wHat}[1]{\widehat{#1}}
%
\newcommand{\GL}[1]{\rm{GL}_{#1}}%grupo general lineal
\newcommand{\SL}[1]{\rm{SL}_{#1}}%grupo lineal especial
\newcommand{\SO}[1]{\rm{SO}_{#1}}%grupo ortogonal especial
\renewcommand{\det}{\rm{det}}%determinante (de matrices)
\newcommand{\tr}{\rm{tr}}%traza de matrices
\newcommand{\Id}{\rm{Id}}%identidad
%
\renewcommand{\mod}{\mathrm{mod}}
\newcommand{\absGal}[1]{\Gal(\overline{#1}/#1)}%grpo de Gal absoluto
\newcommand{\MM}[1]{\rm{M}_{#1\times #1}}%matrices cuadradas
\newcommand{\Frob}{\rm{Frob}}%Frobenius
\newcommand{\Cl}{\mathsf{Cl}}%grupo de clases
\newcommand{\id}{\mathrm{id}}%grupo de ideales
\newcommand{\Nm}{\rm{Nm}}%norma en cuerpos
\newcommand{\Tr}{\rm{Tr}}%traza en cuerpos
\newcommand{\nrd}{\rm{nrd}}%norma reducida (cuaterniones)
\newcommand{\trd}{\rm{trd}}%traza reducida (cuaterniones)
\newcommand{\nm}{\mathnormal{nm}}%norma de ideales (usada para ideales en cuerpos cuadr\'{a}ticos imaginarios})
%
\newcommand{\new}{\mathit{new}}%subespacio nuevo
\newcommand{\old}{\mathit{old}}%subespacio viejo
\newcommand{\ES}{\mathrm{ES}}%Eichler-Shimura
%
\newcommand{\Empty}{\emptyset}
\newcommand{\deriv}[2][]{\frac{\partial #2}{\partial #1}}
\newcommand{\setmin}{\smallsetminus}
\newcommand{\backs}{\backslash}
%
\renewcommand{\ker}{\mathrm{ker}}
\newcommand{\img}{\mathrm{img}}
\newcommand{\Gal}{\mathsf{Gal}}
\newcommand{\Hom}{\mathrm{Hom}}


%losNueveCuerpos
\theoremstyle{plain}
\newtheorem{propoEquivsNumClasUno}{Proposici\'{o}n}[section]
\newtheorem{teoStarkHeegner}[propoEquivsNumClasUno]{Teorema}

%-------------------------
%generalidades
\theoremstyle{plain}
\newtheorem{propoDefinidasSobreQ}{Proposici\'{o}n}[section]
\newtheorem{propoPuntosEnterosCuerpoQuad}[propoDefinidasSobreQ]{Proposici\'{o}n}
\theoremstyle{remark}
\newtheorem*{defSubgrupoDeCartan}{Definici\'{o}n}
\newtheorem*{ejemploSubgrupoDeCartan}{Ejemplo}
\newtheorem*{obsDetSobreUnidades}{Observaci\'{o}n}

%ramBaran
\theoremstyle{plain}
\newtheorem{lemaRepsCoclasesDelNormalizador}[propoDefinidasSobreQ]{Lema}

%--------------------------
%nivel24
\theoremstyle{plain}
\newtheorem{teoUnifsOchoYTres}{Teorema}[section]
\theoremstyle{remark}
\newtheorem*{obsSerre}{Observaci\'{o}n}

%nivel7
\theoremstyle{plain}
\newtheorem{lemaUnoKenku}[teoUnifsOchoYTres]{Lema}

%nivel5

%nivel11
\theoremstyle{plain}
\newtheorem{teoSchoofTzanakis}[teoUnifsOchoYTres]{Teorema}
\theoremstyle{remark}
\newtheorem*{obsGenUno}{Observaci\'{o}n}

%nivel9
\theoremstyle{plain}
\newtheorem{propoBaranNineXB}[teoUnifsOchoYTres]{Proposici\'{o}n}
\newtheorem{propoBaranNineXNine}[teoUnifsOchoYTres]{Proposici\'{o}n}

\begin{document}
\maketitle
\begin{section}{Introducci\'{o}n}\label{sec:intro}
 
\paragraph{}
El objectivo de esta monograf\'{\i}a es repasar algunas de las soluciones al
problema del n\'{u}mero de clases igual a $1$%
% para cuerpos de n\'{u}meros cuadr\'{a}ticos imaginarios%
, con un inter\'{e}s particular en una interpretaci\'{o}n modular. Concretamente,
nos interesar\'{a} entender la relaci\'{o}n entre \'{o}rdenes en cuerpos
cuadr\'{a}ticos imaginarios y puntos enteros en las curvas modulares asociadas a
subgrupos de Cartan \textit{non-split} de $\GL{2}(\bb{Z}/N\bb{Z})$ y sus
normalizadores. Esta relaci\'{o}n aparece mencionada en \cite{serre}, y, a lo largo
de los \'{u}ltimos cuarenta a\~{n}os han ido aparenciendo soluciones al problema
basadas en un estudio de estas curvas.

Comenzamos este trabajo enunciando el teorema de Stark-Heegner y,
siguiendo \cite{serre}, verificamos parte del enunciado. A continuaci\'{o}n,
en la secci\'{o}n \ref{sec:generalidades},
describimos, en mayor o menor detalle, algunas de las
propiedades de los objetos de inter\'{e}s: definimos los subgrupos de Cartan
\textit{non-split} y los objetos modulares correspondientes. Si $\Gamma(N)$ es
un subgrupo principal de congruencia de nivel $N$, dado un subgrupo
$H$ de $\GL{2}(\bb{Z}/N\bb{Z})$, podemos asociarle una superficie de Riemann
como cociente del semiplano complejo superior, $X_{H}$, y un cociente de una curva,
$H\backs X(N)$, donde $X(N)$ es, esencialmente, una uni\'{o}n disjunta de copias
de $\Gamma(N)\backs\frak{h}^{*}$.
%Estos objetos se
%identifican con cocientes de ciertas curvas (disconexas) emparentadas con
%$\Gamma(N)\backs\frak{h}^{*}$, el cociente (su compactificaci\'{o}n) del
%semiplano complejo superior por un subgrupo principal de congruencia.
%
%Bajo ciertas condiciones sobre el subgrupo $H$ podemos identificar ambos.
%Para terminar, analizamos algunas de las demostraciones del teorema que
%utilizan de manera esencial estos objetos%
% -informaci\'{o}n sobre sus c\'{u}spides, puntos el\'{\i}pticos, ramificaci\'{o}n%
Para terminar, en la secci\'{o}n \ref{sec:soluciones}, resumimos diversas
demostraciones del teorema de Stark-Heegner cuyo punto de contacto es la
interpretaci\'{o}n modular del problema.

\paragraph{}
Si llamamos $X(1)$ a la superficie de Riemann $\SL{2}(\bb{Z})\backs\frak{h}^{*}$,
se puede ver que existen morfismos naturales $X_{H}\rightarrow X(1)$. Si la
curva $X_{H}$ est\'{a} definida sobre $\bb{Q}$, dicho morfismo tambi\'{e}n
es un $\bb{Q}$-morfismo, y resulta ser relativamente sencillo, directo,
determinar si un punto de $X_{H}$ es \emph{\'{\i}ntegro} en funci\'{o}n de
propiedades que pueda tener su imagen en $X(1)$.

Existen distintas maneras de determinar los puntos enteros sobre las curvas
$X_{H}$; cada uno de ellos indica una conexi\'{o}n con un objeto distinto. En
\cite{kenkuLevelSeven}, se obtiene una parametrizaci\'{o}n de
la curva $X_{ns}^{+}(7)$ (ver m\'{a}s adelante secci\'{o}n \ref{sec:generalidades})
por medio de la relaci\'{o}n con formas de Klein, de acuerdo con la descripci\'{o}n
en \cite{ligozat}. En \cite{baranLevelNine}, \cite{baranNormalizers}, \cite{booher}
y \cite{chenLevelFive}, el m\'{e}todo consiste en extraer, de las parametrizaciones
de las curvas correspondientes, ecuaciones diof\'{a}nticas. Determinar las
soluciones a estas ecuaciones permite, luego, determinar los puntos enteros en cada
caso.
En \cite{schoofTzanakisLevelEleven}, la metodolog\'{\i}a para determinar puntos
enteros es completamente distinta, basada en formas lineales en logaritmos y en la
existencia de un isomorfismo entre $X_{ns}^{+}(11)$, la curva modular estudiada en
el trabajo citado, y una curva el\'{\i}ptica.

Independientemente del m\'{e}todo, la conexi\'{o}n
con el problema del n\'{u}mero de clases, y la motivaci\'{o}n que por este lado
pueda venir para estudiar las curvas asociadas a subgrupos de Cartan de
$\GL{2}(\bb{Z}/N\bb{Z})$, ya se encuentra en las observaciones de Serre en el
ap\'{e}ndice a \cite{serre}.
El problema de determinar los cuerpos cuyo n\'{u}mero de clases es igual a $1$
no es la \'{u}nica motivaci\'{o}n para estudiar las curvas asociadas a distintos
tipos de subgrupos de $\GL{2}(\bb{Z}/N\bb{Z})$. Como se ver\'{a} en los ejemplos,
no es cierto que exista una correspondencia entre cuerpos cuadr\'{a}ticos
imaginarios de n\'{u}mero de clase $1$ y puntos \'{\i}ntegros en las curvas
$X_{ns}^{+}(N)$. Estas curvas parametrizan clases de isomorfismo de curvas
el\'{\i}pticas incorporando una \emph{estructura de nivel}. En estos t\'{e}rminos,
no todos los puntos \'{\i}ntegros de $X_{ns}^{+}(N)$ est\'{a}n asociados a curvas
el\'{\i}pticas con multiplicaci\'{o}n compleja; los que s\'{\i}, tendr\'{a}n que
ver con \'{o}rdenes en cuerpos cuadr\'{a}ticos imaginarios. Seg\'{u}n
\cite{baranAnExceptionalIso}, contar puntos en curvas asociadas a subgrupos de
Cartan \textit{non-split} de $\GL{2}(\bb{Z}/N\bb{Z})$ y sus normalizadores est\'{a}
relacionado con el siguiente problema: determinar si existe una constante
tal que, para todo primo $p$ mayor que dicha constante, si $E$ es una curva
el\'{\i}ptica sin multiplicaci\'{o}n compleja, entonces la representaci\'{o}n de
Galois \textit{modulo} $p$ asociada es sobreyectiva. El problema parece reducirse a
determinar si existe una constante $C>0$ tal que, si $p>C$, entonces los \'{u}nicos
puntos $\bb{Q}$-racionales en $X_{ns}^{+}(p)$ son puntos CM, puntos asociados a
curvas el\'{\i}pticas con multiplicaci\'{o}n compleja.

Por otro lado, el tipo de soluciones mencionado permite relacionar y encarar desde
un mismo punto algunas de las demostraciones ya existentes del teorema
de Stark-Heegner: la soluci\'{o}n presentada en \cite{chenLevelFive} es
una interpretaci\'{o}n modular de la soluci\'{o}n dada por Siegel, y en
\cite{booher}, siguiendo la sugerencia en \cite{serre}, se intenta hacer lo
mismo con la demostraci\'{o}n de Heegner.

 \begin{subsection}{El teorema de Stark-Heegner %
			y su relaci\'{o}n con formas %
			cuadr\'{a}ticas}\label{subsec:losNueve}
 
\paragraph{Relaci\'{o}n con formas cuadr\'{a}ticas}
%Describimos brevemente la relaci\'{o}n entre formas
%cuadr\'{a}ticas. Esta relaci\'{o}n no s\'{o}lo muestra el origen
%del problema del n\'{u}mero de clases igual a $1$, sino que,
%tambi\'{e}n, indica d\'{o}nde radica la dificultad del teorema
%de Stark-Heegner. En cuanto a las definiciones necesarias,
%remitimos a \cite{cox}.
Describimos brevemente la relaci\'{o}n entre formas
cuadr\'{a}ticas y \'{o}rdenes en cuerpos cuadr\'{a}ticos imaginarios.
En cuanto a las definiciones necesarias, remitimos a \cite{cox}.

Sea $I$ un orden en un cuerpo cuadr\'{a}tico imaginario $K$ de
discriminante $D<0$, y sea $d_{K}$ el discriminante del cuerpo.
Sea $f=ax^{2}+bxy+cy^{2}$ una forma cuadr\'{a}tica (primitiva y
definida positiva) de discriminante $D$. Como $D$ es negativo, existe
una \'{u}nica ra\'{\i}z $\tau$ del polinomio cuadr\'{a}tico $f(x,1)$
perteneciente al semiplano complejo superior, $\frak{h}$. Porque $f$
es definida positiva, $\tau$, en t\'{e}rminos del discriminante y de
los coeficientes de $f$, es igual a $(-b+\sqrt{D})/2a$.

Sea $f$ el conductor del orden $I$, de manera que $D=f^{2}d_{K}$, y sea
$w_{K}\in K$ el elemento

\begin{align*}
w_{K} & \,:=\,\frac{d_{K}+\sqrt{d_{K}}}{2}
\text{ .}
\end{align*}
Por un lado, $\cal{O}_{K}$, el anillo de enteros de $K$, es igual, en tanto
$\bb{Z}$-m\'{o}dulo, a $(1,w_{K})_{\bb{Z}}$, el m\'{o}dulo generado por $1$ y
$w_{K}$ en $K$. Se puede ver que $(1,a\tau)_{\bb{Z}}=(1,fw_{K})_{\bb{Z}}=I$,
con lo que $(a,a\tau)_{\bb{Z}}$ es un ideal propio de $I$ en $K$.

Dadas dos formas $f$ y $g$, ellas son propiamente equivalentes,
si, y s\'{o}lo si sus ra\'{\i}ces en $\frak{h}$, $\tau$ y $\tau'$,
pertenecen a la misma \'{o}rbita en el semiplano por la acci\'{o}n de
$\SL{2}(\bb{Z})$. Esto \'{u}ltimo es, a su vez, equivalente a que los
ret\'{\i}culos $[1,\tau]$ y $[1,\tau']$ en $\bb{C}$ est\'{e}n
relacionados por una homotecia definida sobre $K$: que exista $\lambda$
en $K$ tal que

\begin{align*}
[1,\tau] & \,=\,\lambda [1,\tau']
\text{ .}
\end{align*}

Todo esto muestra que, si $f$ es una forma cuadr\'{a}tica, cuya
ra\'{\i}z asociada es $\tau$, la aplicaci\'{o}n
$f\mapsto[a,a\tau]$ determina una aplicaci\'{o}n biyectiva
del grupo de clases de formas de discriminante $D$ en el grupo de clases del
orden $I$. (En principio, esta aplicaci\'{o}n es inyectiva, pero la
sobreyectividad se sigue de que todo ideal propio de $I$ est\'{a} generado,
en tanto $\bb{Z}$-m\'{o}dulo, por dos elementos de $K$). Esta
aplicaci\'{o}n resulta ser un morfismo de grupos, con lo que
ambos son isomorfos. Los n\'{u}meros de los grupos de clases, $h(D)$ y $h(I)$,
respectivamente, son, entonces, iguales.

\paragraph{El teorema de Stark-Heegner}
Sea $K$ un cuerpo cuadr\'{a}tico imaginario y sea $d_{K}$ su
discriminante. En ese caso, $d_{K}$ siendo un discriminante
fundamental, $d_{K}$ es un entero libre de cuadrados congruente a
$1$ \textit{modulo} $4$, o es igual a $-4n$, donde $n$ es un
entero libre de cuadrados, no congruente a $3$ \textit{modulo} $4$.

Si $n$ es un entero positivo, entonces el n\'{u}mero de clases de
formas cuadr\'{a}ticas de discriminante
$-4n$ es igual a $1$, si, y s\'{o}lo si $n$ pertenece al conjunto
$\{ 1,2,3,4,7 \}$. Una demostraci\'{o}n elemental de esto se puede
encontrar en el teorema 2.18 de \cite{cox}. Esto muestra que, para
un cuerpo cuadr\'{a}tico $K$ de discriminante $d_{K}$, si
$d_{K}=-4n$, entonces $h(K)=1$ \'{u}nicamente en los casos en que
$n$ sea igual a $1$ o a $2$.
%El primero de nuestros objetivos ser\'{a} describir
%esta relaci\'{o}n. Tomamos como referencia el mencionado libro de Cox.

La demostraci\'{o}n de que los cuerpos $K=\bb{Q}(\sqrt{d_{K}})$,
con $-d_{K}$ igual a $3$, $4$, $7$, $8$, $11$, $19$, $43$, $67$ o $163$ tienen
n\'{u}mero de clases igual a $1$ es elemental: para $d_{K}=-3$, $-4$, $-8$,
los cuerpos correspondientes son
$K=\bb{Q}(\sqrt{-3})$, $\bb{Q}(i)$, $\bb{Q}(\sqrt{-2})$, cuyos anillos de
enteros son dominios eucl\'{\i}deos.

%A trav\'{e}s de la ``correspondencia'' con formas
%cuadr\'{a}ticas (secci\'{o}n 7.B de \cite{cox}), s
Si consideramos la familia de formas cuadr\'{a}ticas de discriminante $d_{K}$,
hay exactamente $2^{\mu-1}$ g\'{e}neros de formas, donde $\mu$ es la cantidad de
factores primos en $d_{K}$. Pero, si el n\'{u}mero de clases es $1$, el n\'{u}mero
de g\'{e}neros tambi\'{e}n lo es, y $\mu=1$, es decir, $d_{K}=-p$, para alg\'{u}n
primo $p\equiv 3\,(4)$. Si $p$ es congruente a $7$ \textit{modulo} $8$, entonces,
comparando con el orden $I$ de conductor $2$ en $\cal{O}_{K}$, la relaci\'{o}n
entre los n\'{u}meros de clases es

\begin{align*}
 h(I) & \,=\,h(\cal{O}_{K})\frac{2}{|\cal{O}_{K}^{\times}:I^{\times}|}
 \left(1-\left(\frac{d_{K}}{2}\right)\frac{1}{2}\right)\text{ ,}
\end{align*}
y el factor que multiplica a $h(\cal{O}_{K})$ es un entero
(teorema 7.24 de \cite{cox}). Pero, entonces,
$|\cal{O}_{K}^{\times}:I^{\times}|=1$ y $(d_{K}/2)$
(el s\'{\i}mbolo de Kronecker en $d_{K}$) es igual a $1$, con lo cual,
$h(-4p)=h(I)=h(\cal{O}_{K})=h(-p)=1$ y $p=7$ es la \'{u}nica opci\'{o}n.
%

La proposici\'{o}n que ahora enunciamos, y el argumento que le sigue, se pueden
encontrar en el ap\'{e}ndice de \cite{serre}.

\begin{propoEquivsNumClasUno}\label{thm:propoEquivsNumClasUno}
Las siguientes propiedades de un primo $p>3$ congruente a $3$
\textit{modulo} $4$ son equivalentes:

\begin{itemize}
\item[i)] $h(-p)=1$;
\item[ii)] $\left( \frac{l}{p} \right)=-1$ para todo primo $l<p/4$;
\item[ii)'] $\left( \frac{l}{p} \right)=-1$ para todo primo $l<\sqrt{p/3}$;
\item[iii)] (si $p>7$) $p\equiv 3\,(8)$ y $R-N=3$, si $R$ (respectivamente, $N$)
es el n\'{u}mero de residuos (no residuos) cuadr\'{a}ticos \textit{modulo} $p$ en
$[1,(p-1)/2]$;
\item[iv)] si $x\in[\![ 0,(p-7)/4 ]\!]$, $P_{p}(x)=x^{2}+x+(p+1)/4$ es primo;
\item[iv)'] si $x\in[0,1/2(\sqrt{p/3}-1)]$, $P_{p}(x)$ es primo.
\end{itemize}
\end{propoEquivsNumClasUno}

\begin{proof}[Demostraci\'{o}n]
El anillo de enteros de $\bb{Q}(\sqrt{-p})$ es $\bb{Z}[w]$, donde
$w=\frac{1+\sqrt{-p}}{2}$; un primo $l$ es inerte en $\bb{Q}(\sqrt{-p})/\bb{Q}$,
si, y s\'{o}lo si $(l/p)=-1$ (ver \cite{cox}). Dicho de otra manera, si $h(-p)=1$,
la condici\'{o}n $(l/p)=-1$ implica que $l$ es una norma: existe
$\alpha\in\bb{Z}[w]$, $\alpha=x+yw$, tal que $\Nm(\alpha)=l$. Pero $\Nm(\alpha)$
es igual a $(x+\frac{y}{2})^{2}+(\frac{y}{2})^{2}p$. As\'{\i}, $l$ es primo,
$y\not =0$ y $\Nm(\alpha)\geq p/4$. Rec\'{\i}procamente, asumamos que $(ii)'$ se
cumple. Entonces, si $\frak{l}\subset\bb{Z}[w]$ es un ideal primo, y vale
$\Nm(\frak{l})<\sqrt{p/3}$, dado $l$ primo que divide a $\Nm(\frak{l})$, es
$l<\sqrt{p/3}$. En particular, $(l/p)=-1$, por hip\'{o}tesis, y $\frak{l}=(l)$ es
un ideal principal. Todo ideal de norma menor que $\sqrt{p/3}$ es, entonces,
principal. Pero toda clase en $\Cl(\bb{Z}[w])$ contiene un elemento con esta
propiedad.

El resto de la demostraci\'{o}n (excepto las equivalencias con (iii)) es,
tambi\'{e}n, elemental.
\end{proof}

Sea $p\equiv 3\,(\rm{mod}\,4)$ un primo, $3<p<75$, y sea $m=(p+1)/4$. En este caso,
por (iv)', $h(-p)=1$ es equivalente a que $m$ y $m+2$ sean primos. Esto \'{u}ltimo
es cierto para $m$ en el conjunto $\{3,5,11,17\}$. El primo $p$ es
$11$, $19$, $43$ o $67$.

\begin{teoStarkHeegner}\label{thm:teoStarkHeegner}
 Sea $d_{K}<0$ el discriminante de un cuerpo cuadr\'{a}tico imaginario. Entonces
 $h(d_{K})=1$, si, y s\'{o}lo si $d_{K}$ es igual a $-3$, $-4$, $-7$, $-8$, $-11$,
 $-19$, $-43$, $-67$ o a $-163$.
\end{teoStarkHeegner}
%introducci\'{o}n, algo de [[serre]]
 \end{subsection}
\end{section}

\begin{section}{Generalidades}%{\mdseries \scshape Generalidades}
 \label{sec:generalidades}
 
Sea $\Lambda=\omega_{1}\bb{Z}\oplus\omega_{2}\bb{Z}$ un ret\'{\i}culo en $\bb{C}$,
generado por $\omega_{1}$ y $\omega_{2}$ tales que
$\omega_{1}/\omega_{2}\in\frak{h}$. Los puntos de $N$-torsi\'{o}n de
$\bb{C}/\Lambda$ conforman el grupo
\begin{math}
\langle(\omega_{1}/N)+\Lambda\rangle\times
\langle(\omega_{2}/N)+\Lambda\rangle
\end{math}
isomorfo a $(\bb{Z}/N\bb{Z})^{2}$. Sea $\mu_{N}$ el grupo de ra\'{\i}ces
$N$-\'{e}simas de la unidad en $\bb{C}$, $\mu_{N}=\langle e^{2\pi i/N}\rangle$.
Dados $P$ y $Q$ en $\bb{C}/\Lambda$ de $N$-torsi\'{o}n, existe
$\gamma\in\MM{2}(\bb{Z}/N\bb{Z})$ tal que

\begin{align*}
\begin{bmatrix}
P\\Q
\end{bmatrix} & \,=\,\gamma
\begin{bmatrix}
(\omega_{1}/N)+\Lambda\\(\omega_{2}/N)+\Lambda
\end{bmatrix}\text{ .}
\end{align*}
Definimos $e_{N}(P,Q):=e^{2\pi i\det(\gamma)/N}$, el \textit{pairing} de Weil en
$P$ y $Q$. Esto define, v\'{\i}a la correspondencia con curvas el\'{\i}pticas
definidas sobre $\bb{C}$, una aplicaci\'{o}n,
$E[N]\times E[N]\rightarrow\mu_{N}$, en pares de puntos de $N$-torsi\'{o}n de una
curva el\'{\i}ptica $E$, tomando valores en el conjunto de ra\'{\i}ces
$N$-\'{e}simas de la unidad.

Sean $E$ y $E'$ dos curvas complejas, y sean $(P,Q)$ y $(P',Q')$ pares de puntos de
orden $N$ en $E$ y en $E'$, respectivamente, tales que
$e_{N}(P,Q)=e_{N}(P',Q')=e^{2\pi i/N}=\zeta_{N}$. La dupla $E$, junto con $(P,Q)$
se dice relacionada con $E'$, con $(P',Q')$, si existe un isomorfismo
$E\xrightarrow{\sim}E'$ (definido sobre $\bb{C}$) tal que $P\mapsto P'$ y
$Q\mapsto Q'$. Esto determina una relaci\'{o}n de equivalencia, y denotamos
$S'(N)$ al conjunto que resulta de tomar las clases de equivalencia correspondientes %corrige ``clses''
en

\begin{align*}
& \left\lbrace (E,(P,Q))\,:\,E\text{ curva el\'{\i}ptica sobre }\bb{C},
e_{N}(P,Q)=\zeta_{N}\right\rbrace\text{ ,}
\end{align*}
y con $[E,(P,Q)]$ a la clase de $(E,(P,Q))$.

Sea, ahora, $\Gamma(N)$ el subgrupo principal de congruencia de nivel $N$, el
n\'{u}cleo del morfismo sobreyectivo
$\SL{2}(\bb{Z})\rightarrow\SL{2}(\bb{Z}/N\bb{Z})$ dado por reducir coordenadas
\textit{modulo} $N$. Sean $Y'(N)$ la superficie de Riemann
$\Gamma(N)\backs\frak{h}$, y $X'(N)=\Gamma(N)\backs\frak{h}^{*}$ su
compactificaci\'{o}n agregando las c\'{u}spides correspondientes a $\Gamma(N)$.
El conjunto $S'(N)$ est\'{a} dado por

\begin{align*}
& \left\lbrace [\bb{C}/\Lambda_{\tau},\,
(\tau/N+\Lambda_{\tau},1/N+\Lambda_{\tau})]\,:\,\tau\in\frak{h} \right\rbrace
\text{ ,}
\end{align*}
y dos puntos $[\bb{C}/\Lambda_{\tau},\,
(\tau/N+\Lambda_{\tau},1/N+\Lambda_{\tau})]$ y $[\bb{C}/\Lambda_{\tau'},\,
(\tau'/N+\Lambda_{\tau'},1/N+\Lambda_{\tau'})]$ son iguales, si, y s\'{o}lo si
$\Gamma(N)\tau=\Gamma(N)\tau'$. La aplicaci\'{o}n $[\bb{C}/\Lambda_{\tau},\,
(\tau/N+\Lambda_{\tau},1/N+\Lambda_{\tau})]\mapsto\Gamma(N)\tau$ da una
biyecci\'{o}n entre $S'(N)$ y la curva $Y'(N)$. Es decir, $X'(N)$ parametriza
(clases de equivalencia de) curvas el\'{\i}pticas junto con un par de generadores
de la $N$-torsi\'{o}n y con un \textit{pairing} particular. \cite{diamondShurman}

Las curvas en las que fijaremos nuestra atenci\'{o}n ser\'{a}n ciertos cocientes
de las curvas $X'(N)$. Sea $E/\bb{C}$ una curva el\'{\i}ptica, y sea $E[N]$ su
subgrupo de $N$-torsi\'{o}n. Fijar un isomorfismo
$\varphi:\,E[N]\xrightarrow{\sim}(\bb{Z}/N\bb{Z})^{2}$ equivale a dar una base de
$E[N]$, es decir, dos puntos $P$ y $Q$ de $E$ que generen $E[N]$. En este sentido,
la informaci\'{o}n relativa a la torsi\'{o}n que brinda un par $(E,\varphi)$ es
menos espec\'{\i}fica que la contenida en los pares $(E,(P,Q))$. Dados dos pares
$(E,\varphi)$ y $(E',\varphi')$, los mismos son equivalentes, si existe un
isomorfismo $E\xrightarrow{\sim}E'$ tal que $\varphi\mapsto\varphi'$, es decir,
si $\{P,Q\}$ es la base de $E[N]$ y $\{P',Q'\}$ la de $E'[N]$, $P\mapsto P'$ y
$Q\mapsto Q'$. La relaci\'{o}n es la misma; simplemente estamos permiti\'{e}ndonos
ver otros puntos que antes, en $X'(N)$, no consideramos. \'{E}sta es una
relaci\'{o}n de equivalencia, y denotamos $S(N)$ al conjunto de clases de pares
$(E,\varphi)$ (denotadas $[E,\varphi]$) por dicha relaci\'{o}n.

Sea $E/\bb{C}$ una curva el\'{\i}ptica, y sea
$\varphi:\,E[N]\rightarrow(\bb{Z}/N\bb{Z})^{2}$ una estructura de nivel. Sea
$e_{N}:\,E[N]\times E[N]\rightarrow\mu_{N}$ una forma alternada, no degenerada,
el \textit{pairing} de Weil, sea $\zeta_{N}:=e^{2\pi i/N}$ y $\zeta(\varphi)$ la
ra\'{\i}z de la unidad

\begin{align*}
\zeta(\varphi) & \,:=\,e_{N}\left(
\varphi^{-1}\left(\begin{bmatrix}1\\0\end{bmatrix}\right),\,
\varphi^{-1}\left(\begin{bmatrix}0\\1\end{bmatrix}\right)\right)
\text{ .}
\end{align*}

El par $(E,\varphi)$ es equivalente a uno de la forma
$(\bb{C}/\Lambda,\,(\tau/N+\Lambda_{\tau},1/N+\Lambda))$, $\tau\in\frak{h}$, si,
y s\'{o}lo si $\zeta(\varphi)=\zeta_{N}$.

En general, la forma alternada $e_{N}$ determina una funci\'{o}n
$S(N)\rightarrow\mu_{N}$. Haciendo de \'{e}sta una funci\'{o}n continua, vemos
que se puede obtener una superficie de Riemann compacta $X(N)$ a partir de $S(N)$,
y, como $e_{N}(aP+bQ,cP+dQ)=e_{N}(P,Q)^{\det(\gamma)}$, para
\begin{math}\gamma=
\left[\begin{smallmatrix}a&b\\c&d\end{smallmatrix}\right]\in\GL{2}(\bb{Z}/N\bb{Z})
\end{math}, la curva $X(N)$ es disconexa.
M\'{a}s precisamente, es la uni\'{o}n disjunta de $\phi(N)$ componentes. Cada una
de estas componentes es una copia de $X'(N)$ y est\'{a}n en correspondencia con
las ra\'{\i}ces primitivas $N$-\'{e}simas de $1$: si $\zeta_{N}^{k}$ es una de
ellas, la componente correspondiente es la que contiene las clases $[E,\varphi]$
con $\zeta(\varphi)=\zeta_{N}^{k}$ (secci\'{o}n 7 de \cite{booher} y
\cite{deligneRapoport}).
Para ser m\'{a}s claros, el grupo $\GL{2}(\bb{Z}/N\bb{Z})$ act\'{u}a sobre los
puntos no cuspidales de la curva compactificada $X(N)$. Esta acci\'{o}n est\'{a}
determinada por $(E,\phi)\mapsto(E,\gamma\circ\varphi)$.

Sea $H$ un subgrupo de $\GL{2}(\bb{Z}/N\bb{Z})$ tal que
$\det(H)=(\bb{Z}/N\bb{Z})^{\times}$. A trav\'{e}s del determinante,
$\GL{2}(\bb{Z}/N\bb{Z})$ permuta las componentes de la curva $X(N)$. Como la
imagen de $H$ por $\det(\cdot)$ es todo $(\bb{Z}/N\bb{Z})^{\times}$, el cociente
$H\backs X(N)$ es conexo. En el caso en que $H$ es igual a todo el grupo general
lineal, obtenemos $X(1)=\SL{2}(\bb{Z})\backs\frak{h}^{*}$.

Dado un subgrupo arbitrario $H$ de $\GL{2}(\bb{Z}/N\bb{Z})$, llamamos $H'$ a la
intersecci\'{o}n $H\cap\SL{2}(\bb{Z}/N\bb{Z})$, y $\Gamma_{H}$ a la preimagen de
$H'$ por $\SL{2}(\bb{Z})\rightarrow\SL{2}(\bb{Z}/N\bb{Z})$. Notando que
$\Gamma_{H}$ es un subgrupo de congruencia (de nivel $N$), definimos

\begin{align*}
X_{H} & \,:=\,\Gamma_{H}\backs\frak{h}^{*}\text{ .}
\end{align*}
Esta curva se identifica con el cociente $H'\backs X'(N)$, y tambi\'{e}n con
$H\backs X(N)$, si $\det(H)=(\bb{Z}/N\bb{Z})^{\times}$. En otras palabras, $X_{H}$
es el cociente de $X'(N)$ por la acci\'{o}n del subgrupo de automorfismos
determinado por $H'$.
%En particular, $H'$ act\'{u}a sobre el conjunto de c\'{u}spides de $X'(N)$,
%de manera que las c\'{u}spides de $X_{H}$ se identifican con las \'{o}rbitas
%de  esta acci\'{o}n.

Necesitaremos saber en qu\'{e} casos estas curvas est\'{a}n definidas sobre
$\bb{Q}$.

\begin{propoDefinidasSobreQ}\label{thm:propoDefinidasSobreQ}
Sea $H$ un subgrupo de $\GL{2}(\bb{Z}/N\bb{Z})$ tal que su imagen por el
determinante cubra $(\bb{Z}/N\bb{Z})^{\times}$. Entonces $X_{H}=H\backs X(N)$ es
una variedad algebraica conexa definida sobre $\bb{Q}$.
\end{propoDefinidasSobreQ}
El enunciado de esta proposici\'{o}n, como su demostraci\'{o}n se pueden encontrar
en el teorema 43 de \cite{booher}.

\begin{proof}[Demostraci\'{o}n]
Primero se definen ciertas funciones modulares para el grupo $\Gamma(N)$
\cite{diamondShurman}: sea $v=(c_{v},d_{v})$ un par donde $c_{v}$ y $d_{v}$ son
enteros y sus reducciones no son ambas divisibles por $N$, es decir,
$\overline{v}(\overline{c_{v}},\overline{d_{v}})\not =0$ en $(\bb{Z}/N\bb{Z})^{2}$.
Dado $(\bb{C}/\Lambda,(P,Q))$ con $e_{N}(P.Q)=\zeta_{N}$, definimos

\begin{align*}
F_{0}^{\overline{v}}(\bb{C}/\Lambda,(P,Q)) & \,=\,
\frac{g_{2}(\Lambda)}{g_{3}(\Lambda)}\wp_{\Lambda}(c_{v}P+d_{v}Q)\text{ ,}
\end{align*}
donde $g_{2}$ y $g_{3}$ son las funciones que se obtienen de las series de
Eisenstein $G_{4}$ y $G_{6}$, y $\wp_{\Lambda}$ es la funci\'{o}n de
Weierstra{\ss} del ret\'{\i}culo $\Lambda$. De manera equivalente, si
$\tau\in\frak{h}$, podemos definir

\begin{align*}
f_{0}^{\overline{v}}(\tau) & \,=\,
\frac{g_{2}(\tau)}{g_{3}(\tau)}\wp_{\tau}(\frac{c_{v}\tau+d_{v}}{N})
\text{ ,}
\end{align*}
y
\begin{math}
f_{0}^{\overline{v}}(\tau)=
F_{0}^{\overline{v}}(\bb{C}/\Lambda_{\tau},(\tau/N,1/N))
\end{math}.
Las funciones $F_{0}^{\overline{v}}$ no dependen del representante de la clase de
$(\bb{C}/\Lambda,(P,Q))$, y las funciones $f_{0}^{\overline{v}}$ dependen s\'{o}lo
de la \'{o}rbita $\Gamma(N)\tau\subset\frak{h}$. Se veifica que \'{e}stas
son funciones invariantes por $\Gamma(N)$ y meromorfas, tanto en $\frak{h}$, como
en las c\'{u}spides de $\Gamma(N)$. Es decir, son funciones meromorfas en $X'(N)$.

Sea $\bb{C}(X'(N))$ el cuerpo de funciones meromorfas en $X'(N)$, y sea

\begin{align*}
\theta & \,:\,\SL{2}(\bb{Z})\rightarrow\rm{Aut}(\bb{C}(X'(N)))\,|\\
 & \gamma\mapsto(
\theta(\gamma):\,f\mapsto f^{\theta(\gamma)}=f\circ\gamma)\text{ .}
\end{align*}
Esta aplicaci\'{o}n es un morfismo de grupos, y $\ker(\theta)$ contiene a
$\widetilde{\Gamma}(N):=\{\pm\}\Gamma(N)$. Notemos, por otra parte, que dos
funciones $f_{0}^{\overline{u}}$ y $f_{0}^{\overline{v}}$ son iguales, si, y
s\'{o}lo si $\overline{u}=\pm\overline{v}$, pues $\wp_{\tau}(z)=\wp_{\tau}(z')$,
si, y s\'{o}lo si $z+\Lambda_{\tau}=\pm z'+\Lambda_{\tau}$. Adem\'{a}s,
\begin{math}
(f_{0}^{\overline{v}})^{\theta(\gamma)}=
f_{0}^{\overline{v}}\circ\gamma=f_{0}^{\overline{v\gamma}}
\end{math},
con lo que, como $f_{0}^{\overline{v}}\in\bb{C}(X'(N))$, el n\'{u}cleo
$\ker(\theta)$ est\'{a} contenido en $\widetilde{\Gamma}(N)$.

Ahora, $\theta(\SL{2}(\bb{Z}))$ es un subgrupo del grupo de automorfismos del
cuerpo $\bb{C}(X'(N))$, y su cuerpo fijo es $\bb{C}(X(1))=\bb{C}(j)$. En
definitiva, la extensi\'{o}n $\bb{C}(X'(N))/\bb{C}(X(1))$ es galoisiana con
grupo de Galois

\begin{align*}
\theta(\SL{2}(\bb{Z}))& \,\simeq\,\SL{2}(\bb{Z})/\widetilde{\Gamma}(N)
\,\simeq\,\SL{2}(\bb{Z}/N\bb{Z})/\{\pm 1\}\text{ .}
\end{align*}
El mismo argumento que los \'{u}nicos elementos de $\SL{2}(\bb{Z})$ que act\'{u}an
trivialmente sobre
\begin{math}
\bb{C}(j,\{f_{0}^{\pm\overline{v}}\,:\,
\pm\overline{v}\in((\bb{Z}/N\bb{Z})^{2}\setmin\{(0,0)\})/\{\pm 1\}\})
\end{math}
son los pertenecientes a $\widetilde{\Gamma}(N)$. Lo mismo es cierto para el
cuerpo $\bb{C}(j,f_{0}^{\pm\overline{(0,1)}},f_{0}^{\pm\overline{(1,0)}})$, que
coincide, entonces, con $\bb{C}(X'(N))$.

El siguiente paso consiste en demostrar que la extensi\'{o}n
$\bb{Q}(\mu_{N},j,\{f_{0}^{\pm\overline{v}}\})/\bb{Q}(j)$ tambi\'{e}n es Galois y
que su grupo se identifica con $\GL{2}(\bb{Z}/N\bb{Z})/\{\pm 1\}$. Si $\sigma$ es
un elemento del grupo de Galois de esta extensi\'{o}n, entonces

\begin{align*}
 \begin{bmatrix}
  (f_{0}^{\pm(1,0)})^{\sigma}\\
  (f_{0}^{\pm(0,1)})^{\sigma}
 \end{bmatrix} & \,=\,\rho(\sigma)
 \begin{bmatrix} f_{0}^{\pm(1,0)}\\f_{0}^{\pm(0,1)} \end{bmatrix}
\end{align*}
para alguna transformaci\'{o}n lineal $\rho(\sigma)\in\GL{2}(\bb{Z}/N\bb{Z})$.
Si $\mu=e_{N}(P,Q)$ es una ra\'{\i}z $N$-\'{e}sima primitiva de la unidad,

\begin{align*}
 \mu^{\sigma} &\,=\,\sigma(e_{N}(P,Q))\,=\,e_{N}(P^{\sigma},Q^{\sigma})\\
 &\,=\,e_{N}(P,Q)^{\det(\rho(\sigma))}\,=\,\mu^{\det(\rho(\sigma))}\text{ .}
\end{align*}

Finalmente, si $H\subset\GL{2}(\bb{Z}/N\bb{Z})$ es un subgrupo que satisface
$\det(H)=(\bb{Z}/N\bb{Z})^{\times}$, definimos los cuerpos

\begin{align*}
K_{1} & \,=\,\bb{Q}(\mu_{N},j,\{f_{0}^{\pm\overline{v}}\})^{\Gamma_{H}}\\
K_{2} & \,=\,\bb{C}(j,\{f_{0}^{\pm\overline{v}}\})^{H'/\{\pm 1\}}\text{ .}
\end{align*}
Un poco de Teor\'{\i}a de Galois muestra que
$K_{1}=\bb{Q}(j,\{f_{0}^{\pm\overline{v}}\})^{H'/\{\pm 1\}}$ y que
$K_{1}\cap\overline{\bb{Q}}=\bb{Q}$, correspondi\'{e}ndose con una curva
poryectiva, no singular, y esta curva est\'{a} definida sobre $\bb{Q}$.
Con respecto a $K_{2}$, este cuerpo tiene que ser el cuerpo de funciones del
cociente $H'\backs X'(N)$, ya que, para $f\in\bb{C}(X'(N))$,
$f^{\theta(\gamma)}=f\circ\gamma$. Ahora bien, el cuerpo de funciones de la
curva definida sobre los n\'{u}meros complejos que se obtiene a partir de las
ecuaciones que definen la curva correspondiente al cuerpo funcional $K_{1}$,
coincide con $K_{2}$. Es decir, $H\backs X(N)$ es isomorfa sobre $\bb{C}$ a una
curva definida sobre $\bb{Q}$. Usaremos la misma notaci\'{o}n para referirnos
tanto a $H\backs X(N)$ como a su modelo sobre $\bb{Q}$.

\end{proof}

%-----------------
\begin{subsection}{Subgrupos de Cartan}\label{subsec:subgruposDeCartan}
Sea $E$ una curva el\'{\i}ptica con CM. Su anillo de endomorfismos es un orden
$I=[1,\tau]$ en un cuerpo cuadr\'{a}tico imaginario. Si $N>1$ es un entero coprimo
con el discriminante de $I$ y $p\in\bb{Z}$ es un primo divisor de $N$, entonces
$p$ se parte o es inerte en $I$. Sea $A:=I/NI$. Este anillo es una
$(\bb{Z}/N\bb{Z})$-\'{a}lgebra con base $\{1,\tau\}$, y, dependiendo de $p$, si se
parte o no en $I$, el cociente $A/pA$ es isomorfo a $\bb{F}_{p}\times\bb{F}_{p}$
o a $\bb{F}_{p^{2}}$, respectivamente.

\begin{defSubgrupoDeCartan}\label{thm:defSubgrupoDeCartan}
Dada $A$ un \'{a}lgeba sobre $\bb{Z}/N\bb{Z}$, libre, conmutativa, de rango $2$ y
tal que, si $p|N$, entonces $A/pA$ es isomorfo a $\bb{F}_{p}\times\bb{F}_{p}$ o a
$\bb{F}_{p^{2}}$, el grupo de unidades, $A^{\times}$, act\'{u}a sobre $A$ por
multiplicaci\'{o}n. Elegir una base de $A$ sobre $\bb{Z}/N\bb{Z}$, equivale a
establecer un morfismo

\begin{align*}
\iota & \,:\,A^{\times}\hookrightarrow\GL{2}(\bb{Z}/N\bb{Z})
\end{align*}
--elegir otra base resulta en un subgrupo conjugado a $\iota(A^{\times})$.
Decimos que un subgrupo de $\GL{2}(\bb{Z}/N\bb{Z})$ es \emph{de Cartan}, si es de
la forma $\iota(A^{\times})$ para alg\'{u}n \'{a}lgebra $A$ que cumpla con las
condiciones especificadas. Con respecto a la \'{u}ltima de estas condiciones,
tambi\'{e}n se dice que $A$ es \textit{\'{e}tale}. Si $A/pA\simeq\bb{F}_{p^{2}}$,
se dice que $A$ es \textit{non-split} en $p$, y \textit{split} en el otro caso.
Un subgrupo de Cartan se dice \textit{non-split}, si el \'{a}lgebra correspondiente
lo es en todo primo que divide a $N$.

\end{defSubgrupoDeCartan}

\begin{ejemploSubgrupoDeCartan}\label{thm:ejemploSubgrupoDeCartan}
Si $I$ es un orden en un cuerpo cuadr\'{a}tico imaginario, $A=I/NI$ y
$N$ es coprimo con $\rm{disc}(I)$ y todo divisor primo de $N$ es inerte en $I$, %correcci\'{o}n
entonces si $A$ es \textit{non-split} y, %correcci\'{o}n
fijando una base, $\iota(A^{\times})\subset\GL{2}(\bb{Z}/N\bb{Z})$ es un subgrupo
de Cartan \textit{non-split}. Denotamos con $C_{ns}(N)$ al subgrupo
$\iota(A^{\times})$.

Si $N=p$ es  primo, $|A^{\times}|=|\bb{F}_{p^{2}}^{\times}|=p^{2}-1$.
Si $N=p^{r}$, $r\geq 1$, todo elemento de $(I/pI)^{\times}$ se levanta a una
unidad en $I/p^{r}I$, y cada elemento tiene $p^{2(r-1)}$ preim\'{a}genes;
entonces

\begin{align*}
|(I/p^{r}I)^{\times}| & \,=\, p^{2(r-1)}(p^{2}-1)\text{ .}
\end{align*}
Para $N>1$ entero coprimo con $\rm{disc}(I)$, el orden $|A^{\times}|$ es igual a

\begin{align*}
|(I/NI)^{\times}|=N^{2}\prod_{p|N}\,(1-(1/p^{2}))\text{ .}
\end{align*}
Este es el orden del grupo de Cartan \textit{non-split} $C_{ns}(N)$.
%
%\begin{align*}
%a+b\tau\in(I/pI)^{\times}\Rightarrow\exists c,d
%\,|\,(a+b\tau)(-c+d\tau)\equiv 1\,(pI)
%a,b,c,d<p
%-ac-v=1+kp & \text{ tomar } c=c+sp\text{ para que } -ac-v\equiv 1\,(p^{r})\\
%ad-bc+u=k'p\text{ tomar } d=d+s'p\text{ para que } ad-bc+u\equiv 0\,(p^{r})
%\end{align*}
%cada $a+b\tau$ se levanta a $I/p^{r}$ de $p^{2(r-1)}$ maneras.

\end{ejemploSubgrupoDeCartan}

\end{subsection}

\begin{subsection}{El normalizador de $C_{ns}(N)$ en $\GL{2}(\bb{Z}/N\bb{Z})$ %
		y la curva modular asociada}\label{subsec:normalizadorYCurva}
En la descripci\'{o}n que hacemos a continuaci\'{o}n tomamos como referencia a
\cite{baranNormalizers}.
Sean $I=[1,\tau]$ un orden en un cuerpo cuadr\'{a}tico imaginario $K$ tal que
$(N,\rm{disc}(I))=1$ y todo primo que divide a $N$ es inerte en $I$, y sea %correcci\'{o}n
$A=I/NI$. Sea $C_{ns}(N)=\iota(A^{\times})$. Si $\tau$ %correcci\'{o}n
satisface el polinomio $X^{2}-uX+v\in\bb{Z}[X]$, definimos una involuci\'{o}n
en $\tau$ por $\overline{\cdot}:\,\tau\mapsto(u-\tau)$, y tambi\'{e}n
$\sigma_{p}:\,A\rightarrow A$ como el \'{u}nico automorfismo en $A=I/NI$ que
coincide con esta involuci\'{o}n en $I/p^{r(p)}I$ y con la identidad en
$I/\frac{N}{p^{r(p)}}I$, donde $r(p)$ es la m\'{a}xima potencia de $p$ que divide
a $N$. La misma elecci\'{o}n de base que determina la
inclusi\'{o}n $\iota$ determina una matriz $S_{p}\in\GL{2}(\bb{Z}/N\bb{Z})$
que representa a $\sigma_{p}$.

Sea $N=p^{r}$, $p\in\bb{Z}$ un primo y $r\geq 1$. Podemos identificar $C_{ns}(N)$
con el grupo $A^{\times}$ de unidades de $A=I/p^{r}I$. Si $\alpha$ es un elemento
de $A$, tiene una matriz asociada, en $\MM{2}(\bb{Z})$:
\begin{math}
 \gamma(\alpha):=
 \left[\begin{smallmatrix}a&b\\c&d\end{smallmatrix}\right]
\end{math},
determinada por

\begin{align*}
 \alpha\cdot\begin{bmatrix} 1\\ \tau \end{bmatrix} & \,=\,
 \begin{bmatrix} \alpha\\ \alpha\tau \end{bmatrix} \,=\,
 \begin{bmatrix} a+b\tau\\ c+d\tau \end{bmatrix} \,=\,
 \begin{bmatrix} a&b\\c&d \end{bmatrix}
 \begin{bmatrix} 1\\ \tau \end{bmatrix}\text{ .}
\end{align*}
Si $\tau^{2}-u\tau+v=0$, con $u$ y $v$ en $\bb{Z}$, la matriz de $\tau$ es
\begin{math}
 \gamma(\tau):=
 \left[\begin{smallmatrix}0&1\\-v&u\end{smallmatrix}\right]
\end{math}.
Como $A$ es \textit{non-split}, $A/pA\simeq\bb{F}_{p^{2}}$. Pero $\tau\not\in pA$,
%$#\bb{F}_{p^{2}}=p^{2}$. Si $\tau\in pA$, $A/pA$ est\'{a} generado por $1$
%y $\tau$ sobre $\bb{Z}/p\bb{Z}$ implica cardinalidad menor.
con lo que $\tau$ es una unidad en $I/pI$, es decir que la matriz $\gamma(\tau)$
es una unidad en $\MM{2}(\bb{Z}/p\bb{Z})$. Entonces, $v=\det(\gamma(\tau))$
es una unidad en $\bb{Z}/p\bb{Z}$. En particular, $p$ no divide a $v$, y
$\det(\gamma(\tau))\in(\bb{Z}/p^{r}\bb{Z})^{\times}$. En otras palabras,
$\tau$ es una unidad en $A$.

Si ahora tomamos $\kappa\in\GL{2}(\bb{Z}/p^{r}\bb{Z})$ tal que
$\kappa C_{ns}(p^{r})=C_{ns}(p^{r})\kappa$, $\kappa$ induce un automorfismo de
$A^{\times}$ por conugaci\'{o}n: si $\alpha\in A^{\times}$, definimos
$t_{\kappa}(\alpha)$ como la primera coordenada de

\begin{align*}
 & \kappa\gamma(\alpha)\kappa^{-1}\cdot\begin{bmatrix}1\\ \tau\end{bmatrix}
 \text{ .}
\end{align*}
Si $n\in\bb{Z}$ es coprimo con $p$, $n\cdot 1_{A}\in A^{\times}$ y
$\gamma(n1_{A})$ es la matriz diagonal
\begin{math}
\left[\begin{smallmatrix} n&0\\0&n\end{smallmatrix}\right]
\end{math}.
As\'{\i}, $t_{\kappa}(n1_{A})=n1_{A}$. Extendemos $t_{\kappa}$ a un endomorfismo
de la $(\bb{Z}/p^{r}\bb{Z})$-\'{a}lgebra $A$. En particular,
$t_{\kappa}(\tau)\in A$ tiene que ser un cero de $f=X^{2}-uX+v$. Las soluciones
$\tau$ y $u-\tau$ de $f$ son distintas en $I/pI\simeq\bb{F}_{p^{2}}$ y
$f'(\tau)=\tau-(u-\tau)\not =0$, entonces,
si $\alpha\in I$ es soluci\'{o}n de $f$ \textit{modulo} $p^{r'}I$ para alg\'{u}n
$r'$, por el argumento del lema de Hensel, se levanta a \'{u}nica soluci\'{o}n
\textit{modulo} $p^{r'+1}I$. Entonces $t_{\kappa}(\tau)=\tau$ o $u-\tau$, con lo
que el automorfismo $t_{\kappa}$ de $A$ es, o bien, $\sigma_{p}$, o bien la
identidad. En t\'{e}rminos de matrices, o bien $\kappa z\kappa^{-1}=z$, o bien
$\kappa z\kappa^{-1}=S_{p}zS_{p}$ ($S_{p}$ tiene orden $2$). O bien $\kappa$,
o bien $S_{p}\kappa$, pertenece al centralizador de $C_{ns}(p^{r})$. Pero el
centralizador de $C_{ns}(p^{r})$ es el mismo grupo. As\'{\i}
$\kappa$ pertenece a $C_{ns}(p^{r})$, o a $S_{p}C_{ns}(p^{r})$, es decir que el
normalizador de $C_{ns}(p^{r})$ es el subgrupo de $\GL{2}(\bb{Z}/p^{r}\bb{Z})$
generado por $C_{ns}(p^{r})$ y por $S_{p}$. En general, para $N>1$,
%por el teorema chino del resto,
el normalizador de $C_{ns}(N)$ es $\langle C_{ns}(N),\{S_{p}\,:\,p|N\} \rangle$.
Lo denotamos $C_{ns}^{+}(N)$. Su orden es

\begin{align*}
N^{2}2^{\omega}\prod_{p|N}\,\left(1-\frac{1}{p^{2}}\right) &
\,=\,|A^{\times}|2^{\omega}\text{ ,}
\end{align*}
donde $\omega$ es la cantidad de primos distintos que dividen a $N$.

Los grupos $C_{ns}(N)$ y $C_{ns}^{+}(N)$ son subgrupos de $\GL{2}(\bb{Z}/N\bb{Z})$,
y, como tales, podemos asociarles las curvas modulares

\begin{align*}
 X_{ns}(N) & \,:=\,C_{ns}(N)\backs X(N)\\
 X_{ns}^{+}(N) &\,:=\,C_{ns}^{+}(N)\backs X(N)\text{ .}
\end{align*}
Llamamos $Y_{ns}(N)$ e $Y_{ns}^{+}(N)$ a los abiertos que son complementos de las
c\'{u}spides de $X_{ns}(N)$ y $X_{ns}^{+}(N)$, respectivamente.

La curva $X_{ns}^{+}(N)$ est\'{a} definida sobre $\bb{Q}$:
$\det(C_{ns}^{+}(N))=(\bb{Z}/N\bb{Z})^{\times}$ \cite{booher}, el argumento es el
siguiente: sean $m\in\bb{Z}$ coprimo con $N$, y $p|N$ un primo. Como $p$ se
supone inerte en $K$ (el cuerpo cuadr\'{a}tico que contiene al orden
$I=[1,\tau]$), $K_{p}/\bb{Q}_{p}$ es una extensi\'{o}n cuadr\'{a}tica de cuerpos
locales. Existe, entonces, $\mu_{p}\in K_{p}^{\times}$ \'{\i}ntegro tal que

\begin{align*}
 \Nm_{K_{p}/\bb{Q}_{p}}(\mu_{p})& \,=\,\pm m\text{ .}
\end{align*}
Si $p^{r(p)}||N$, elegimos $\widetilde{\mu}_{p}\in I$ que aproxime $\mu_{p}$ a
orden $p^{r(p)}$:

\begin{align*}
 \mu_{p} & \,\equiv\,\widetilde{\mu}_{p}\,(p^{r(p)})\text{ .}
\end{align*}
As\'{\i}, $\Nm_{K_{p}/\bb{Q}_{p}}(\widetilde{\mu}_{p})$ es congruente con
$\pm m\,(p^{r(p)})$. Haciendo esto para cada primo que divide a $N$, se toma
$\mu\in I$ que satisfaga $\mu\equiv\widetilde{\mu}_{p}$ \textit{modulo} $p^{r(p)}$
para cada $p$. De esta manera, $\Nm_{K/\bb{Q}}(\mu)\equiv \pm m\,(p^{r(p)})$ para
cada primo. Corrigiendo con las matrices $S_{p}$ (cuyo determinante es $-1$ en el
factor correspondiente a $p$, y $1$ en el resto), vemos que
$\det:\,C_{ns}^{+}(N)\rightarrow(\bb{Z}/N\bb{Z})^{\times}$ es sobreyectiva.

\begin{obsDetSobreUnidades}\label{thm:obsDetSobreUnidades}
 Hemos demostrado, tambi\'{e}n, que

 \begin{align*}
  \det & \,:\,C_{ns}(p^{r})\rightarrow(\bb{Z}/p^{r}\bb{Z})^{\times}/\{\pm 1\}
 \end{align*}
es sobre.

\end{obsDetSobreUnidades}

Sean $C_{ns}^{+}(N)'=C_{ns}^{+}(N)\cap\SL{2}(\bb{Z}/N\bb{Z})$ y
$\Gamma_{C_{ns}^{+}(N)}$ el subgrupo de $\SL{2}(\bb{Z})$ conformado por las
matrices que caen en $C_{ns}^{+}(N)'$ al reducir las coordenadas \textit{modulo}
$N$. Como $\det:\,C_{ns}^{+}(N)\rightarrow(\bb{Z}/N\bb{Z})^{\times}$ es
sobreyectiva y su n\'{u}cleo es $C_{ns}^{+}(N)'$, el \'{\i}ndice
$|C_{ns}^{+}(N):C_{ns}^{+}(N)'|$ es igual a $\phi(N)$, con lo cual, 

\begin{align*}
 |\SL{2}(\bb{Z}):\Gamma_{C_{ns}^{+}(N)}| & \,=\,
 |\SL{2}(\bb{Z}/N\bb{Z}):C_{ns}^{+}(N)'| \\
 & \,=\,\frac{\phi(N)}{2^{\omega}N^{2}\prod_{p|N}\,(1-(1/p^{2}))}
 N^{3}\prod_{p|N}\,(1-(1/p^{2})) \,=\,\frac{N\phi(N)}{2^{\omega}}
 \text{ .}
\end{align*}
Tenemos morfismos

\begin{align*}
 & X_{ns}(N)\xrightarrow{\Phi_{1}}
 X_{ns}^{+}(N)\xrightarrow{\Phi_{2}} X(1)\text{ ,}
\end{align*}
de grados $\rm{deg}(\Phi_{1})=2^{\omega}$ y
$\rm{deg}(\Phi_{2})=N\phi(N)/2^{\omega}$.

En resumen, existe una curva modular, $X_{ns}^{+}(N)$, isomorfa a
$\Gamma_{C_{ns}^{+}(N)}\backs\frak{h}^{*}$, y definida sobre $\bb{Q}$.
El abierto
$Y_{ns}^{+}(N)=X_{ns}^{+}(N)\setmin\{\text{ c\'{u}spides }\}$
parametriza clases de equivalencia de pares $(E,\varphi)$, donde
$E/\bb{C}$ es una curva el\'{\i}ptica y
$\varphi:\,E[N]\rightarrow(\bb{Z}/N\bb{Z})^{2}$ un isomorfismo; dos pares
$(E,\varphi)$ y $(E',\varphi')$ son equivalentes, si existe
$\gamma\in C_{ns}^{+}(N)$ tal que los puntos $[E,\varphi]$ y
$[E',\gamma\circ\varphi']$ sean iguales en $X(N)$. Olvidando la estructura de
nivel, $\varphi$ en cada par $(E,\varphi)$, resulta un morfismo de $X_{ns}^{+}(N)$
en $X(1)$ de grado $N\phi(N)/2^{\omega}$.
%Si llamamos $g$ al g\'{e}nero de esta curva,
%
%\begin{align*}
% g & \,=\,1\,+\,\frac{\mu}{12}\,-\,\frac{\mu_{2}}{4}\,-\,\frac{\mu_{3}}{6}
% \,-\,\frac{\mu_{\infty}}{2}\text{ ,}
%\end{align*}
%donde $\mu:=|\SL{2}(\bb{Z}):\Gamma_{C_{ns}^{+}(N)}|$, $\mu_{k}$ es la cantidad
%de puntos el\'{\i}pticos de orden $k=2,3$ y $\mu_{\infty}$ es la cantidad de
%c\'{u}spides.

\end{subsection}

\begin{subsection}{Puntos racionales e \'{\i}ntegros}\label{subsec:puntosRacionalesYEnteros}
 En tanto $X_{ns}^{+}(N)$ tiene un modelo sobre $\bb{Q}$, podemos referirnos a sus
 puntos racionales y a sus puntos enteros. Para $N=5$ o $N\geq 7$, los puntos
 $\bb{Q}$-racionales de $X_{ns}^{+}(N)$ pertenecen a $Y_{ns}^{+}(N)$, es decir,
 no est\'{a}n entre las c\'{u}spides \cite{baranNormalizers}. Recordemos que
 existe un morfismo $X(N)\rightarrow X_{ns}^{+}(N)$ que nos permite ver
 $X_{ns}^{+}(N)$ como un cociente de $X(N)$ y sus c\'{u}spides como \'{o}rbitas
 por la acci\'{o}n de $C_{ns}^{+}(N)$ sobre las c\'{u}spides de $X(N)$. Si
 $[E,\varphi]\in Y(N)(\overline{\bb{Q}})$ y $\sigma\in\absGal{\bb{Q}}$,
 
 \begin{align*}
  [E,\varphi]^{\sigma} & \,=\,[E^{\sigma},\varphi\circ\sigma]
 \end{align*}
determina una acci\'{o}n del grupo de Galois absoluto sobre los puntos
$\overline{\bb{Q}}$-racionales en el abierto $Y(N)$. Adem\'{a}s, si $E$ es una
curva el\'{\i}ptica definida sobre $\bb{Q}$, $\absGal{\bb{Q}}$ act\'{u}a sobre
$E[N]$, dando lugar a una representaci\'{o}n

 \begin{align*}
  \rho_{N} & \,:\,\absGal{\bb{Q}}\rightarrow\rm{Aut}(E[N])
  \,\simeq\,\GL{2}(\bb{Z}/N\bb{Z})\text{ ,}
 \end{align*}
donde $\rm{Aut}(E[N])$ se identifica con $\GL{2}(\bb{Z}/N\bb{Z})$ v\'{\i}a
el isomorfismo $\varphi$ en el par $[E,\varphi]$.

 Sea $[E,\varphi]\in Y(N)(\overline{\bb{Q}})$ y $\overline{[E,\varphi]}$ su
 imagen en $Y_{ns}^{+}(N)(\overline{\bb{Q}})$. Este punto es la \'{o}rbita
 de $[E,\varphi]$ por $C_{ns}^{+}(N)$:

 \begin{align*}
  \overline{[E,\varphi]} & \,=\,\left\lbrace
  [E,\gamma\circ\varphi]\,:\,\gamma\in C_{ns}^{+}(N)\right\rbrace
  \text{ .}
 \end{align*}
Y esta \'{o}rbita define un punto $\bb{Q}$-racional en $Y_{ns}^{+}(N)$, si,
y s\'{o}lo si es estable por la acci\'{o}n de Galois: si para $\sigma$ en
$\absGal{\bb{Q}}$ existe $\gamma_{\sigma}$ en $C_{ns}^{+}(N)$ tal que
$[E^{\sigma},\varphi\circ\sigma]=[E,\gamma_{\sigma}\circ\varphi]$. Para $E$
definida sobre $\bb{Q}$, esto equivale a que la imagen de Galois por la
representaci\'{o}n $\rho_{N}$ est\'{e} contenida en el normalizador de un
subgrupo de Cartan \textit{non-split}.

 Sea $I$ un orden en un cuerpo cuadr\'{a}tico imaginario $K$, sea $\Cl(I)$ su
 grupo de clases y sea $h(I)=\#\Cl(I)$. Sea $\rm{CM}(I)$ el conjunto de curvas
 el\'{\i}pticas definidas sobre $\bb{C}$ con anillo de endomorfismos $I$,
 salvo $\bb{C}$-isomorfismo, y sea $\psi:\,\Cl(I)\rightarrow\rm{CM}(I)$ la
 biyecci\'{o}n dada por $M\mapsto\bb{C}/M$ sobre un $I$-ideal $M$ en $K$
 \cite{cox}.
%
 Si $h(I)=1$, hay una \'{u}nica curva el\'{\i}ptica con CM por $I$, salvo
 isomorfismo, y, en consecuencia, el orden $I$ determina un punto
 $j(I)=j(E)\in X(1)$, donde $E$ es cualquier representante de $\rm{CM}(I)$.
 Dado que $j(E^{\sigma})=j(E)^{\sigma}$ para $\sigma\in\rm{Aut}(\bb{C}/\bb{Q})$,
 y que $E^{\sigma}$ tambi\'{e}n es CM con anillo de endomorfismos $I$,
 $j(E)^{\sigma}=j(E)$, y $j(E)$ queda fijo por $\rm{Aut}(\bb{C}/\bb{Q})$. Al ser
 racional, pertenece a $\bb{Z}$ (el $j$-invariante de una curva el\'{\i}ptica
 con multiplicaci\'{o}n compleja es un entero algebraico).

 Sea $K$ un cuerpo cuadr\'{a}tico imaginario con n\'{u}mero de clases $1$, y sea
 $\cal{O}_{K}$ su anillo de enteros. Fijemos $E$ una curva el\'{\i}ptica definida
 sobre $\bb{Q}$, con CM por $\cal{O}_{K}$ y con $j$-invariante
 $j(E)=j(\cal{O}_{K})\in\bb{Z}$. Fijamos, tambi\'{e}n, un entero $N\geq1$ coprimo
 con el discriminante de $\cal{O}_{K}$, y un isomorfismo
 $\varphi:\,E[N]\rightarrow(\bb{Z}/N\bb{Z})^{2}$. Sea $\rho_{N}$ la
 representaci\'{o}n de $\absGal{\bb{Q}}$ en $\GL{2}(\bb{Z}/N\bb{Z})$
 correspondiente a $\varphi$, es decir, identificando $\GL{2}(\bb{Z}/N\bb{Z})$
 con $\rm{Aut}(E[N])$ v\'{\i}a $\varphi$. Por restricci\'{o}n, contamos con un
 morfismo de anillos
 $f:\,\rm{End}(E)=\cal{O}_{K}\rightarrow\rm{End}(E[N])$ que se factoriza por
 $N\cal{O}_{K}$,

 \begin{align*}
  f' & \,:\,A\,=\,\cal{O}_{K}/N\cal{O}_{K}\rightarrow\rm{End}(E[N])\text{ .}
 \end{align*}
V\'{\i}a $f'$ y $\varphi$, obtenemos el grupo de Cartan \textit{non-split}
$f'(A^{\times})\subset\rm{Aut}(E[N])=\GL{2}(\bb{Z}/N\bb{Z})$. Llamemos
$C_{ns}(N)$, $G$ y $C$ a los subgrupos de $\GL{2}(\bb{Z}/N\bb{Z})$
$f'(A^{\times})$, $\rho_{N}(\absGal{\bb{Q}})$ y
$\rho_{N}(\Gal(\overline{\bb{Q}}/K)$, respectivamente. Si $\tau\in\cal{O}_{K}$,
$\sigma\in\absGal{\bb{Q}}$ y $w\in E[N]$, o bien $\tau=\tau^{\sigma}$, o bien
$\tau\not =\tau^{\sigma}$. En el primer caso,
$\tau\cdot w^{\sigma}=(\tau w)^{\sigma}$ y $\rho_{N}(\sigma)$ conmuta con
$f'(\tau)$. En el segundo, $(\tau w^{\sigma})^{\sigma{-1}}=\tau^{\sigma^{-1}}w$
y, como $\tau^{\sigma^{-1}}$ pertenece a $\cal{O}_{K}$ (la extensi\'{o}n
$K/\bb{Q}$ es normal), $\rho_{N}(\sigma)$ pertenece al normalizador de
$f'(A^{\times})$. En definitiva, $C\subset C_{ns}(N)$ y
$G\subset C_{ns}^{+}(N)$.

\begin{propoPuntosEnterosCuerpoQuad}\label{thm:propoPuntosEnterosCuerpoQuad}
 Sea $K$ un cuerpo cuadr\'{a}tico imaginario y sea $N\geq 1$ un entero tal que
 todo primo $p|N$ es inerte en $K$. Si el n\'{u}mero de clases de $K$ es igual 
 a $1$, cualquier curva el\'{\i}ptica con CM por $K$ da lugar a un punto
 $\overline{[E,\varphi]}\in Y_{ns}^{+}(N)(\overline{\bb{Q}})$, donde $E$ est\'{a}
 definida sobre $\bb{Q}$ y la imagen de la representaci\'{o}n de Galois
 asociada, $\rho_{N}$, est\'{a} contenida en $C_{ns}^{+}(N)$. En particular,
 $\overline{[E,\varphi]}$ es un punto $\bb{Q}$-racional y $j(E)$ es \'{\i}ntegro.
% Este punto es \'{u}nico.
\end{propoPuntosEnterosCuerpoQuad}
De un punto $\overline{[E,\varphi]}$ en $Y_{ns}^{+}(N)(\overline{\bb{Q}})$ tal que
$\rho_{N}(\absGal{\bb{Q}})\subset C_{ns}^{+}(N)$ y $E$ definida sobre $\bb{Q}$
(equivalentemente, $\bb{Q}$-racional), y tal que $j(E)\in\bb{Z}$, se dice que es
un punto \'{\i}ntegro de $Y_{ns}^{+}(N)$ \cite{serre}.

\end{subsection}


 \begin{subsection}{Parametrizaciones}\label{subsec:params}
  Dado un cuerpo cuadr\'{a}tico imaginario $K$ y $N\geq 1$ entero tal que sus
divisores primos sean inertes en $K$, se obtiene, a partir de una curva
el\'{\i}ptica definida sobre $\bb{Q}$ con CM por el orden $\cal{O}_{K}$, un
\'{u}nico punto \'{\i}ntegro en $Y_{ns}^{+}(N)$. El m\'{e}todo para resolver el
problema del n\'{u}mero de clases $1$ usando las curvas modulares correspondientes
a grupos de Cartan \textit{non-split} consiste en hallar los puntos enteros de
$X_{ns}^{+}(N)$. Para conseguir este objetivo es crucial elegir las curvas de
manera adecuada: de forma que el g\'{e}nero y n\'{u}mero de c\'{u}pides sean
suficientemente peque\~{n}os para poder hallar una parametrizaci\'{o}n de
$X_{ns}^{+}(N)$ y suficientemente grandes para permitir s\'{o}lo una cantidad
finita de puntos enteros.

Sea $X$ una curva proyectiva, no singular, definida sobre $\bb{Q}$ y de g\'{e}nero
$0$. Para fijar ideas, y porque este es el caso que nos interesar\'{a}, sea $X$
una de las curvas $X_{ns}^{+}(N)$ (si bien no todas ellas tienen g\'{e}nero %agrega comentario
$0$). Si $X$ tiene, al menos, un punto %acerca del g\'{e}nero
$\bb{Q}$-racional, existe un isomorfismo definido sobre $\bb{Q}$

\begin{align*}
 t & \,:\,X\rightarrow\bb{P}^{1}
 \text{ ,}
\end{align*}
\'{u}nico salvo $\bb{Q}$-automorfismo.

Sean $\widetilde{X}$ y $X$ dos curvas %correcci\'{o}n
proyectivas, no singulares, definidas sobre $\bb{Q}$ y cada una con, al menos,
un punto $\bb{Q}$-racional. Asumamos que existe un morfismo
$\pi:\,X\rightarrow\widetilde{X}$ definido sobre $\bb{Q}$ tal que
$\pi(X)=\widetilde{X}$ (por ejemplo este morfismo podr\'{\i}a ser la
proyecci\'{o}n $X_{ns}^{+}(N)\rightarrow X(1)$). Existe un diagrama

\begin{center}
\begin{tikzcd}
 X\arrow{r}{u}\arrow{d}{\pi} & \bb{P}^{1}\arrow{d}{\phi_{\pi}}\\
 \widetilde{X}\arrow{r}{j} & \bb{P}^{1}
\end{tikzcd} .
\end{center}
Ya sea describiendo el morfismo $\phi_{\pi}$ en coordenadas afines, o
identificando los cuerpos de funciones de las curvas con $\bb{Q}(u)$ y
con $\bb{Q}(j)$, obtenemos polinomios m\'{o}nicos $P,Q$ con coeficientes en el
cuerpo $\bb{Q}$ y una constante $\lambda$, tambi\'{e}n racional, tales que,

\begin{align*}
 \pi^{*}j & \,=\,\lambda\frac{P(u)}{Q(u)}\text{ ,}
\end{align*}
donde $\pi^{*}j$ es el \textit{pullback} de $j$ por $\pi$. Por
\emph{parametrizaci\'{o}n} de una curva $X$ como arriba, nos referiremos a una
elecci\'{o}n de morfismos $u,j$, junto con su correspondiente relaci\'{o}n en
t\'{e}rminos de $P$, $Q$ y $\lambda$. Los morfismos como $u$ y $j$ reciben el
nombre de \emph{uniformizadores}.

Cambiemos, por un momento, el cuerpo de base, $\bb{Q}$, por $\bb{C}$. Podemos
repetir el mismo argumento de arriba para obtener, una vez elegidos los
uniformizadores (ahora definidos sobre los complejos), una relaci\'{o}n
$\pi^{*}j=\lambda P(u)/Q(u)$ (sobre $\bb{C}$).
Supongamos, ahora, que $X=X_{ns}^{+}(N)$ (o la compactificaci\'{o}n de
cualquier cociente de $\frak{h}$ por un subgrupo de congruencia) y que
$\widetilde{X}=X(1)=\SL{2}(\bb{Z})\backs\frak{h}^{*}$. La
funci\'{o}n $j$, el $j$-invariante, es un ejemplo de uniformizador para la
curva modular $X(1)$. Pero, adem\'{a}s, en esta situaci\'{o}n, un uniformizador
se identifica con una funci\'{o}n definida en $\frak{h}$, meromorfa en $\frak{h}$,
invariante por cierto subgrupo de congruencia y meromorfa en las c\'{u}spides.
As\'{\i}, la funci\'{o}n $\pi^{*}j$ no es m\'{a}s que la misma funci\'{o}n $j$,
pero considerada, no como una funci\'{o}n modular para $\SL{2}(\bb{Z})$, sino
considerada como funci\'{o}n modular para un subgrupo de congruencia.

Sean $X=X_{ns}^{+}(N)$ para cierto $N\geq 1$ entero, y
$\widetilde{X}=X(1)$. Asumimos que el conjunto de puntos $\bb{Q}$-racionales
$X(\bb{Q})$ es no vac\'{\i}o. Elegimos el $j$-invariante como uniformizador para
$X(1)$. Este uniformizador est\'{a} definido sobre $\bb{Q}$. Es decir, los puntos
$\rho =e^{2\pi i/3}$, $i=\sqrt{-1}$, $\infty$ de $X(1)$ son $\bb{Q}$-racionales, y
el $j$-invariante --que, a cada clase de isomorfismo de curvas el\'{\i}pticas le
asigna el correspondiente invariante-- coincide con el uniformizador que, a estos
tres puntos, asigna los valores $0$, $1728$, $\infty$, respectivamente
(y \'{e}ste es
un uniformizador para $X(1)$ en tanto curva definida sobre $\bb{Q}$). De manera
an\'{a}loga, podemos elegir un uniformizador para $X$ definido sobre $\bb{Q}$, y,
as\'{\i}, la relaci\'{o}n entre $u$ y $j$ es $j=\lambda P(u)/Q(u)$, donde
$P,Q\in\bb{Q}[T]$ y $\lambda\in\bb{Q}$. Ahora bien, sobre $\bb{C}$,
$j$, pensada como funci\'{o}n en $X$, tiene que cumplir con lo siguiente:
todo cero es un punto en la preimagen de $\rho$, el \'{u}nico cero de
$j\in\bb{C}(X(1))$, y todo polo es un punto arriba de $\infty$, el \'{u}nico polo
de $j$ en $X(1)$. Es decir, la relaci\'{o}n entre los uniformizadores es

\begin{align*}
 j & \,=\,\lambda
 \frac{\prod_{z|\rho}\,(u\,-\,u(z))^{e_{z}}}%
 {\prod_{z|\infty}\,(u\,-\,u(z))^{e_{z}}}\text{ ,}
\end{align*}
donde, si $\pi:\,X\rightarrow X(1)$ es la proyecci\'{o}n dada por olvidar la
estructura de nivel, $z| a$ indica que $z$ pertenece a $\pi^{-1}(a)$ y $e_{z}$
denota el \'{\i}ndice de ramificaci\'{o}n de $\pi$ en $z$. \cite{chenLevelFive}

%
%Siguiendo \cite{booher}, daremos una soluci\'{o}n
%relacionada con el argumento en la demostraci\'{o}n original de Heegner.
%
%
%Para poder sacar provecho de las curvas $X_{H}$ ser\'{a} \'{u}til obtener una
%parametrizaci\'{o}n de las mismas. Para conseguirlo es necesario considerar los
%morfismos $X_{H}\rightarrow X(1)$, y la informaci\'{o}n relativa a la
%ramificaci\'{o}n del cubrimiento, en tanto superficies de Riemann. Para los
%normalizadores de subgrupos de Cartan \textit{non-split} de nivel $N$, sabemos que
%el grado del morfismo es $N\phi(N)/2^{\omega}$, donde $\omega$ es la cantidad de
%primos distintos que dividen a $N$. Excepto sobre los puntos el\'{\i}pticos y las
%c\'{u}spides de $X(1)$, la fibra contiene dicha cantidad de puntos. En general, la
%suma de los \'{\i}ndices de ramificaci\'{o}n en cada una de las preim\'{a}genes de
%un punto es igual al grado del morfismo.
 \end{subsection}

 \begin{subsection}{Ramificaci\'{o}n}\label{subsec:ram}
  
Para poder sacar provecho de las curvas $X_{H}$ ser\'{a} \'{u}til obtener una
parametrizaci\'{o}n de las mismas. Para conseguirlo es necesario considerar los
morfismos $X_{H}\rightarrow X(1)$, y la informaci\'{o}n relativa a la
ramificaci\'{o}n del cubrimiento, en tanto superficies de Riemann. Para los
normalizadores de subgrupos de Cartan \textit{non-split} de nivel $N$, sabemos que
el grado del morfismo es $N\phi(N)/2^{\omega}$, donde $\omega$ es la cantidad de
primos distintos que dividen a $N$. Excepto sobre los puntos el\'{\i}pticos y las
c\'{u}spides de $X(1)$, la fibra contiene dicha cantidad de puntos. En general, la
suma de los \'{\i}ndices de ramificaci\'{o}n en cada una de las preim\'{a}genes de
un punto es igual al grado del morfismo.

Sea $\Gamma$ un subgrupo de $\SL{2}(\bb{Z})$ que contiene la matriz $-1$.
Sea $\pi$ la proyecci\'{o}n
$\Gamma\backs\frak{h}^{*}\rightarrow \SL{2}(\bb{Z})\backs\frak{h}^{*}$. Si
$z\in\frak{h}^{*}$, denotamos con $\Gamma_{z}$ su estabilizador en $\Gamma$.
Si $z$ es un punto el\'{\i}ptico para $\Gamma$, es decir, si $\Gamma_{z}$
contiene propiamente a $\{\pm 1\}$, entonces $\Gamma_{z}$ es un grupo
c\'{\i}clico finito (proposici\'{o}n 2.2.2 de \cite{diamondShurman}). Si
$a$ pertenece a $\SL{2}(\bb{Z})\backs\frak{h}^{*}$, si $z\in\frak{h}^{*}$ es un
punto arriba $a$ y $b\in\Gamma\backs\frak{h}^{*}$ es tal que $\pi(b)=a$, entonces
el \'{\i}ndice de ramificaci\'{o}n $e_{b}$ de $\pi$ en $b$ es igual al
\'{\i}ndice $|\SL{2}(\bb{Z})_{z}:\Gamma_{w}|$, donde $w\in\frak{h}^{*}$ es un
punto arriba de $b$. Si, adem\'{a}s, $w=\sigma z$ para alg\'{u}n elemento
$\sigma$ en $\SL{2}(\bb{Z})$, entonces $e_{b}$ tambi\'{e}n es igual a
$|\SL{2}(\bb{Z})_{z}:\sigma^{-1}\Gamma\sigma\cap\Gamma_{z}|$.

El estabilizador de un punto el\'{\i}ptico para $\SL{2}(\bb{Z})$ es de orden
$4$ o $6$ (el orden del punto es, respectivamente, $2$ o $3$). \textit{Modulo}
$\SL{2}(\bb{Z})$, el \'{u}nico punto el\'{\i}ptico de orden $2$ es $i$, y el
\'{u}nico de orden $3$ es $\rho$. Si $\Gamma\subset\SL{2}(\bb{Z})$ y
$\{\gamma_{j}\}_{j}$ es un conjunto (finito) de representantes de las coclases
de $\Gamma$ en $\SL{2}(\bb{Z})$, entonces un punto el\'{\i}ptico para $\Gamma$
tiene que ser de la forma $\Gamma\gamma_{j}\cdot i$ o $\Gamma\gamma_{j}\cdot\rho$
para alg\'{u}n $j$. En definitiva, $e_{b}$ es igual a $2$ o a $1$, si $z$
es el\'{\i}ptico de orden $2$, o bien igual a $3$ o a $1$, si $z$ es el\'{\i}ptico
de orden $3$. Adem\'{a}s, si $b$ es un punto el\'{\i}ptico de
$\Gamma\backs\frak{h}^{*}$ (si $w$ es un punto el\'{\i}ptico para $\Gamma$),
$\Gamma_{w}\not =\{\pm 1\}$ implica $e_{b}=1$, es decir, si el orden del punto
el\'{\i}ptico es mayor a $1$, entonces el \'{\i}ndice de ramificaci\'{o}n tiene
que ser $1$.

Describiremos, ahora, un sistema de representantes de las coclases de $\Gamma$ en
$\SL{2}(\bb{Z})$ para grupos $\Gamma$ particulares. Sea $H=C_{ns}(N)$ o
$H=C_{ns}^{+}(N)$ un grupo de Cartan \textit{non-split} de nivel $N$, o el
normalizador de un grupo tal. Sea $H'=H\cap\SL{2}(\bb{Z}/N\bb{Z})$ y sea
$\Gamma_{H}$ el subgrupo de $\SL{2}(\bb{Z})$ de matrices pertenecientes a $H'$
al reducir coordenadas \textit{modulo} $N$. Esta reducci\'{o}n establece,
adem\'{a}s, una biyecci\'{o}n entre las coclases de $\Gamma_{H}$ en
$\SL{2}(\bb{Z})$ y las de $H'$ en $\SL{2}(\bb{Z}/N\bb{Z})$.

Sea $I$ el orden en un cuerpo cuadr\'{a}tico imaginario con $\bb{Z}$-base
$\{1,\tau\}$, donde $\tau$ es ra\'{\i}z de un polinomio de la forma
$X^{2}-uX+v$, $u,v\in\bb{Z}$. Sea $A=I/NI$. Definimos una involuci\'{o}n en $A$
por $1\mapsto 1$ y $\tau\mapsto u-\tau$, y denotamos por $\overline{y}$ el
conjugado de un elemento $y\in A$. Definimos la \emph{norma} de un elemento $y$
en $A$ como $\nu(y):=y\overline{y}$.

Supongamos que $N=p^{r}$. El subgrupo $\nu(A^{\times})$ puede no ser
todo $(\bb{Z}/p^{r}\bb{Z})^{\times}$, pero, para una unidad, $a$, en
$\bb{Z}/p^{r}\bb{Z}$, existe $y_{a}\in A^{\times}$ tal que $\nu(y_{a})=\pm a$
(c.f. la demostraci\'{o}n de que
$\det:\,C_{ns}^{+}(N)\rightarrow(\bb{Z}/N\bb{Z})^{\times}$ es sobreyectivo).
Para cada $\pm a\in(\bb{Z}/p^{r}\bb{Z})^{\times}/\{\pm 1\}$, elegimos $y_{\pm a}$
de norma $a$ o $-a$, y definimos

\begin{align*}
 \cal{Y} & \,:=\,\left\lbrace y_{\pm a}
 \,:\,\pm a\in(\bb{Z}/p^{r}\bb{Z})^{\times}/\{\pm 1\}\right\rbrace
 \text{ .}
\end{align*}

\begin{lemaRepsCoclasesDelNormalizador}\label{thm:lemaRepsCoclasesDelNormalizador}
 Sean $x\in\bb{Z}/p^{r}\bb{Z}$ e $y\in \cal{Y}$. Sea $\gamma_{x,y}$ la matriz que
 representa la transformaci\'{o}n lineal determinada por $1\mapsto y^{-1}$ y
 $\tau\mapsto\overline{y}(\tau+x)$. Entonces las matrices $\gamma_{x,y}$
 constituyen un sistema de representantes de las coclases de
 $C_{ns}^{+}(p^{r})'=C_{ns}^{+}(p^{r})\cap\SL{2}(\bb{Z}/p^{r}\bb{Z})$.
\end{lemaRepsCoclasesDelNormalizador}

\begin{proof}[Demostraci\'{o}n]
 Ya hemos visto que el \'{\i}ndice de $C_{ns}^{+}(p^{r})'$ en
 $\SL{2}(\bb{Z}/p^{r}\bb{Z})$ es igual al de $C_{ns}^{+}(p^{r})$ en
 $\GL{2}(\bb{Z}/p^{r}\bb{Z})$, y que \'{e}ste es igual a $p^{r}\phi(p^{r})/2$.
 Por otra parte, si llamamos $M_{y}$ al morfismo dado por multiplicaci\'{o}n por
 un elemento $y\in A^{\times}$, los elementos de $C_{ns}^{+}(p^{r})$ son de la
 forma $M_{y}$ para $y$ de norma $1$, o $\sigma_{p}\circ M_{y}$ --$\sigma_{p}$ es,
 esencialmente, conjugaci\'{o}n compleja ($N$, en este caso, es divisible por un
 \'{u}nico primo)-- para $y$ de norma $-1$. Ahora, si $\gamma_{x,y}$ y
 $\gamma_{x',y'}$ son dos elementos de aquellos considerados en el enunciado, y
 si los mismos pertenecen a la misma coclase de $C_{ns}^{+}(p^{r})'$ en
 $\SL{2}(\bb{Z}/p^{r}\bb{Z})$, entonces existe $z$ tal que

 \begin{align*}
  \nu(z)=1 & \text{ y }\gamma_{x,y}=M_{z}\gamma_{x',y'}\text{ , o}\\
  \nu(z)=-1 & \text{ y }\gamma_{x,y}=\sigma_{p}M_{z}\gamma_{x',y'}\text{ .}
 \end{align*}
Evaluando en $1\in A$, se deduce que $\nu(y)=\pm\nu(y')$ y que $y=y'$ por la
definici\'{o}n del conjunto $\cal{Y}$. Evaluando en $\tau$, deducimos que $x$ y
$x'$ han de ser iguales en $\bb{Z}/p^{r}\bb{Z}$. Los pares $(x,y)$ dan lugar a
representantes de coclases distintas, pero, por cardinalidad, las coclases
representadas han de ser todas.
\end{proof}
%
%Si para cada $\pm a\in (\bb{Z}/p^{r})^{\times}/\{\pm 1\}$ tomamos
%$y_{\pm a}\in A^{\times}$ de norma igual a $\pm a$, y consideramos el conjunto
%$\cal{Y}=\{\sigma_{p}^{\epsilon}y_{\pm a}\,:\,
%\pm a\in(\bb{Z}/p^{r})^{\times}/\{\pm 1\},\,\epsilon=0,1\}$, el mismo argumento
%muestra que dos elementos de este conjunto no pueden pertenecer a la misma
%coclase de $C_{ns}(p^{r})'$ en $\SL{2}(\bb{Z}/p^{r})$, a menos que sean iguales.
%Dada la cardinalidad del conjunto, esta asociaci\'{o}n con transformaciones lineales
%da como resultado un sistema de representantes de dichas coclases, pues sabemos
%que las mismas son $p^{r}\phi(p^{r})$ en cantidad.

El g\'{e}nero $g$ de una curva como $X_{ns}(N)$ o $X_{ns}^{+}(N)$, al igual que el
g\'{e}nero de cualquier curva asociada a un subgrupo de congruencia, est\'{a}
dado por

\begin{align*}
 g & \,=\, 1\,+\,\frac{\mu}{12}\,-\,\frac{\mu_{2}}{4}
 \,-\,\frac{\mu_{3}}{3}\,-\,\frac{\mu_{\infty}}{2}\text{ .}
\end{align*}
El entero $\mu$ es el \'{\i}ndice del grupo en $\SL{2}(\bb{Z})$, los n\'{u}meros
$\mu_{k}$ denotan la cantidad de puntos el\'{\i}pticos de orden $k$ en la curva y
$\mu_{\infty}$ denota la cantidad de c\'{u}spides. Las cantidades $\mu_{2}$,
$\mu_{3}$ y $\mu_{\infty}$ son multiplicativas en $N$ para las curvas
$X_{ns}(N)$ y $X_{ns}^{+}(N)$, por lo que es suficiente considerar niveles
$N=p^{r}$.

Lo demostrado hasta ahora es suficiente para calcular la cantidad de c\'{u}spides
en $X_{ns}^{+}(p^{r})$: el estabilizador de $\infty$ en $\SL{2}(\bb{Z})$ est\'{a}
generado por la matriz
\begin{math}\left[
 \begin{smallmatrix}
  1&1\\0&1
 \end{smallmatrix}
\right]
\end{math}.
Si $a\in\bb{Z}$ es tal que

\begin{align*}
 &
 \begin{bmatrix}
  1&a\\0&1
 \end{bmatrix}
 \,\in\,\gamma C_{ns}^{+}(p^{r})\gamma^{-1}
\end{align*}
donde $\gamma$ es alguno de los representantes del lema anterior, existe
$\zeta\in C_{ns}^{+}(p^{r})$ tal que, en tanto endomorfismos de $A$,
\begin{math}\gamma^{-1}\left[
 \begin{smallmatrix}
  1&1\\0&1
 \end{smallmatrix}
\right]=\zeta\gamma^{-1}
\end{math}. Evaluando en la $(\bb{Z}/p^{r}\bb{Z})$-base de $A$, se deduce que
$a$ tiene que ser divisible por $p^{r}$. En particular, el \'{\i}ndice de
ramificaci\'{o}n de cualquier c\'{u}spide de $X_{ns}^{+}(p^{r})$ es igual a
$p^{r}$. Como el grado de $X_{ns}^{+}(p^{r})\rightarrow X(1)$ es
$p^{r}\phi(p^{r})/2$, la curva asociada al normalizador de un subgrupo de Cartan
\textit{non-split} de $\GL{2}(\bb{Z}/p^{r}\bb{Z})$ tiene, precisamente,
$\phi(p^{r})/2$ c\'{u}spides.

En \cite{baranNormalizers} se encuentran resultados m\'{a}s completos. Por ejemplo,
para terminar la descripci\'{o}n de $X_{ns}^{+}(p^{r})$ ($p\not =2$),
la cantidad de puntos el\'{\i}pticos de orden $3$ es

\begin{align*}
 \mu_{3} & \,=\,\begin{cases}
                 1  & \text{ si } p\equiv 2\,(\rm{mod}\,3)\text{,}\\
                 0  & \text{ si no.}
                \end{cases}
\end{align*}
La cantidad de puntos el\'{\i}pticos de orden $2$ es

\begin{align*}
 \mu_{2} & \,=\,\begin{cases}
                 \frac{1}{2}p^{r}\left(1-\frac{1}{p}\right) &  \text{ si }
                 p\equiv 1\,(\rm{mod}\,4)\text{ ,}\\
                 1+\frac{1}{2}p^{r}\left(1+\frac{1}{p}\right) & \text{ si }
                 p\equiv 3\,(\rm{mod}\,4)\text{ ,}\\
                 2^{r-1} & \text{ si } p=2\text{ .}
                \end{cases}
\end{align*}

%\begin{lemaCongruenciaDelNormalizador}
% La cantidad $f(p^{r})$ de pares $x\in\bb{Z}/p^{r}\bb{Z}$ e $y\in \cal{Y}$ tales
% que la ecuaci\'{o}n
%
% \begin{align*}
%  \overline{y}(\tau+x) & \,=\,\overline{y}^{-1}k
% \end{align*}
%admite soluci\'{o}n con $k$ en $A^{\times}$ de norma $\nu(k)=-1$ es igual a
%$p^{r-1}(p-1)/2$, a $p^{r-1}(p+1)/2$ o a $2^{r-1}$, si $p$ es, respectivamente,
%congruente a $1$ o a $3$ \textit{modulo} $4$, o igual a $2$.
%\end{lemaCongruenciaDelNormalizador}
%
%\begin{proof}[Demostraci\'{o}]
% Si $x$, $y$ y $k$ son como en el enunciado y est\'{a}n relacionados por la
% ecuaci\'{o}n $\overline{y}(\tau+x)=\overline{y}^{-1}k$, tomando norma,
% $\nu(\tau+x)\equiv -\nu(\overline{y})^{-2}\,(\rm{mod}\,p^{r})$. Como
% $\nu$ establece una biyecci\'{o}n entre los conjuntos $\cal{Y}$ y
% $(\bb{Z}/p^{r}\bb{Z})^{\times}/\{\pm 1\}$, buscamos conocer la cantidad de
% elementos $x$ pertenecientes a $\bb{Z}/p^{r}\bb{Z}$ y $a^{-1}$ en
% $(\bb{Z}/p^{r}\bb{Z})^{\times}/\{\pm 1\}$ tales que
% $\nu(\tau+x)=-a^{2}$. Si $h=(\tau+x)/a$, entonces $h$ es un elemento bien
% definido en $A^{\times}/\{\pm 1\}$ y, adem\'{a}s, $\nu(h)\equiv -1\,(p^{r})$.
%\end{proof}


 \end{subsection}
\end{section}
 
\begin{section}{Soluciones al problema del n\'{u}mero de %
		clases igual a $1$}\label{sec:soluciones}
 \begin{subsection}{Nivel $24$}\label{subsec:nivelVeinticuatro}
  %soluci\'{o}n N=24

Sea $N=24$. La referencia para la soluci\'{o}n al problema del n\'{u}mero de
clases $1$ usando la curva de nivel $24$ es la \cite{booher} (secci\'{o}n 9).
Para poder encontrar los puntos enteros de $X_{ns}^{+}(24)$,
nos concentraremos en las curvas $X_{ns}^{+}(3)$ y $X_{ns}^{+}(8)$. Existen
morfismos naturales definidos sobre $\bb{Q}$:

\begin{center}
 \begin{tikzcd}[row sep=small, column sep=small]
  \null & X_{ns}^{+}(3)\arrow{dr} & \\
  X_{ns}^{+}(24)\arrow{ur}\arrow{dr} & & X(1)\\
  \null & X_{ns}^{+}(8)\arrow{ur} & 
 \end{tikzcd}
\end{center}
En particular, todo punto $\bb{Q}$-racional en $X_{ns}^{+}(24)$ se proyecta a un
par de puntos $\bb{Q}$-racionales en $X_{ns}^{+}(3)$ y en $X_{ns}^{+}(8)$. Estas
dos \'{u}ltimas curvas, %y tambi\'{e}n $X_{ns}^{+}(4)$,
son de g\'{e}nero $0$ y definidas sobre $\bb{Q}$.

Teniendo en cuenta lo desarrollado en secciones anteriores, la curva
$X_{ns}^{+}(3)$ es ramificada en $\rho$ y en $\infty$. Los \'{\i}ndices de
ramificaci\'{o}n son, en ambos casos, iguales a $3$. Arriba de $i$, hay tres
puntos el\'{\i}pticos no ramificados. Ya hemos mencionado que, en $X(1)$, el
punto $\rho$ y la c\'{u}spide $\infty$ son racionales (al igual que $i$).
Como s\'{o}lo existe un punto, $P\in X_{ns}^{+}(3)$, que se proyecta sobre $\infty$
y s\'{o}lo uno, $Q$, arriba de $\rho$, vemos que tienen que ser invariantes por
la acci\'{o}n de $\absGal{\bb{Q}}$. Son, entonces, puntos $\bb{Q}$-racionales.

Sea $s:\,X_{ns}^{+}(3)\rightarrow\bb{P}^{1}$ un uniformizador definido sobre
$\bb{Q}$. Componiendo $s$ con un automorfismo de $\bb{P}^{1}$
(definido sobre $\bb{Q}$), podemos suponer que $s(P)=\infty$ y que $s(Q)=0$.
La relaci\'{o}n entre $j$ y $s$ tiene, as\'{\i}, el siguiente aspecto:

\begin{align*}
 j & \,=\,\lambda s^{3}\text{ ,}
\end{align*}
donde la constante $\lambda$ es un n\'{u}mero racional. Pero, adem\'{a}s, $\lambda$
tiene que ser un cubo, y, modificando nuevamente $s$, obtenemos $j=s^{3}$.

Por un procedimiento similar, se obtiene una parametrizaci\'{o}n para
$X_{ns}^{+}(8)$.

\begin{teoUnifsOchoYTres}\label{thm:teoUnifsOchoYTres}
 Existen uniformizadores

 \begin{align*}
  s & \,:\,X_{ns}^{+}(3)\rightarrow\bb{P}^{1}\text{ ,}\\
  v & \,:\,X_{ns}^{+}(8)\rightarrow\bb{P}^{1}
 \end{align*}
definidos sobre $\bb{Q}$ tales que

\begin{align*}
 j & \,=\,s^{3}\text{ ,}\\
 j & \,=\,\frac{-2^{17}(v+1)^{3}(8(v+1){3}+(v^{2}-2)^{2})^{3}}{(v^{2}-2)^{8}}
 \text{ .}
\end{align*}
\end{teoUnifsOchoYTres}

Un punto \'{\i}ntegro en $X_{ns}^{+}(24)$, un punto racional con
$j$-invariante entero, se proyecta a un punto racional en la curva de nivel $3$
y a un punto racional en la curva de nivel $8$. Si $j$ ha de ser entero,
tanto $s$, como $v$, deber\'{a}n ser racionales, pero $s$ deber\'{a}
ser entero tambi\'{e}n. De las parametrizaciones de
$X_{ns}^{+}(3)$ y de $X_{ns}^{+}(8)$ se deduce que $s$ y $v$ tienen que cumplir
con

\begin{align}\label{eq:relUnifsTresYOcho}
 s^{3} & \,=\,
 \frac{-2^{17}(v+1)^{3}(8(v+1){3}+(v^{2}-2)^{2})^{3}}{(v^{2}-2)^{8}}
 \text{ .}
\end{align}
Suponiendo que $(v,s)$ es una soluci\'{o}n, y que $v$ es de la forma $v=x/y$,
con $x$ e $y$ enteros coprimos, homogeneizamos y obtenemos la relaci\'{o}n

\begin{align*}
 t^{3} & \,=\,
 \frac{2^{17}y(x+y)^{3}(8(x+y){3}+(x^{2}-2y^{2})^{2})^{3}}{(x^{2}-2y^{2})^{8}}
 \text{ ,}
\end{align*}
donde $t=-s$. El \'{u}nico primo que puede dividir a $x^{2}-2y^{2}$ es $p=2$, y
como $x$ e $y$ son coprimos, $4$ no divide.
Todo se reduce, entonces a hallar las soluciones enteras $(x,y)$
de $x^{2}-2y^{2}=\pm 1,\pm 2$. En cualquier caso, como $t$ tiene que ser un entero,
el lado derecho tiene que ser un cubo en $\bb{Z}$. Si $x^{2}-2y^{2}=\pm 1$,
el \'{u}nico factor que no es, \textit{a priori}, un cubo es $2^{17}y$. Pero,
entonces $y=2z^{3}$, para alg\'{u}n entero $z$. Reemplazando y definiendo
$w:=2z^{2}$, el problema pasa a ser el de hallar las soluciones enteras de
$x^{2}=w^{3}\pm 1$. Si, en cambio, $x^{2}-2y^{2}=\pm 2$, el entero $y$ tiene que
un cubo $z^{3}$. As\'{\i}, $x$ tiene que ser par, y, reemplazando $x$ por $2x_{1}$,
llegamos a la ecuaci\'{o}n $2x_{1}^{2}=z^{6}\pm 1$.

En la secci\'{o}n 12 de \cite{cox}, la demostraci\'{o}n de que no existe un d\'{e}cimo cuerpo
cuadr\'{a}tico imaginario con n\'{u}mero de clases $1$, basada en la original de
Heegner, reduce el problema a hallar las soluciones a las cuatro ecuaciones

\begin{align*}
x^{2} & \,=\,w^{3}+1\text{ ,}\\
x^{2} &\,=\,w^{3}-1\text{ ,}\\
z^{6}\,+\,1 &\,=\,2x_{1}^{2}\text{ ,}\\
(-w)^{3}\,+\,1 &\,=\,-2x_{1}^{2}\text{ , haciendo el cambio } w=z^{2}\text{ .}
\end{align*}
Las soluciones a estas ecuaciones son
$(x,w)$ igual a $(0,-1)$, $(\pm 1,0)$ o $(\pm 3,2)$ para la primera,
$(x,w)=(0,1)$ para la segunda, $(x_{1},z)=(\pm1,\pm1)$ para la tercera y
$(x_{1},w)=(0,1)$ para la cuarta (ver la secci\'{o}n 6 de \cite{booher} o
la secci\'{o}n 12 de \cite{cox} para una idea de c\'{o}mo demostrarlo).
Pero, desandando el proceso que nos condujo a
estas ecuaciones, no todas las soluciones a las mismas dan lugar a posibles
valores de $v$ y de $t$. Por ejemplo, una soluci\'{o}n a $x^{2}=w^{3}\pm 1$ con
$x$ igual a $0$ y $w=\pm 1$ no da lugar a $v=x/y$ con $y$ de la forma $2z^{3}$,
pues $w$ no es de la forma $2z^{2}$.

Los posibles valores para el uniformizador $s$ son $0$, $-32$, $-96$, $-960$,
$-5280$ y $-640320$. Ya sabemos que el punto de $X_{ns}^{+}(3)$ con $s=0$ es
$\rho$, y que este punto se obtiene a partir del cuerpo cuadr\'{a}tico imaginario
$K=\bb{Q}(\sqrt{-3})$: recordemos que el punto asociado es $[E,\varphi]$,
donde $E$ es una curva el\'{\i}ptica definida sobre $\bb{Q}$ con CM por el anillo
de enteros de $K$ y $\varphi:\,E[n]\xrightarrow{\sim}(\bb{Z}/N\bb{Z})^{2}$ es una
estructura de nivel. La funci\'{o}n $j$ en $[E,\varphi]$ toma el valor $j(E)$,
el $j$-invariante de la curva el\'{\i}ptica, que coincide con $j(\cal{O}_{K})$,
viendo a $\cal{O}_{K}$ como ret\'{\i}culo en $\bb{C}$. \'{E}ste es el procedimiento
general para obtener un punto asociado a un cuerpo cuadr\'{a}tico imaginario, $K$,
cuyo n\'{u}mero de clases es igual a $1$ y tal que todo divisor primo del nivel,
$N$, es inerte en $K$. De las propiedades equivalentes
en la proposici\'{o}n \ref{thm:propoEquivsNumClasUno},
%REVISAR SI EN ALG\'{U}N OTRO LUGAR DEL TRABAJO
%SE HACE REFERENCIA A OTRA PARTE DEL MISMO
%
deducimos que, si $K$ es el cuerpo cuar\'{a}tico imaginario de discriminante $d$,
y $h(d)=1$, todo primo menor que $(1+|d|)/4$ es inerte.
Visto del otro lado, esto dice que, fijado $N$, toda curva el\'{\i}ptica con
multiplicaci\'{o}n compleja por un cuerpo cuadr\'{a}tico imaginario de
discriminante $d\geq 4p$ (donde $p$ es el primo de valor absoluto mayor entre
aquellos que dividen a $N$) y n\'{u}mero de clases $1$ da lugar a un punto
\'{\i}ntegro en $Y_{ns}^{+}(N)$.

Si $d$ es un discriminante fundamental, y $h(d)=1$, el $j$-invariante del orden
cuadr\'{a}tico imaginario de discriminante $d$ est\'{a} dado por la siguiente
tabla:

\begin{tabular}[b]{r|l}
 $d$ & $j$\\% & $j^{1/3}$\\
 \hline
 $-3$ & $0$\\% & 0\\
 $-4$ & $2^{6}\cdot 3^{3}$\\% & $2^{2}\cdot 3$\\
 $-7$ & $-3^{3}\cdot 5^{3}$\\% & $ -3\cdot 5$\\
 $-8$ & $2^{6}\cdot 5^{3}$\\% & $2^{2}\cdot 5$\\
 $-11$ & $-2^{15}$\\% & $-2^{5}$\\
 $-19$ & $-2^{15}\cdot 3^{3}$\\% & $-2^{5}\cdot 3$\\
 $-43$ & $-2^{18}\cdot 3^{3}\cdot 5{^3}$\\% & $-2^{6}\cdot 3\cdot 5$\\
 $-67$ & $-2^{15}\cdot 3^{3}\cdot 5^{3}\cdot 11^{3}$\\% &
% $-2^{5}\cdot 3\cdot 5\cdot 11$\\
 $-163$ & $-2^{18}\cdot 3^{3}\cdot 5^{3}\cdot 23^{3}\cdot 29^{3}$\\% &
% $-2^{6}\cdot 3\cdot 5\cdot 23\cdot 29$
\end{tabular}

Volvamos a la curva de nivel $24$.
Si $P$ es un punto $\bb{Q}$-racional de $X_{ns}^{+}(24)$ tal que $j(P)$ es uno
de $0$, $(-32)^{3}$, $(-96)^{3}$, $(-960)^{3}$,
$(-5280)^{3}$ o $(-640320)^{3}$, entonces $s(P)$ es la \'{u}nica ra\'{\i}z
racional de $j(P)$ y $s(P)$ pertenece al conjunto de los posibles valores
enteros calculados para $s$. Esto que parece obvio lo aplicamos de la siguiente
manera: conocemos el valor de $j(\cal{O}_{K})$ para cada cuerpo cuadr\'{a}tico
imaginario de discriminante $d$ que aparece en la tabla. Algunos de estos cuerpos
dan puntos enteros en $X_{ns}^{+}(24)$ %(lo que podr\'{\i}a fallar es que den
%puntos, directamente)
y cada uno de estos puntos tiene asociado un valor de $j$. Por otro lado, si
$j(\cal{O}_{K})=j(\cal{O}_{K'})$, entonces $K=K'$ y los \'{o}rdenes son iguales.
Si $d\geq 12$ y $K$ es el cuerpo cuadr\'{a}tico de discriminante $d$, y si
$h(d)=1$, $K$ tiene asociado un punto \'{\i}ntegro $P$ en la curva.
Evaluando $j$ en $P$, obtenemos $j(P)=j(\cal{O}_{K})$ y tiene que ser el cubo de
uno de los posibles de $s$. Pero todo valor de $s$ entero viene de un punto
asociado a uno de los cuerpos de discriminante $-3$, $-11$, $-19$, $-43$, $-67$
y $-163$, con lo cual, $K$, cuyo discriminante $d$ es $d\leq -12$,
tiene que ser uno de estos seis. Como para $0>d>-12$, los \'{u}nicos con
$h(d)=1$ son los de la tabla, esto demuestra que no hay un d\'{e}cimo cuerpo
cuadr\'{a}tico imaginario con n\'{u}mero de clases $1$.

\begin{obsSerre}\label{thm:obsSerre}
De la relaci\'{o}n entre los uniformizadores $s$ y $v$,
la ecuaci\'{o}n \ref{eq:relUnifsTresYOcho},
notemos que $s$ ser\'{a} racional, si $4/(v^{2}-2)^{2}$ es un cubo.
Equivalentemente, ser\'{a} suficiente que $4(v^{2}-2)$ sea un cubo. Para hallar
los posibles valores de $s$, uno podr\'{\i}a intentar encontrar los puntos
racionales en la curva el\'{\i}ptica $v'^{2}=u^{3}+8$ tales que, si $v=v'/2$,
entonces $s$ sea entero.
\end{obsSerre}

%Pero $P$ es $\bb{Q}$-racional, con lo que $s(P)$ es
%uno de $0$, $-32$, $-96$, $-960$, $-5280$ o $-640320$.
%
 \end{subsection}

 \begin{subsection}{Nivel $7$}\label{subsec:nivelSiete}
  \input{./files/nivel7.tex}
 \end{subsection}

 \begin{subsection}{Nivel $5$ (o $15$)}\label{subsec:nivelCinco}
  

%\begin{subsection}{Nivel $5$}
Hemos visto que la curva $X_{ns}^{+}(3)$ admite un uniformizador $s$
que satisface $j=s^{3}$. En particular,
se deduce que toda curva el\'{\i}ptica $E$ definida sobre
$\overline{\bb{Q}}$ da lugar a un punto $\bb{Q}$-racional en
$X_{ns}^{+}(3)$, si, y s\'{o}lo si $j(E)$ es un cubo en $\bb{Q}$.
Si $d$ es igual a $-7$, $-8$, $-28$, $-43$, $-67$ o a $-163$, entonces, en
el orden cuad\'{a}tico imaginario $I$ de discriminante $d$, el primo $3$ es
no ramificado y $5$ es inerte. Se obtienen, as\'{\i}, al menos seis
puntos $\bb{Q}$-racionales en $X_{ns}^{+}(3)$ y en $X_{ns}^{+}(5)$ cuyos
$j$-invariantes son cubos en $\bb{Z}$. Si $d$ es menor a $-163$, tanto
$3$, como $5$, es inerte en el orden de discriminante $d$. Por lo tanto,
a partir de un orden cuadr\'{a}tico imaginario $I$ de discriminante $d<-163$
con n\'{u}mero de clases igual a $1$, se obtiene un punto $\bb{Q}$-racional
en $X_{ns}^{+}(5)$ con $j$ un cubo entero. Pero, por medio de una
parametrizaci\'{o}n de dicha curva, se obtiene la lista completa de los
posibles puntos con los que se debe corresponder. Como los seis \'{o}rdenes
mencionados y aquel de discriminante $-3$ dan cuenta de todos estos puntos,
$I$ debe ser uno de ellos. En \cite{chenLevelFive}, el autor obtiene una
parametrizaci\'{o}n de $X_{ns}^{+}(5)$, permiti\'{e}ndole dar una
soluci\'{o}n al problema del n\'{u}mero de clases igual a $1$, como
tambi\'{e}n interpretar la soluci\'{o}n por Siegel del problema
(en un trabajo titulado \textit{Zum Beweise des Starkschen Satzes})
en t\'{e}rminos de $X_{ns}^{+}(5)$. A continuaci\'{o}n
resumimos el proceso que conduce a dicha parametrizaci\'{o}n.

Como en los dos ejemplos anteriores, consideremos el cubrimiento de
$X_{ns}^{+}(5)$ sobre $X(1)$. Recordemos que los puntos el\'{\i}pticos de
orden dos o tres en $X_{ns}^{+}(5)$ son puntos que se proyectan, respectivamente,
sobre $i$ o $\rho$ en $X(1)$ y cuyo \'{\i}ndice de ramificaci\'{o}n es igual a $1$.
Por otra parte, las f\'{o}rmulas al final de \ref{subsec:ram} indican que, para
esta curva, hay un \'{u}nico punto el\'{\i}ptico de orden tres (y, por lo tanto,
otros tres puntos arriba de $\rho$ cuyo \'{\i}ndice de ramificaci\'{o}n es $3$),
y dos de orden dos (y cuatro puntos arriba de $i$ cuyo \'{\i}ndice es $2$). Con
respecto a las c\'{u}spides, las mismas f\'{o}rmulas nos muestran que
$X_{ns}^{+}(5)$ cuenta con dos c\'{u}spides, y la ramificaci\'{o}n es, en ambas,
de \'{\i}ndice $5$.
%
Como el grupo de Galois act\'{u}a por permutaciones sobre el conjunto de
c\'{u}spides, y dado que las c\'{u}sides de $X_{ns}^{+}(5)$ est\'{a}n
definidas sobre $\bb{Q}(\zeta_{5})$, donde $\zeta_{5}:=e^{2\pi i/5}$,
si $\sigma\in\Gal(\bb{Q}(\zeta_{5})/\bb{Q})$, entonces $\sigma^{2}$
act\'{u}a trivialmente sobre dicho conjunto. En particular, las c\'{u}spides
est\'{a}n definidas sobre la subextensi\'{o}n cuadr\'{a}tica
$\bb{Q}(\sqrt{5})/\bb{Q}$.

Sea $\eta:\,X_{ns}^{+}(5)\rightarrow\bb{P}^{1}$ un uniformizador definido
sobre $\bb{Q}$. Componiendo $\eta$ con un $\bb{Q}$-automorfismo de
$\bb{P}^{1}$, podemos suponer que las c\'{u}spides de $X_{ns}^{+}(5)$ son
las ra\'{\i}ces de $X^{2}-5$. Por otra parte, como la curva tiene un
\'{u}nico punto el\'{\i}ptico de orden tres, el mismo tiene que ser un punto
$\bb{Q}$-racional de $X_{ns}^{+}(5)$, tiene que quedar fijo por la
acci\'{o}n del grupo de Galois. Podemos asumir tambi\'{e}n que
$\eta$ en este punto toma el valor $0$.
Entonces, la parametrizaci\'{o}n de la curva va a estar dada por una
relaci\'{o}n de la forma:

\begin{align*}
j & \,=\,\lambda
\frac{\eta(\eta-A)^{3}(\eta^{2}-B\eta+C)^{3}}{(\eta^{2}-5)^{5}}\text{ .}
\end{align*}
Las constantes $A$, $B$ y $C$ est\'{a}n determinadas por el valor de $\eta$
en los puntos de $X_{ns}^{+}(5)$ arriba de $\rho$ cuyo \'{\i}ndice de
ramificaci\'{o}n es $3$. Usando un cubrimiento intermedio de manera
similar a lo explicado en \ref{subsec:nivelSiete} es posible calcular los
valores de estas constantes. A trav\'{e}s de la transformaci\'{o}n
$z\mapsto 2z/(z+5)$, se llega a la relaci\'{o}n siguiente \cite{chenLevelFive}

\begin{align*}
j & \,=\,5^{3}
\frac{\eta(2\eta+1)^{3}(2\eta^{2}+7\eta+8)^{3}}{(\eta^{2}+\eta-1)^{5}}
\text{ .}
\end{align*}
%
Una vez hallada esta parametrizaci\'{o}n de $X_{ns}^{+}(5)$, si llamamos
$t$ al uniformizador de $X_{ns}^{+}(3)$ que es la ra\'{\i}z c\'{u}bica
de $j$, sabemos que, dado un orden cuadr\'{a}tico imaginario $I$
cuyo n\'{u}mero de clases es $1$, y elegida una curva el\'{\i}ptica $E$
(definida sobre $\bb{Q}$) con multiplicaci\'{o}n compleja por $I$,
si el primo $3$ no ramifica en $I$ y $5$ es inerte, entonces, por medio
de $t$ y $\eta$, obtenemos enteros $m$, $x$ e $y$ tales que
$j(E)=m^{3}$, $\eta=x/y$ y $(x,y)=1$. Adem\'{a}s, por la relaci\'{o}n
con $j$, sabemos que la terna $(x,y,m)$ tiene que ser una soluci\'{o}n
en $\bb{Z}$ a la ecuaci\'{o}n

\begin{align*}
m^{3} & \,=\,u(x,y)\,:=\,5^{3}
\frac{x(2x+y)^{3}(2x^{2}+7xy+8y^{2})^{3}}{(x^{2}+xy-y^{2})^{5}}
\text{ .}
\end{align*}
Las soluciones con $m=0$ son $(0,1,0)$, $(0,-1,0)$, $(-1,2,0)$ y
$(1,-2,0)$. 

En general, si $(x,y,m)$ es soluci\'{o}n a la ecuaci\'{o}n, tambi\'{e}n
lo es $(-x,-y,m)$. Sea $m\not =0$ y $(x,y,m)$ una
soluci\'{o}n. Si $l$ es un primo racional que divide a $x^{2}+xy-y^{2}$,
entonces $(x,y)=1$ implica que $l$ no puede dividir ni a $x$, ni a $y$.
Dado que $u(x,y)$ es un entero y $l$ divide al denominador en la
expresi\'{o}n para $u(x,y)$, el primo $l$ tiene que ser $5$. Esto se
deduce de que el sistema de ecuaciones \textit{modulo} $l$ (con $l$
primo)

\begin{align*}
z^{2}\,+\,z\,-\,1 & \,\equiv\, 0\,(\rm{mod}\,l)\\
2z\,+\,1 & \,\equiv\, 0\,(\rm{mod}\,l)
\end{align*}
tiene soluciones en enteras s\'{o}lo si $l=5$, y de que lo mismo es
cierto para

\begin{align*}
z^{2}\,+\,z\,-\,1 & \,\equiv\, 0\,(\rm{mod}\,l)\\
2z^{2}\,+\,7z\,+\,8 & \,\equiv\, 0\,(\rm{mod}\,l)\text{ .}
\end{align*}
Pero la ecuaci\'{o}n $z^{2}+z-1\equiv 0\,(5^{2})$ no admite soluciones
en $\bb{Z}$. En definitiva, si $(x,y,m)$ es una soluci\'{o}n
para $m^{3}=u(x,y)$ con $x$, $y$ y $m$ en $\bb{Z}$ y $(x,y)=1$,
la expresi\'{o}n $x^{2}+xy-y^{2}$ es igual a $\pm 5$ o $\pm 1$. En el
primer caso, $u(x,y)$ no es un cubo en $\bb{Z}$. Por esta raz\'{o}n,
$x^{2}+xy-y^{2}=\pm 1$.

Si denotamos con $\epsilon$ al elemento $(-1+\sqrt{5})/2$ de
$F:=\bb{Q}(\sqrt{5})$, y $\cal{O}_{F}$ al anillo de enteros de este
cuerpo, la condici\'{o}n sobre $x^{2}+xy-y^{2}$ equivale a que
la norma de $x+y\epsilon\in\cal{O}_{F}$ sea igual a $1$ o a $-1$, a que
$x+y\epsilon$ sea una unidad en este anillo.
Pero las unidades de $\cal{O}_{F}=\bb{Z}[\epsilon]$ son de la forma

\begin{align*}
\pm\epsilon^{n} & \,=\,\pm(x_{n}\,+\,y_{n}\epsilon)\text{ ,}
\end{align*}
donde $x_{n}:=(-1)^{n+1}F_{n}$ e $y_{n}:=(-1)^{n}F_{n+1}$
($F_{n}$ es el $n$-\'{e}simo n\'{u}mero de Fibonacci). En particular,
Dada la soluci\'{o}n $(x,y,m)$, tenemos $x=\pm x_{n}$ para alg\'{u}n
$n$. Pero, de la expresi\'{o}n para $u(x,y)$, deducimos que $x$ es un
cubo en $\bb{Z}$, y que, entonces $x_{n}$ tambi\'{e}n lo es.

Los \'{u}nicos n\'{u}meros de Fibonacci que son cubos son $F_{1}=1$, $F_{2}=1$
y $F_{6}=8$ (ver \cite{chenLevelFive}). Si definimos

\begin{align*}
 & L_{1}\,:=\,1\text{ , }\,L_{2}\,:=\,3\text{ , }
 \,L_{n+1}\,=\,L_{n}\,+\,L_{n-1}\text{ ,}\\
 & a\,=\,\frac{1\,+\,\sqrt{5}}{2}\text{ , }
 \,b\,=\,\frac{1\,-\,\sqrt{5}}{2}\text{ ,}
\end{align*}
inductivamente,

\begin{align*}
 & F_{n}\,=\,\frac{a^{n}\,-\,b^{n}}{\sqrt{5}}\text{ , }
 \,L_{n}\,=\,a^{n}\,+\,b^{n}\text{ y}\\
 & L_{n}^{2}\,-\,5F_{n}^{2}\,=\,4(-1)^{n}\text{ .}
\end{align*}
Sobre la curva de nivel $24$ el problema de hallar los puntos enteros se
reduc\'{\i}a a hallar soluciones a ciertas ecuaciones diof\'{a}nticas. Sobre la
curva de nivel $5$, la ecuaci\'{o}n diof\'{a}ntica es
%el problema pasa a ser la resoluci\'{o}n de

\begin{align*}
 Y^{2} & \,=\,5Z^{6}\,\pm\,4\text{ .}
\end{align*}

En resumen, las posibles soluciones $(x,y,m)$ a $m^{3}=u(x,y)$ en $\bb{Z}$
con $(x,y)=1$, cumplen con $x=0$, $\pm 1$ o $\pm 8$. Requiriendo que $y$ en
$(x,y,m)$ sea positivo, las posibles soluciones son las ternas en la siguiente
tabla.

\begin{tabular}{r|l|l}
 $d$ & $j=t^{3}$ & $(x,y,m)$\\
\hline
 $-3$ & $0$ & $(0,1,0)$\\
 $-3$ & $0$ & $(-1,2,0)$\\
 $-7$ & $-3^{3}\cdot 5^{3}$ & $(-1,1,-15)$\\
 $-8$ & $2^{6}\cdot 5^{3}$%
	& $(1,0,20)$\\
 $-28$ & $3^{3}\cdot 5^{3}\cdot 7^{3}$%
	& $(1,1,255)$\\
 $-43$ & $-2^{15}\cdot 3^{3}$%
	& $(1,2,-96)$\\
 $-67$ & $-2^{15}\cdot 3^{3}\cdot 5^{3}\cdot 11^{3}$%
	& $(-8,5,-5280)$\\
 $-163$ & $-2^{18}\cdot 3^{3}\cdot 5^{3}\cdot 23^{3}\cdot 29^{3}$%
	& $(8,13,-640320)$
\end{tabular}



%\end{subsection}
 \end{subsection}

 \begin{subsection}{Nivel $9$}\label{subsec:nivelNueve}
  En la soluci\'{o}n al problema del n\'{u}mero de clases igual a $1$ que damos
a continuaci\'{o}n, y que se encuentra en \cite{baranLevelNine},
el m\'{e}todo es similar al que se puede encontrar en \cite{chenLevelFive}
y que est\'{a} detr\'{a}s de la descripci\'{o}n en la secci\'{o}n anterior.

Sea $\rm{r}:\,\SL{2}(\bb{Z}/9\bb{Z})\rightarrow\SL{2}(\bb{Z}/3\bb{Z})$ el  morfismo
dado por reducci\'{o}n de coordenadas \textit{modulo} $3$, y sea $N$ el
n\'{u}cleo de este morfismo. Si llamamos $H$ al subgrupo de
$\SL{2}(\bb{Z}/9\bb{Z})$ generado por las matrices
\begin{math}\left[\begin{smallmatrix}0&-1\\1&0\end{smallmatrix}\right]\end{math}
y
\begin{math}\left[\begin{smallmatrix}-1&-4\\-4&1\end{smallmatrix}\right]\end{math},
y $N'$ al generado por
\begin{math}\left[\begin{smallmatrix}1&-3\\3&1\end{smallmatrix}\right]\end{math},
entonces $H$ normaliza a $N'$
y $C_{ns}^{+}(9)'=C_{ns}^{+}(9)\cap\SL{2}(\bb{Z}/9\bb{Z})$ es el producto
semidirecto de $N'$ por $H$.
Si $N''$ es el subgrupo generado por
\begin{math}\left[\begin{smallmatrix}1&-3\\3&1\end{smallmatrix}\right]\end{math}
y por
\begin{math}\left[\begin{smallmatrix}-2&3\\3&4\end{smallmatrix}\right]\end{math},
entonces $H$ normaliza a $N''$ tambi\'{e}n, y

\begin{align*}
 & C_{ns}^{+}(9)'\,\simeq\,N'\,\rtimes\,H\text{ y }
 \,r^{-1}(C_{ns}^{+}(3))\,\simeq\,N\,\rtimes\,H\text{ .}
\end{align*}
Adem\'{a}s, valen las inclusiones
$C_{ns}^{+}(9)\subset N''\rtimes H\subset r^{-1}(C_{ns}^{+}(3))$, ambas de
\'{\i}ndice $3$.
Llamaremos $B$ al subgrupo de $r^{-1}(C_{ns}^{+}(3))$ isomorfo a $N''\rtimes H$, y
$\Gamma_{B}$ al subgrupo de $\SL{2}(\bb{Z})$ conformado por las matrices
congruentes a alg\'{u}n elemento de $B$ al reducir coordenadas \textit{modulo} $9$.

En cuanto a las inclusiones
$\Gamma_{ns}^{+}(9)\subset\Gamma_{B}\subset\Gamma_{ns}^{+}(3)$ los \'{\i}ndices
son, tambi\'{e}n, $3$. En particular, los cubrimientos $\Phi_{1}$ y
$\Phi_{2}$ en

\begin{align*} 
 & X_{ns}^{+}(9)\xrightarrow{\Phi_{1}}
 X_{B}\xrightarrow{\Phi_{2}} X_{ns}^{+}(3)\xrightarrow{\Phi_{3}}
 X(1)\text{ ,}
\end{align*}
son, ambos, de grado $3$. El grado del cubrimiento $\Phi_{3}$ tambi\'{e}n es $3$.
A continuaci\'{o}n hacemos uso de la informaci\'{o}n de ramificaci\'{o}n de
estos cubrimientos, refiriendo a \cite{baranLevelNine} con respecto a los detalles
de c\'{o}mo obtenerla.

Sabemos que las curvas $X_{ns}^{+}(3)$ y $X_{ns}^{+}(9)$ est\'{a}n definidas
sobre $\bb{Q}$ y que los morfismos $\Phi_{2}\circ\Phi_{1}$ y
$\Phi_{3}$ son $\bb{Q}$-morfismos. Si consideramos el subgrupo
$(\bb{Z}/9\bb{Z})^{\times}\cdot B$ de $\GL{2}(\bb{Z}/9\bb{Z})$, y argumentando
como con el grupo $S$ al tratar la curva de nivel $7$, vemos que $X_{B}$ est\'{a}
definida sobre $\bb{Q}(\sqrt{-3})$.

En cuanto a $X_{ns}^{+}(3)$ elgimos el uniformizador $t$ tal que $j=t^{3}$.
En esta curva hay un \'{u}nico punto $\rho'$ arriba de $\rho\in X(1)$ y
un \'{u}nico punto $\infty'$ arriba de $\infty$. Sobre el punto $i\in X(1)$, en
cambio, se proyectan tres, $i_{1}$, $i_{2}$ e $i_{3}$, eligiendo los
sub\'{\i}ndices de manera que $t(i_{1})=12$, $t(i_{2})=12\zeta_{3}^{-1}$
y $t(i_{3})=12\zeta_{3}$, donde $\zeta_{3}=e^{2\pi i/3}$.

En $X_{B}$ hay una \'{u}nica c\'{u}spide, que denotamos $\infty''$.
Hay dos puntos $i_{3,1}$ e $i_{3,2}$ en $X_{B}$ tales que
$\Phi_{2}(i_{3,k})=i_{3}$. El \'{\i}ndice de ramificaci\'{o}n de $\Phi_{2}$
en $i_{3,1}$ es $1$ y en $i_{3,2}$ es igual a $2$. \'{E}stos son los \'{u}nicos
puntos arriba de $i_{3}$; la acci\'{o}n
de $\Gal(\overline{\bb{Q}}/\bb{Q}(\sqrt{-3}))$ los permuta. Pero
uno es ramificado y el otro no, entonces los tiene que dejar fijos, es decir,
son $\bb{Q}(\sqrt{-3})$-racionales.

\begin{propoBaranNineXB}\label{thm:propoBaranNineXB}
 Existe un uniformizador $w:\,X_{B}\rightarrow\bb{P}^{1}$, definido
 sobre $\bb{Q}(\sqrt{-3})$ y tal que $w(\infty'')=\infty$, $w(i_{3,1})=2\sqrt{-3}$
 y $w(i_{3,2})=-\sqrt{-3}$. Adem\'{a}s, la relaci\'{o}n entre $w$ y $t$ es

 \begin{align*}
  t & \,=\,\zeta_{3}^{-1}(w^{3}\,+\,9w\,-\,6)\text{ .}
 \end{align*}
\end{propoBaranNineXB}

\begin{proof}[Demostraci\'{o}n]
 Supongamos que $\eta:\,X_{B}\rightarrow\bb{P}^{1}$ es el uniformizador determinado
 por $\eta(\infty'')=\infty$, $\eta(i_{3,1})=1$ y $\eta(i_{3,2})=0$. Sabemos que,
 entonces, la relaci\'{o}n con $t$ tiene que ser de la forma
 (ver la secci\'{o}n \ref{subsec:params})

 \begin{align*}
  t &\,=\,\lambda\prod_{k=1}^{3}\,(\eta\,-\,\eta(\rho_{k}))
  \,=\,\lambda(\eta^{3}\,+\,A\eta^{2}\,+\,B\eta\,+\,C)
 \end{align*}
ya que los puntos $\rho_{k}$, arriba de $\rho'$, son no ramificados.
Las constantes $\lambda\not = 0$, $A$, $B$ y $C$ pertenecen al
cuerpo $\bb{Q}(\sqrt{-3})$. Tambi\'{e}n podemos expresar $t$ en t\'{e}rminos de
los valores de $\eta$ en otros puntos: por ejemplo, teniendo en cuenta la
ramificaci\'{o}n de $\Phi_{2}$,

\begin{align*}
 t & \,=\,\lambda(\eta\,-\,\eta(i_{3,2}))^{2}(\eta\,-\,\eta(i_{3,1}))
 \,+\,t(i_{3})\\
 & \,=\,\lambda(\eta\,-\,\eta(i_{1,2}))^{2}(\eta\,-\,\eta(i_{1,1}))
 \,+\,t(i_{1})\text{ .}
\end{align*}
El valor de $t$ en $i_{3}$ es $12\zeta_{3}$ y, en $i_{1}$, $12$. Evaluando en
$i_{3,2}$, se deduce que el valor de $\lambda C$ es $12\zeta_{3}$, y, como el
\'{\i}ndice de ramificaci\'{o}n de $\Phi_{2}$ en $i_{3,2}$ es igual a $2$, que
$t-t(i_{3})$ tiene un cero doble en $i_{3,2}$. En particular, la constante $B$
tiene que ser igual a $0$. Por otra parte, evaluando en $i_{3,1}$, se ve que el
valor de $A$ es $-1$. De la misma manera, al evaluar en $i_{1,1}$ e $i_{1,2}$,
se obtiene un sistema de ecuaciones que relacionan las constantes $\lambda$ y $C$
con $\eta(i_{1,1})$ y $\eta(i_{1,2})$ de donde se puede deducir los valores de
estos cuatro elementos de $\bb{Q}(\sqrt{-3})$. La relaci\'{o}n entre los
uniformizadores es

\begin{align*}
 t & \,=\,-81(\zeta_{3}\,-\,1)\left(\eta^{3}\,-\,\eta^{2}\,+\,
 \frac{-4(\zeta_{3}-1)}{81}\right)\text{ .}
\end{align*}
El enunciado de la proposici\'{o}n se puede deducir de hacer un cambio de
variables: $w=-(\sqrt{-3})(-3\eta+1)$.
\end{proof}

\begin{propoBaranNineXNine}\label{thm:propoBaranNineXNine}
 Existe un uniformizador $y:\,X_{ns}^{+}(9)\rightarrow\bb{P}^{1}$, definido
 sobre $\bb{Q}$, tal que su relaci\'{o}n con el uniformizador $t$ est\'{a} dada por

 \begin{align}\label{eq:relUnifsTresYNueve}
  t & \,=\,\frac{-3(y^{3}+3y^{2}-6y+4)(y^{3}+3y^{2}+3y+4)%
  (5y^{3}-3y^{2}-3y+2)}{(y^{3}-3y+1)^{3}}\text{ .}
 \end{align}
\end{propoBaranNineXNine}

\begin{proof}[Demostraci\'{o}n]
 El uniformizador $y$ se define en t\'{e}rminos de $w$ y un conjugado. Veamos,
 en primer lugar, que $X_{B}$ no puede estar definida sobre $\bb{Q}$ de manera
 compatible con $X_{ns}^{+}(9)$. Es decir, sabemos que $X_{B}$ est\'{a} definida
 sobre $K:=\bb{Q}(\sqrt{-3})$, y que el cuerpo de funciones $K(X_{B})$ est\'{a}
 contenido en $K(X_{ns}^{+}(9))$. Si $\sigma$ un generador de $\Gal(K/\bb{Q})$,
 entonces, lo que queremos decir con que no es posible que $X_{B}$ est\'{e}
 definida de manera compatible sobre $X_{ns}^{+}(9)$ es que $\sigma$ no deja fijo
 al subcuerpo $K(X_{B})$ bajo la acci\'{o}n de $\Gal(K/\bb{Q})$ sobre
 $K(X_{ns}^{+}(9))$. La raz\'{o}n de esto es que, por ejemplo, los puntos
 $i_{1}$ e $i_{2}$ de $X_{ns}^{+}(3)$ son conjugados%?`en qu\'{e} sentido?
 , pero uno ramifica en $X_{B}$ y el otro no, con lo cual $\Phi_{2}$ no puede
 ser un $\bb{Q}$-morfismo y $X_{B}$ estar definida sobre $\bb{Q}$.

 La imagen del cuerpo $K(X_{B})=K(w)$ por el automorfismo $\sigma$ es
 $K(w')$, donde $w'=w^{\sigma}$. Dado el grado de $K(X_{ns}^{+}(9))/K(t)$,
 resulta que $K(X_{ns}^{+}(9))=K(w,w')$. Por otra parte, dado que $t^{\sigma}=t$,
 se deduce la igualdad

 \begin{align*}
  \zeta_{3}^{-1}(w^{3}+9w-6)&\,=\,\zeta_{3}(w'^{3}+9w'-6)\text{ .}
 \end{align*}
Esta igualdad permite escribir $w$ en t\'{e}rminos de
$u:=(w-\sqrt{-3})/(w'+\sqrt{-3})$. La relaci\'{o}n que se obtiene es

\begin{align*}
 w & \,=\,3u\sqrt{-3}\left(
 \frac{-u^{2}-\zeta_{3}^{-1}}{u^{3}-\zeta_{3}^{-1}}\right)\,+\,\sqrt{-3}
 \text{ .}
\end{align*}
En conjunto con la relaci\'{o}n entre $t$ y $w$, podemos expresar la relaci\'{o}n
entre los uniformizadores $u$ y $t$. Pero $u$ est\'{a} definido sobre
$\bb{Q}(\sqrt{-3})$. Haciendo el cambio de variables
$u=(y+\zeta_{3})/(\zeta_{3}y+1)$, $y$ cumple con que $y^{\sigma}=y$ y, adem\'{a}s,
con que la relaci\'{o}n con $t$ es la del enunciado.
\end{proof}

Una vez encontrada esta parametrizaci\'{o}n, con el fin de determinar los puntos
enteros en $X_{ns}^{+}(9)$, el paso siguiente es determinar las soluciones
$(y,t)$, con $t$ un n\'{u}mero entero e $y=m/n$ ($m$, $n$ enteros coprimos), de la
ecuaci\'{o}n \ref{eq:relUnifsTresYNueve}.
%, o, lo que es lo mismo,
%ternas $(m,n,t)$ de enteros, $(m,n)=1$, tales que
%
%\begin{align*}
% t & \,=\,\frac{-3(m^{3}+3m^{2}n-6mn^{2}+4n^{3})(m^{3}+3m^{2}n+3mn^{2}+4n^{3})%
%		  (5m^{3}-3m^{2}n-3mn^{2}+2n^{3})}{(m^{3}-3mn^{2}+n^{3})^{3}}
%		  \text{ .}
%\end{align*}

Se puede ver de manera elemental que, si $(m/n,t)$ es una soluci\'{o}n en $\bb{Z}$,
con $m$ y $n$ coprimos, entonces

\begin{align}\label{eq:denominadorTresYNueve}
 m^{3}\,-\,3mn^{2}\,+\,n^{3} & \,=\,k\text{ ,}
\end{align}
con $k\in\{\pm 1,\,\pm 3\}$.
El prolema queda reducido a hallar soluciones a la ecuaci\'{o}n
\ref{eq:denominadorTresYNueve}
con $k=1$ o $3$. Las soluciones en estos casos est\'{a}n completamente
determinadas y son nueve pares $(m,n)$. Los puntos que estos pares determinan en
$X_{ns}^{+}(9)$ son los puntos enteros de la curva, y todos, excepto uno, se
corresponden con \'{o}rdenes en cuerpos cuadr\'{a}ticos imaginarios (tabla 5.2
en \cite{baranLevelNine}). El caso excepcional es $j=3^{3}41^{3}61^{3}149^{3}$. \'{E}ste es
el $j$-invariante de una curva el\'{\i}ptica que, si bien no es CM, parece serlo
\textit{modulo} $9$. Concretamente, la acci\'{o}n de Galois sobre la
$9$-torsi\'{o}n de esta curva se realiza en el normalizador de un subgrupo de
Cartan \textit{non-split}.
 \end{subsection}

 \begin{subsection}{Nivel $11$}\label{subsec:nivelOnce}
  A diferencia de las curvas consideradas en las soluciones anteriores
($N=3$, $5$, $7$, $8$, $9$), la curva de nivel $11$, $X_{ns}^{+}(11)$
es $\bb{Q}$-isomorfa a la curva de g\'{e}nero $1$ dada por la primera
de las siguientes ecuaciones de Weierstra{\ss}:

\begin{align*}
y^{2}\,+\,y & \,=\,x^{3}\,-\,x^{2}\,-\,7x\,+\,10\text{ ,}\\
y^{2}\,+\,11y & \,=\,x^{3}\,+\,11x^{2}\,+\,33x\text{ .}
\end{align*}
La segunda de estas ecuaciones se obtiene tras reemplzar $x$ e $y$ por
$x+4$ e $y+5$, y determina una curva isomorfa, que denotaremos $C$.
Entonces, al igual que $X_{ns}^{+}(11)$, la curva el\'{\i}ptica $C$
parametriza clases de isomorfismo de curvas el\'{\i}pticas con cierta
estructura de nivel. En \cite{schoofTzanakisLevelEleven} se demuestra que un punto
$\bb{Q}$-racional de $C$, $P$, corresponde a una curva el\'{\i}ptica
cuyo $j$-invariante es un entero (racional), si, y s\'{o}lo si
$P=(x,y)$ tiene la propiedad de que $x/(xy-11)$ pertenece a $\bb{Z}$.
El problema de contar los puntos enteros de $X_{ns}^{+}(11)$ se
convierte en el problema de contar los puntos racionales de $C$ cuyas
coordenadas tienen esta propiedad. El resultado central en
\cite{schoofTzanakisLevelEleven} es:

\begin{teoSchoofTzanakis}\label{thm:teoSchoofTzanakis}
Sea $C$ la curva el\'{\i}ptica dada por la ecuaci\'{o}n de
Weierstra{\ss}

\begin{align*}
y^{2}\,+\,11y & \,=\,x^{3}\,+\,11x^{2}\,+\,33x\text{ .}
\end{align*}
Existen \'{u}nicamente siete puntos $P=(x,y)$ en $C(\bb{Q})$ tales que
$x/(xy-11)$ sea un n\'{u}mero entero.
\end{teoSchoofTzanakis}
Por lo tanto, $X_{ns}^{+}(11)$ cuenta s\'{o}lo con siete puntos
enteros. Por otro lado, estos puntos enteros vienen exclusivamente
de \'{o}rdenes en cuerpos cuadr\'{a}ticos imaginarios: el primo $11$ es
inerte en los \'{o}rdenes cuadr\'{a}ticos con n\'{u}mero de clases
$1$ y de discriminante $-3$, $-4$, $-12$, $-16$, $-27$, $-67$ y $-163$.
En particular, si
$|d|$ es suficientemente grande, si $d<-44$ por ejemplo, como $11$ es
inerte en un orden de discriminante $d$, su n\'{u}mero de clases no
podr\'{a} ser igual a $1$, excepto que $d$ sea uno de los ya
mencionados. Esta soluci\'{o}n del problema difiere del resto en que
el g\'{e}nero de la curva modular es $1$ (si bien para $N=24$ el
g\'{e}nero tambi\'{e}n es $1$, la soluci\'{o}n, en ese caso, viene de
considerar parametrizaciones de las curvas de g\'{e}nero $0$
$X_{ns}^{+}(3)$ y $X_{ns}^{+}(8)$ \cite{booher}).
Resumimos, a continuaci\'{o}n, la demostraci\'{o}n del teorema
\ref{thm:teoSchoofTzanakis}.

En la ecuaci\'{o}n que define a $C$, reemplazando $y$ por
$(y-11)/2$ y luego $y/2$ por $y$, obtenemos

\begin{align*}
y^{2} & \,=\,x^{3}\,+\,11x^{2}\,+\,33x\,+\,\frac{121}{4}
\,=:\,q(x)\text{ .}
\end{align*}
Llamemos $\widetilde{C}$ a la curva el\'{\i}ptica que esta ecuaci\'{o}n
determina.
El polinomio $q$ tiene un \'{u}nico cero en $\bb{R}$. En particular,
dado que, si $(x,y)$ es un punto de orden $2$ de $\widetilde{C}$,
$y$ debe ser igual a $0$ y $x$ una ra\'{\i}z de $q$, el subgrupo
$\widetilde{C}[2]$ no puede estar contenido en $\widetilde{C}(\bb{R})$.
Esto implica que
$\widetilde{C}(\bb{R})$ es isomorfo a $\bb{R}/\bb{Z}$ y, en particular,
tiene una \'{u}nica componente conexa. Todo esto es cierto, tambi\'{e}n,
para la curva $C$.

Sea $t$ la funci\'{o}n en $C$ definida por la expresi\'{o}n
$t=y-(11/x)$. Si $(x,y)$ es un cero de $t$, entonces $x$ es un cero del
polinomio $x^{5}+11x^{4}+33x^{3}-121x-121$. Las ra\'{\i}ces de este
polinomio son reales, con lo cual, los (cinco) puntos $(x,y)$ que son
ceros de la funci\'{o}n $t$ son todos reales. El morfismo
$j:\,X_{ns}^{+}(11)\rightarrow\bb{P}^{1}$ determina un morfismo de la
curva el\'{\i}ptica $C$ en $\bb{P}^{1}$ a trav\'{e}s de un
isomorfismo definido sobre $\bb{Q}$ entre $X_{ns}^{+}(11)$ y $C$.
Este isomorfismo es elegido de manera que las c\'{u}spides de
$X_{ns}^{+}(11)$, que son cinco, se coprrespondan con los ceros de $t$
(ver \cite{schoofTzanakisLevelEleven}, y las referencias que all\'{\i} se
encuentran).

Sea $\omega=\omega_{C}=dx/(2y+11)$ el diferencial invariante de $C$.
La integral $\int_{\cal{O}}^{P}\,\omega$ (donde $\cal{O}$ es el
punto neutro de la curva el\'{\i}ptica y $P$ un punto arbitrario)
no est\'{a} bien definida como elemento de $\bb{C}$, pero las
posibles ambig\"{u}edades surgen de la elecci\'{o}n del camino entre
los puntos. As\'{\i}, \textit{modulo} el ret\'{\i}culo que se le asocia
a $C$ v\'{\i}a sus per\'{\i}odos, podemos definir $\lambda(P)$ como el
valor de esta integral \textit{modulo} el ret\'{\i}culo. Ahora, si
$P$ es un punto real de $C$, entonces existe un camino que lo une con
$\cal{O}$ y que est\'{a} contenido en $C(\bb{R})$. Esto muestra que
$\lambda(P)$ es un n\'{u}mero real, si $P$ es un punto real. El grupo
de puntos reales de $C$ es isomorfo a $\bb{R}/\bb{Z}$ v\'{\i}a la
aplicaci\'{o}n $\lambda$ y la elecci\'{o}n de un per\'{\i}odo real
para $C$. Por ejemplo, (identificando las curvas $C$ y $\widetilde{C}$)
se puede tomar
$\Omega:=\int_{r}^{\infty}\,dx/\sqrt{q(x)}$, donde $r$ es la
ra\'{\i}z real del polinomio $q$ definido antes.
% [[silverman, ACE cap.V?]].

Antes de pasar a la demostraci\'{o}n, es necesaria una \'{u}ltima
definici\'{o}n. Si $f$ es una funci\'{o}n en la curva $C$ y si no es
constante, dado un punto racional, $P\in C(\bb{Q})$, definimos la
altura de $P$ (respecto de $f$) como el producto
$H_{f}(P):=\prod_{v}\,\rm{max}\{1,|f(P)|_{v}\}$, donde $v$ recorre
los lugares de $\bb{Q}$. La altura logar\'{\i}tmica asociada es
$h_{f}(P):=\rm{log}(H_{f}(P))$ y, finalmente, si $f$ es, adem\'{a}s,
una funci\'{o}n par, la altura can\'{o}nica se define como

\begin{align*}
\widehat{h}(P) & \,:=\,\frac{1}{\rm{deg}(f)}
\rm{lim}_{n\rightarrow\infty}\,\frac{h_{f}([2^{n}]P)}{4^{n}}\text{ .}
\end{align*}
La funci\'{o}n $t=y-(11/x)$ no es par, pero se la puede relacionar con
la altura can\'{o}nica: si $P\in C(\bb{Q})$, entonces
$\widehat{h}(P)$ est\'{a} acotada por $(1/3)h_{t}(P)+4,52$.

La idea de la demostraci\'{o}n del teorema \ref{thm:teoSchoofTzanakis} es
traducir las restricciones sobre un punto $(x,y)$ tal que
$x/(xy-11)\in\bb{Z}$ en estimaciones de una forma lineal en logaritmos.
Supongamos que el rango del grupo $C(\bb{Q})$ es $r\geq 1$, y sean
$P_{1},\,\dots,\,P_{r}$ generadores de la parte libre. Todo punto $P$
se puede escribir como una combinaci\'{o}n
$m_{1}P_{1}+\,\dots\,+m_{r}P_{r}+T$, donde los $m_{i}$ son enteros y $T$ es
un punto de torsi\'{o}n.
%Por otro lado, el punto $\cal{O}$ es un polo de $t$,
%entonces, si buscamos puntos racionales $P=(x,y)$ tales que $x/(xy-11)$ sea
%mayor que cierta constante $M$, resulta que, en estos puntos $|t(P)|$ est\'{a}
%acotado por $1/M$; usando $t$ como est\'{a}ndar, estos puntos est\'{a}n
%alejados de la identidad.
El primer objetivo es obtener una cota para $|t(P)|$
dependiente de los coeficientes $m_{i}$ y, as\'{\i}, pasar a
una cota superior sobre cierta forma lineal en $\lambda(P_{i})$ y en $\Omega$.
El rango de la curva el\'{\i}ptica $C$ es $r=1$, lo
que hace que sea m\'{a}s sencillo estimar la forma lineal. El mismo m\'{e}todo,
aplicado a encontrar puntos enteros en una curva el\'{\i}ptica de rango no
necesariamente igual a $1$ est\'{a} descripto en \cite{stroekerTzanakis}.

Sea $P=(x,y)$ un punto de $C$, y sea $k\in\bb{Z}$ tal que
$k=x/(xy-11)$. Entonces $t(P)=y-(11/x)$ es igual a $1/k$. Si $k$ es
mayor que $20$, exite un \'{u}nico punto $Q\in C$, correspondiente a
una de las c\'{u}spides de $X_{ns}^{+}(11)$ y un intervalo $I$ contenido
en $U:=\{\widetilde{P}\in C(\bb{R})\,:\,t(\widetilde{P})<1/20\}$
tal que $P,Q\in I$ (lema 2.1 en \cite{schoofTzanakisLevelEleven}). Si $\omega$ es el
diferencial invariante de $C$, entonces $\omega=dx/F_{y}=-dy/F_{x}$.
Si $f$ es una funci\'{o}n en la curva, su diferencial, $df$, es igual a

\begin{align*}
df & \,=\,f_{x}dx\,+\,f_{y}dy\,=\,
\left(f_{x}F_{y}\,-\,f_{y}F_{x}\right)\,\omega\text{ .}
\end{align*}
Denotaremos con $g$ al t\'{e}rmino entre par\'{e}ntesis.
Como $P$ y $Q$ pertenecen al intervalo $I$, podemos considerar
que $\int_{Q}^{P}\,\omega$ es la integral de $Q$ a $P$ por
un camino contenido en $I$. En $I$, el valor de $|g|$ se puede
acotar por $1$, y as\'{\i}

\begin{align*}
\left|\int_{Q}^{P}\omega\right| & \,=\,
\left|\int_{0}^{t(P)}\frac{dt}{g}\right|\,\leq|t(P)|
\text{ .}
\end{align*}
Entonces, para alg\'{u}n entero $m$, tenemos la cota
$|\lambda(P)-\lambda(Q)+m\Omega|\leq|t(P)|$. Por otra parte, como
$Q$ corresponde a una c\'{u}spide, $\lambda(Q)=(m'/11)\Omega$
para alg\'{u}n entero $m'$ (ver lema 3.1 en \cite{schoofTzanakisLevelEleven}), y,
al ser $1/t(P)$ un entero, la altura logar\'{\i}tmica $h_{t}(P)$ del
punto $P$ es igual a $-\rm{log}|t(P)|$. En definitiva, la siguiente cota
es v\'{a}lida para cualquier punto $P$ tal que $1/t(P)$ sea un entero
mayor que $20$.

\begin{align*}
\left| n\frac{\Omega}{11}\,-\,\lambda(P) \right| &
\,<\,\rm{exp}(13,56-3\widehat{h}(P))\text{ .}
\end{align*}

El grupo $C(\bb{Q})$ es c\'{\i}clico infinito, generado por
$P_{0}=(0,0)$. Si $P$ es un punto como los considerados en el
p\'{a}rrafo anterior, $P=[m]P_{0}$. De esta igualdad se obtiene una
cota superior sobre la forma lineal $n\Omega-m\lambda(11P_{0})$. Por
otro lado, por el hecho de que $P_{0}$ no es un punto de torsi\'{o}n
de la curva, se puede obtener una cota inferior.
Si $|m|\geq 12$, se deduce que $|m|$ debe estar acotado por una
constante $A$. En principio, la cota dada por $A$ no es suficiente,
pero, se puede deducir que
$|n\Omega-m\lambda(11P_{0})|\leq 0,4\Omega/|m|$. En particular,

\begin{align*}
\left| \frac{n}{m}\,-\,\frac{\lambda(11P_{0})}{\Omega} \right|
&\,<\,\frac{1}{2m^{2}}\text{ .}
\end{align*}
Calculando los valores de $\lambda(11P_{0})$ y de $\Omega$ con
precisi\'{o}n suficiente, se verifica que cualquier aproximaici\'{o}n
$p_{k}/q_{k}$ por fracciones continuas del cociente
$\lambda(11P_{0})/\Omega$, o bien no satisface $q_{k}<A$, o bien no
satisface la cota superior sobre
$|p_{k}\Omega-q_{k}\lambda(11P_{0})|$.
Se deduce que el entero $|m|$ tiene que ser menor a $12$. Luego
se verifica que los \'{u}nicos puntos $P=(x,y)=mP_{0}$ tales que
$x/(xy-11)$ es un n\'{u}mero entero son aquellos para los que
$m$ es igual a $-2$, $-1$, $0$, $1$, $2$, $3$ o a $4$.

Resta ver qu\'{e} sucede si $k\leq 20$. En este caso, se verifica que
$x/(xy-11)=k$ tiene soluciones con $(x,y)\in C(\bb{Q})$, s\'{o}lo si
$k$ es igual a $2$, $0$, $-2$, $-6$ o a $-8$. Los puntos que quedan
determinados por este valor de $k$ son los mismos que en el p\'{a}rrafo
anterior: $mP_{0}$ con $m$ como arriba.

\begin{obsGenUno}\label{thm:obsGenUno}
 Ya mencionamos que una particularidad de esta soluci\'{o}n es que %borra ``de la''
 hace uso de una curva de g\'{e}nero $1$. Sin embargo, esto no quiere decir %agrega puntuaci\'{o}n
 que los m\'{e}todos utilizados para estudiar las curvas de g\'{e}nero $0$
 no sean \'{u}tiles en este caso.
 De hecho, en \cite{baranNormalizers} se
 estudian las curvas $X_{ns}^{+}(21)$ de g\'{e}nero $1$ y
 $X_{ns}^{+}(16)$ y $X_{ns}^{+}(20)$ de g\'{e}nero $2$.
 Por \'{u}ltimo mencionamos que un estudio de las representaciones cuspidales
 de los grupos $\GL{2}(\bb{F}_{p})$ permite tratar la curva
 $X_{ns}^{+}(13)$ de g\'{e}nero $3$ y obtener una ecuaci\'{o}n sobre $\bb{Q}$
 para la misma \cite{baranAnExceptionalIso}.

\end{obsGenUno}


 \end{subsection}
\end{section}

\begin{thebibliography}{99}

 \bibitem{baranLevelNine} B. Baran. \emph{A Modular Curve of Level
 $9$ and the Class Number One Problem}. Journal of Number Theory,
 129 (2009), 715-728.

 \bibitem{baranNormalizers} B. Baran. \emph{Normalizers of
 Non-split Cartan Subgroups, Modular Curves, and the Class Number
 One Problem}. Journal of Number Theory, 130 (2010), 2753-2772.

 \bibitem{baranAnExceptionalIso} B. Baran. \emph{An Exceptional
 Isomorphism Between Modular Curves of Level $13$}. Journal of Number
 Theory, 145 (2014), 273-300.

 \bibitem{booher} J. Booher. \emph{Modular Curves and the Class
 Number One Problem}.

 \bibitem{chenLevelFive} I. Chen. \emph{On Siegel's Modular Curve
 of Level $5$ and the Class Number One Problem}. Journal of
 Number Theory, 74 (1999), 278-297.

 \bibitem{cox} D. A. Cox. \emph{Primes of the Form $x^{2}+ny^{2}$:
 Fermat, Class Field Theory, and Complex Multiplication}. Wiley, 1989.

 \bibitem{deligneRapoport} P. Deligne, M. Rapoport. \emph{Les sch\'{e}mas
 de modules de courbes elliptiques}. Antwerp, 1972.

 \bibitem{diamondShurman} F. Diamond, J. Shurman. \emph{A
 First Course in Modular Forms}. Springer-Verlag New York, 2005.

 \bibitem{kenkuLevelSeven} M. A. Kenku. \emph{A Note on the Integral
 Points of a Modular Curve of Level $7$}. Mathematika, 32 (1985),
 45-48.

 \bibitem{ligozat} G. Ligozat. \emph{Courbes modulaires de niveau $11$}.
 En J.-P. Serre, D. Zagier, eds. \emph{Modular Functions of One Variable},
 vol. V. Springer-Verlag New York, 1977.

 \bibitem{schoofTzanakisLevelEleven} R. Schoof, N. Tzanakis.
 \emph{Integral Points of a Modular Curve of Level $11$}. Acta
 Arithmetica, 152 (2012), 39-49.

 \bibitem{stroekerTzanakis} R. J. Stroeker, N. Tzanakis.
 \emph{Solving Elliptic Diophantine Equations by Estimating Linear Forms
 in Elliptic Logarithms}. Acta Arithmetica, 67 (1994), 177-196.

 \bibitem{serre} J.-P. Serre. \emph{Lectures on the
 Mordell-Weil Theorem}. $3^{a}$ ed. Vieweg+Teubner Verlag, 1997.

\end{thebibliography}
\end{document}
