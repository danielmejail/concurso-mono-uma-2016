\begin{subsection}{Subgrupos de Cartan}
Sea $E$ una curva el\'{\i}ptica con CM. Su anillo de endomorfismos es un orden
$I=[1,\tau]$ en un cuerpo cuadr\'{a}tico imaginario. Si $N>1$ es un entero coprimo
con el discriminante de $I$ y $p\in\bb{Z}$ es un primo divisor de $N$, entonces
$p$ se parte o es inerte en $I$. Sea $A:=I/NI$. Este anillo es una
$(\bb{Z}/N\bb{Z})$-\'{a}lgebra con base $\{1,\tau\}$, y, dependiendo de $p$, si se
parte o no en $I$, el cociente $A/pA$ es isomorfo a $\bb{F}_{p}\times\bb{F}_{p}$
o a $\bb{F}_{p^{2}}$, respectivamente.

\begin{defSubgrupoDeCartan}
Dada $A$ un \'{a}lgeba sobre $\bb{Z}/N\bb{Z}$, libre, conmutativa, de rango $2$ y
tal que, si $p|N$, entonces $A/pA$ es isomorfo a $bb{F}_{p}\times\bb{F}_{p}$ o a
$\bb{F}_{p^{2}}$, el grupo de unidades, $A^{\times}$, act\'{u}a sobre $A$ por
multiplicaci\'{o}n. Elegir una base de $A$ sobre $\bb{Z}/N\bb{Z}$, equivale a
establecer un morfismo

\begin{align*}
\iota & \,:\,A^{\times}\hookrightarrow\GL{2}(\bb{Z}/N\bb{Z})
\end{align*}
--elegir otra base resulta en un subgrupo conjugado a $\iota(A^{\times})$.
Decimos que un subgrupo de $\GL{2}(\bb{Z}/N\bb{Z})$ es \emph{de Cartan}, si es de
la forma $\iota(A^{\times})$ para alg\'{u}n \'{a}lgebra $A$ que cumpla con las
condiciones especificadas. Con respecto a la \'{u}ltima de estas condiciones,
tambi\'{e}n se dice que $A$ es \textit{\'{e}tale}. Si $A/pA\simeq\bb{F}_{p^{2}}$,
se dice que $A$ es \textit{non-split} en $p$, y \textit{split} en el otro caso.
Un subgrupo de Cartan se dice \textit{non-split}, si el \'{a}lgebra correspondiente
lo es en todo primo que divide a $N$.

\end{defSubgrupoDeCartan}

\begin{ejemploSubgrupoDeCartan}
Si $I$ es un orden en un cuerpo cuadr\'{a}tico imaginario, $A=I/NI$ y
$(N,\rm{disc}(I))=1$, entonces $A$ es \textit{non-split} y, fijando una base,
$\iota(A^{\times})\subset\GL{2}(\bb{Z}/N\bb{Z})$ es un subgrupo de Cartan
\textit{non-split}. Denotamos con $C_{ns}(N)$ al subgrupo $\iota(A^{\times})$.

Si $N=p$ es  primo, $|A^{\times}|=|\bb{F}_{p^{2}}^{\times}|=p^{2}-1$.
Si $N=p^{r}$, $r\geq 1$, todo elemento de $(I/pI)^{\times}$ se levanta a una
unidad en $I/p^{r}I$, y cada elemento tiene $p^{2(r-1)}$ preim\'{a}genes;
entonces

\begin{align*}
|(I/p^{r}I)^{\times}| & \,=\, p^{2(r-1)}(p^{2}-1)\text{ .}
\end{align*}
Para $N>1$ entero coprimo con $\rm{disc}(I)$, el orden $|A^{\times}|$ es igual a
$|(I/NI)^{\times}|=N^{2}\prod_{p|N}\,(1-(1/p^{2}))$. Este es el orden del grupo de
Cartan \textit{non-split} $C_{ns}(N)$.
%
%\begin{align*}
%a+b\tau\in(I/pI)^{\times}\Rightarrow\exists c,d
%\,|\,(a+b\tau)(-c+d\tau)\equiv 1\,(pI)
%a,b,c,d<p
%-ac-v=1+kp & \text{ tomar } c=c+sp\text{ para que } -ac-v\equiv 1\,(p^{r})\\
%ad-bc+u=k'p\text{ tomar } d=d+s'p\text{ para que } ad-bc+u\equiv 0\,(p^{r})
%\end{align*}
%cada $a+b\tau$ se levanta a $I/p^{r}$ de $p^{2(r-1)}$ maneras.

\end{ejemploSubgrupoDeCartan}

\end{subsection}

\begin{subsection}{El normalizador de $C_{ns}(N)$ en $\GL{2}(\bb{Z}/N\bb{Z})$%
			y la curva modular asociada}
Sean $I=[1,\tau]$ un orden en un cuerpo cuadr\'{a}tico imaginario $K$ tal que
$(N,\rm{disc}(I))=1$ y $A=I/NI$. Sea $C_{ns}(N)=\iota(A^{\times})$. Si $\tau$
satisface el polinomio $X^{2}-uX+v\in\bb{Z}[X]$, definimos una involuci\'{o}n
en $\tau$ por $\overline{\cdot}:\,\tau\mapsto(u-\tau)$, y
$\sigma_{p}:\,A\rightarrow A$ como el \'{u}nico automorfismo en $A=I/NI$ que
coincide con esta involuci\'{o}n en $I/p^{r(p)}I$ y con la identidad en
$I/\frac{N}{p^{r}}I$. La misma elecci\'{o}n de base que determina la
inclusi\'{o}n $\iota$ determina una matriz $S_{p}\in\GL{2}(\bb{Z}/N\bb{Z})$
que representa a $\sigma_{p}$.

Sea $N=p^{r}$, $p\in\bb{Z}$ un primo y $r\geq 1$. Podemos identificar $C_{ns}(N)$
con el grupo $A^{\times}$ de unidades de $A=I/p^{r}I$. Si $\alpha$ es un elemento
de $A$, tiene una matriz asociada, en $\MM{2}(\bb{Z})$:
\begin{math}
 \gamma(\alpha):=
 \left[\begin{smallmatrix}a&b\\c&d\end{smallmatrix}\right]
\end{math},
determinada por

\begin{align*}
 \alpha\cdot\begin{bmatrix} 1\\ \tau \end{bmatrix} & \,=\,
 \begin{bmatrix} \alpha\\ \alpha\tau \end{bmatrix} \,=\,
 \begin{bmatrix} a+b\tau\\ c+d\tau \end{bmatrix} \,=\,
 \begin{bmatrix} a&b\\c&d \end{bmatrix}
 \begin{bmatrix} 1\\ \tau \end{bmatrix}\text{ .}
\end{align*}
Si $\tau^{2}-u\tau+v=0$, con $u$ y $v$ en $\bb{Z}$, la matriz de $\tau$ es
\begin{math}
 \gamma(\tau):=
 \left[\begin{smallmatrix}0&1\\-v&u\end{smallmatrix}\right]
\end{math}.
Como $A$ es \textit{non-split}, $A/pA\simeq\bb{F}_{p^{2}}$. Pero $\tau\not\in pA$,
%$#\bb{F}_{p^{2}}=p^{2}$. Si $\tau\in pA$, $A/pA$ est\'{a} generado por $1$
%y $\tau$ sobre $\bb{Z}/p\bb{Z}$ implica cardinalidad menor.
con lo que $\tau$ es una unidad en $I/pI$, es decir que la matriz $\gamma(\tau)$
es una unidad en $\MM{2}(\bb{Z}/p\bb{Z})$. Entonces, $v=\det(\gamma(\tau))$
es una unidad en $\bb{Z}/p\bb{Z}$. En particular, $p$ no divide a $v$, y
$\det(\gamma(\tau))\in(\bb{Z}/p^{r}\bb{Z})^{\times}$. En otras palabras,
$\tau$ es una unidad en $A$.

Si ahora tomamos $\kappa\in\GL{2}(\bb{Z}/p^{r}\bb{Z})$ tal que
$\kappa C_{ns}(p^{r})=C_{ns}(p^{r})\kappa$, $\kappa$ induce un automorfismo de
$A^{\times}$ por conugaci\'{o}n: si $\alpha\in A^{\times}$, definimos
$t_{\kappa}(\alpha)$ como la primera coordenada de

\begin{align*}
 & \kappa\gamma(\alpha)\kappa^{-1}\cdot\begin{bmatrix}1\\ \tau\end{bmatrix}
 \text{ .}
\end{align*}
Si $n\in\bb{Z}$ es coprimo con $p$, $n\cdot 1_{A}\in A^{\times}$ y
$\gamma(n1_{A})$ es la matriz diagonal
\begin{math}
\left[\begin{smallmatrix} n&0\\0&n\end{smallmatrix}\right]
\end{math}.
As\'{\i}, $t_{\kappa}(n1_{A})=n1_{A}$. Extendemos $t_{\kappa}$ a un endomorfismo
de la $(\bb{Z}/p^{r}\bb{Z})$-\'{a}lgebra $A$. En particular,
$t_{\kappa}(\tau)\in A$ tiene que ser un cero de $f=X^{2}-uX+v$. Las soluciones
$\tau$ y $u-\tau$ de $f$ son distintas en $I/pI\simeq\bb{F}_{p^{2}}$ y
$f'(\tau)=\tau-(u-\tau)\not =0$, entonces,
si $\alpha\in I$ es soluci\'{o}n de $f$ \textit{modulo} $p^{r'}I$ para alg\'{u}n
$r'$, por el argumento del lema de Hensel, se levanta a \'{u}nica soluci\'{o}n
\textit{modulo} $p^{r'+1}I$. Entonces $t_{\kappa}(\tau)=\tau$ o $u-\tau$, con lo
que el automorfismo $t_{\kappa}$ de $A$ es, o bien, $\sigma_{p}$, o bien la
identidad. En t\'{e}rminos de matrices, o bien $\kappa z\kappa^{-1}=z$, o bien
$\kappa z\kappa^{-1}=S_{p}zS_{p}$ ($S_{p}$ tiene orden $2$). O bien $\kappa$,
o bien $S_{p}\kappa$, pertenece al centralizador de $C_{ns}(p^{r})$. Pero el
centralizador de $C_{ns}(p^{r})$ es el mismo grupo. As\'{\i}
$\kappa$ pertenece a $C_{ns}(p^{r})$, o a $S_{p}C_{ns}(p^{r})$, es decir que el
normalizador de $C_{ns}(p^{r})$ es el subgrupo de $\GL{2}(\bb{Z}/p^{r}\bb{Z})$
generado por $C_{ns}(p^{r})$ y por $S_{p}$. En general, para $N>1$,
%por el teorema chino del resto,
el normalizador de $C_{ns}(N)$ es $\langle C_{ns}(N),\{S_{p}\,:\,p|N\} \rangle$.
Lo denotamos $C_{ns}^{+}(N)$. Su orden es

\begin{align*}
N^{2}2^{\omega}\prod_{p|N}\,\left(1-\frac{1}{p^{2}}\right) &
\,=\,|A^{\times}|2^{\omega}\text{ ,}
\end{align*}
donde $\omega$ es la cantidad de primos distintos que dividen a $N$.

Los grupos $C_{ns}(N)$ y $C_{ns}^{+}(N)$ son subgrupos de $\GL{2}(\bb{Z}/N\bb{Z})$,
y, como tales, podemos asociarles las curvas modulares

\begin{align*}
 X_{ns}(N) & \,:=\,C_{ns}(N)\backs X(N)\\
 X_{ns}^{+}(N) &\,:=\,C_{ns}^{+}(N)\backs X(N)\text{ .}
\end{align*}
Llamamos $Y_{ns}(N)$ e $Y_{ns}^{+}(N)$ a los abiertos que son complementos de las
c\'{u}spides de $X_{ns}(N)$ y $X_{ns}^{+}(N)$, respectivamente. La curva
$X_{ns}^{+}(N)$ est\'{a} definida sobre $\bb{Q}$:
$\det(C_{ns}^{+}(N))=(\bb{Z}/N\bb{Z})^{\times}$ [[Booher]], el argumento es el
siguiente: sean $m\in\bb{Z}$ coprimo con $N$, y $p|N$ un primo. Como $p$ se
supone inerte en $K$ (el cuerpo cuadr\'{a}tico que contiene al orden
$I=[1,\tau]$), $K_{p}/\bb{Q}_{p}$ es una extensi\'{o}n cuadr\'{a}tica de cuerpos
locales. Existe, entonces, $\mu_{p}\in K_{p}^{\times}$ \'{\i}ntegro tal que

\begin{align*}
 \Nm_{K_{p}/\bb{Q}_{p}}(\mu_{p})& \,=\,\pm m\text{ .}
\end{align*}
Si $p^{r(p)}||N$, elegimos $\widetilde{\mu}_{p}\in I$ que aproxime $\mu_{p}$ a
orden $p^{r(p)}$:

\begin{align*}
 \mu_{p} & \,\equiv\,\widetilde{\mu}_{p}\,(p^{r(p)})\text{ .}
\end{align*}
As\'{\i}, $\Nm_{K_{p}/\bb{Q}_{p}}(\widetilde{\mu}_{p})$ es congruente con
$\pm m\,(p^{r(p)})$. Haciendo esto para cada primo que divide a $N$, se toma
$\mu\in I$ que satisfaga $\mu\equiv\widetilde{\mu}_{p}$ \textit{modulo} $p^{r(p)}$
para cada $p$. De esta manera, $\Nm_{K/\bb{Q}}(\mu)\equiv \pm m\,(p^{r(p)})$ para
cada primo. Corrigiendo con las matrices $S_{p}$ (cuyo determinante es $-1$ en el
factor correspondiente a $p$, y $1$ en el resto), vemos que
$\det:\,C_{ns}^{+}(N)\rightarrow(\bb{Z}/N\bb{Z})^{\times}$ es sobreyectiva.

\begin{obsDetSobreUnidades}
 Hemos demostrado, tambi\'{e}n, que

 \begin{align*}
  \det & \,:\,C_{ns}(p^{r})\rightarrow(\bb{Z}/p^{r}\bb{Z})^{\times}/\{\pm 1\}
 \end{align*}
es sobre.

\end{obsDetSobreUnidades}

Sean $C_{ns}^{+}(N)'=C_{ns}^{+}(N)\cap\SL{2}(\bb{Z}/N\bb{Z})$ y
$\Gamma_{C_{ns}^{+}(N)}$ el subgrupo de $\SL{2}(\bb{Z})$ conformado por las
matrices que caen en $C_{ns}^{+}(N)'$ al reducir las coordenadas \textit{modulo}
$N$. Como $\det:\,C_{ns}^{+}(N)\rightarrow(\bb{Z}/N\bb{Z})^{\times}$ es
sobreyectiva y su n\'{u}cleo es $C_{ns}^{+}(N)'$, el \'{\i}ndice
$|C_{ns}^{+}(N):C_{ns}^{+}(N)'|$ es igual a $\phi(N)$, con lo cual, 

\begin{align*}
 |\SL{2}(\bb{Z}):\Gamma_{C_{ns}^{+}(N)}| & \,=\\
 |\SL{2}(\bb{Z}/N\bb{Z}):C_{ns}^{+}(N)'| & \,=\,
 \frac{\phi(N)}{2^{\omega}N^{2}\prod_{p|N}\,(1-(1/p^{2}))}
 N^{3}\prod_{p|N}\,(1-(1/p^{2})) \,=\,\frac{N\phi(N)}{2^{\omega}}
 \text{ .}
\end{align*}
Tenemos morfismos

\begin{align*}
 & X_{ns}(N)\xrightarrow{\Phi_{1}}
 X_{ns}^{+}(N)\xrightarrow{\Phi_{2}} X(1)\text{ ,}
\end{align*}
de grados $\rm{deg}(\Phi_{1})=2^{\omega}$ y
$\rm{deg}(\Phi_{2})=N\phi(N)/2^{\omega}$.

En resumen, existe una curva modular, $X_{ns}^{+}(N)$, isomorfa a
$\Gamma_{C_{ns}^{+}(N)}\backs\frak{h}^{*}$, y definida sobre $\bb{Q}$.
El abierto
$Y_{ns}^{+}(N)=X_{ns}^{+}(N)\setmin\{\text{ c\'{u}spides }\}$
parametriza clases de equivalencia de pares $(E,\varphi)$, donde
$E/\bb{C}$ es una curva el\'{\i}ptica y
$\varphi:\,E[N]\rightarrow(\bb{Z}/N\bb{Z})^{2}$ un isomorfismo; dos pares
$(E,\varphi)$ y $(E',\varphi')$ son equivalentes, si existe
$\gamma\in C_{ns}^{+}(N)$ tal que los puntos $[E,\varphi]$ y
$[E',\gamma\circ\varphi']$ sean iguales en $X(N)$. Olvidando la estructura de
nivel, $\varphi$ en cada par $(E,\varphi)$, resulta un morfismo de $X_{ns}^{+}(N)$
en $X(1)$ de grado $N\phi(N)/2^{\omega}$. Si llamamos $g$ al g\'{e}nero de esta
curva,

\begin{align*}
 g & \,=\,1\,+\,\frac{\mu}{12}\,-\,\frac{\mu_{2}}{4}\,-\,\frac{\mu_{3}}{6}
 \,-\,\frac{\mu_{\infty}}_{2}\text{ ,}
\end{align*}
donde $\mu:=|\SL{2}(\bb{Z}):\Gamma_{C_{ns}^{+}(N)}|$, $\mu_{k}$ es la cantidad
de puntos el\'{\i}pticos de orden $k=2,3$ y $\mu_{\infty}$ es la cantidad de
c\'{u}spides.

\end{subsection}

\begin{subsection}{Puntos racionales e \'{\i}ntegros}
 En tanto $X_{ns}^{+}(N)$ tiene un modelo sobre $\bb{Q}$, podemos referirnos a sus
 puntos racionales y a sus puntos enteros. Para $N=5$ o $N\geq 7$, los puntos
 $\bb{Q}$-racionales de $X_{ns}^{+}(N)$ pertenecen a $Y_{ns]^{+}(N)$, es decir,
 no est\'{a}n entre las c\'{u}spides [[Baran, Normalizers]]. Recordemos que
 existe un morfismo $X(N)\rightarrow X_{ns}^{+}(N)$ que nos permite ver
 $X_{ns}^{+}(N)$ como un cociente de $X(N)$ y sus c\'{u}spides como \'{o}rbitas
 por la acci\'{o}n de $C_{ns}^{+}(N)$ sobre las c\'{u}spides de $X(N)$. Si
 $[E,\varphi]\in Y(N)(\overline{\bb{Q}})$ y $\sigma\in\absGal{\bb{Q}}$,
 
 \begin{align*}
  [E,\varphi]^{\sigma} & \,=\,[E^{\sigma},\varphi\circ\sigma]
 \end{align*}
determina una acci\'{o}n del grupo de Galois absoluto sobre los puntos
$\overline{\bb{Q}}$-racionales en el abierto $Y(N)$. Adem\'{a}s, si $E$ es una
curva el\'{\i}ptica definida sobre $\bb{Q}$, $\absGal{\bb{Q}}$ act\'{u}a sobre
$E[N]$, dando lugar a una representaci\'{o}n

 \begin{align*}
  \rho_{N} & \,:\,\absGal{\bb{Q}}\rightarrow\rm{Aut}(E[N])
  \,\simeq\,\GL{2}(\bb{Z}/N\bb{Z})\text{ ,}
 \end{align*}
donde $\rm{Aut}(E[N])$ se identifica con $\GL{2}(\bb{Z}/N\bb{Z})$ v\'{\i}a
el isomorfismo $\varphi$ en el par $[E,\varphi]$.

 Sea $[E,\varphi]\in Y(N)(\overline{\bb{Q}})$ y $\overline{[E,\varphi]}$ su
 imagen en $Y_{ns}^{+}(N)(\overline{\bb{Q}})$. Este punto es la \'{o}rbita
 de $[E,\varphi]$ por $C_{ns}^{+}(N)$:

 \begin{align*}
  \overline{[E,\varphi]} & \,=\,\left\lbrace
  [E,\gamma\circ\varphi]\,:\,\gamma\in C_{ns}^{+}(N)\right\rbrace
  \text{ .}
 \end{align*}
Y esta \'{o}rbita define un punto $\bb{Q}$-racional en $Y_{ns}^{+}(N)$, si,
y s\'{o}lo si es estable por la acci\'{o}n de Galois: si para $\sigma$ en
$\absGal{\bb{Q}}$ existe $\gamma_{\sigma}$ en $C_{ns}^{+}(N)$ tal que
$[E^{\sigma},\varphi\circ\sigma]=[E,\gamma_{\sigma}\circ\varphi]$. Para $E$
definida sobre $\bb{Q}$, esto equivale a que la imagen de Galois por la
representaci\'{o}n $\rho_{N}$ est\'{e} contenida en el normalizador de un
subgrupo de Cartan \textit{non-split}.

 Sea $I$ un orden en un cuerpo cuadr\'{a}tico imaginario $K$, sea $\Cl(I)$ su
 grupo de clases y sea $h(I)=#\Cl(I)$. Sea $\rm{CM}(I)$ el conjunto de curvas
 el\'{\i}pticas definidas sobre $\bb{C}$ con anillo de endomorfismos $I$,
 salvo $\bb{C}$-isomorfismo, y sea $\psi:\,\Cl(I)\rightarrow\rm{CM}(I)$ la
 biyecci\'{o}n dada por $M\mapsto\bb{C}/M$ sobre un $I$-ideal $M$ en $K$
 [[Cox]].
%
 Si $h(I)=1$, hay una \'{u}nica curva el\'{\i}ptica con CM por $I$, salvo
 isomorfismo, y, en consecuencia, el orden $I$ determina un punto
 $j(I)=j(E)\in X(1)$, donde $E$ es cualquier representante de $\rm{CM}(I)$.
 Dado que $j(E^{\sigma})=j(E)^{\sigma}$ para $\sigma\in\rm{Aut}(\bb{C}/\bb{Q})$,
 y que $E^{\sigma}$ tambi\'{e}n es CM con anillo de endomorfismos $I$,
 $j(E)^{\sigma}=j(E)$, y $j(E)$ queda fijo por $\rm{Aut}(\bb{C}/\bb{Q})$. Al ser
 racional, pertenece a $\bb{Z}$ (el $j$-invariante de una curva el\'{\i}ptica
 con multiplicaci\'{o}n compleja es un entero algebraico).

 Sea $K$ un cuerpo cuadr\'{a}tico imaginario con n\'{u}mero de clases $1$, y sea
 $\cal{O}_{K}$ su anillo de enteros. Fijemos $E$ una curva el\'{\i}ptica definida
 sobre $\bb{Q}$, con CM por $\cal{O}_{K}$ y con $j$-invariante
 $j(E)=j(\cal{O}_{K})\in\bb{Z}$. Fijamos, tambi\'{e}n, un entero $N\geq1$ coprimo
 con el discriminante de $\cal{O}_{K}$, y un isomorfismo
 $\varphi:\,E[N]\rightarrow(\bb{Z}/N\bb{Z})^{2}$. Sea $\rho_{N}$ la
 representaci\'{o}n de $\absGal{\bb{Q}}$ en $\GL{2}(\bb{Z}/N\bb{Z})$
 correspondiente a $\varphi$, es decir, identificando $\GL{2}(\bb{Z}/N\bb{Z})$
 con $\rm{Aut}(E[N])$ v\'{\i}a $\varphi$. Por restricci\'{o}n, contamos con un
 morfismo de anillos
 $f:\,\rm{End}(E)=\cal{O}_{K}\rightarrow\rm{End}(E[N])$ que se factoriza por
 $N\cal{O}_{K}$,

 \begin{align*}
  f' & \,:\,A\,=\,\cal{O}_{K}/N\cal{O}_{K}\rightarrow\rm{End}(E[N])\text{ .}
 \end{align*}
V\'{\i}a $f'$ y $\varphi$, obtenemos el grupo de Cartan \textit{non-split}
$f'(A^{\times})\subset\rm{Aut}(E[N])=\GL{2}(\bb{Z}/N\bb{Z})$. Llamemos
$C_{ns}(N)$, $G$ y $C$ a los subgrupos de $\GL{2}(\bb{Z}/N\bb{Z})$
$f'(A^{\times})$, $\rho_{N}(\absGal{\bb{Q}})$ y
$\rho_{N}(\Gal(\overline{\bb{Q}}/K)$, respectivamente. Si $\tau\in\cal{O}_{K}$,
$\sigma\in\absGal{\bb{Q}}$ y $w\in E[N]$, o bien $\tau=\tau^{\sigma}$, o bien
$\tau\not =\tau^{\sigma}$. En el primer caso,
$\tau\cdot w^{\sigma}=(\tau w)^{\sigma}$ y $\rho_{N}(\sigma)$ conmuta con
$f'(\tau)$. En el segundo, $(\tau w^{\sigma})^{\sigma{-1}}=\tau^{\sigma^{-1}}w$
y, como $\tau^{\sigma^{-1}}$ pertenece a $\cal{O}_{K}$ (la extensi\'{o}n
$K/\bb{Q}$ es normal), $\rho_{N}(\sigma)$ pertenece al normalizador de
$f'(A^{\times})$. En definitiva, $C\subset C_{ns}(N)$ y
$G\subset C_{ns}^{+}(N)$.

\begin{propoPuntosEnterosCuerpoQuad}
 Sea $K$ un cuerpo cuadr\'{a}tico imaginario y sea $N\geq 1$ un entero tal que
 todo primo $p|N$ es inerte en $K$. Si el n\'{u}merp de clases de $K$ es igual 
 a $1$, cualquier curva el\'{\i}ptica con CM por $K$ da lugar a un punto
 $\overline{[E,\varphi]}\in Y_{ns}^{+}(N)(\overline{\bb{Q}})$, donde $E$ est\'{a}
 definida sobre $\bb{Q}$ y la imagen de la representaci\'{o}n de Galois
 asociada, $\rho_{N}$, est\'{a} contenida en $C_{ns}^{+}(N)$. En particular,
 $\overline{[E,\varphi]}$ es un punto $\bb{Q}$-racional y $j(E)$ es \'{\i}ntegro.
% Este punto es \'{u}nico.
\end{propoPuntosEnterosCuerpoQuad}
De un punto $\overline{[E,\varphi]}$ en $Y_{ns}^{+}(N)(\overline{\bb{Q}})$ tal que
$\rho_{N}(\absGal{\bb{Q}})\subset C_{ns}^{+}(N)$ y $E$ definida sobre $\bb{Q}$
(equivalentemente, $\bb{Q}$-racional), y tal que $j(E)\in\bb{Z}$, se dice que es
un punto \'{\i}ntegro de $Y_{ns}^{+}(N)$ [[Serre]].

\end{subsection}
