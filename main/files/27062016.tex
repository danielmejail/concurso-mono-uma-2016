\begin{subsection}{Nivel $7$}
Sea $F/\bb{Q}$ la extensi\'{o}n c\'{u}bica totalmente real generada por
$\zeta_{7}+\zeta_{7}^{-1}$, donde $\zeta_{7}=e^{2\pi i/7}$ es una ra\'{\i}z
s\'{e}ptima primitiva de la unidad en $\bb{C}$. Una unidad $\varepsilon$ de
$F$ se dir\'{a} \emph{excepcional}, si existe otra unidad $\varepsilon_{1}$
tal que $\varepsilon+\varepsilon_{1}=1$. Kenku [[kenku]] muestra la
existencia de una correspondencia entre los puntos de $X_{ns}^{+}(7)$,
\'{\i}ntegros sobre $\bb{Z}[1/7]$ y cierto subconjunto de las unidades
excepcionales de la extensi\'{o}n totalmente real $F/\bb{Q}$. Un punto es
\'{\i}ntegro sobre $\bb{Z}[1/7]$, si es $\bb{Q}$-racional y su
$j$-invariante
%(el de la curva el\'{\i}ptica asociada a cualquier
%representante de la clase del punto)
pertenece a $\bb{Z}[1/7]$.

La curva $X_{ns}^{+}(7)$ tiene $\phi(7)/2=3$ c\'{u}spides que, en el peor de
los casos, son $\bb{Q}(\zeta_{7})$-racionales%
%(Basta con considerar $X(7)$ o $X'(7)$)%
. Pero $\absGal{\bb{Q}}$ permuta las c\'{u}spides, con lo que quedan
definidas sobre $F=\bb{Q}(\zeta_{7}+\zeta_{7}^{-1})$.
Sea $\sigma(z)=1-\frac{1}{z}$ el automorfismo de $\bb{P}^{1}$ que permuta
c\'{\i}clicamente los puntos $\infty$, $1$ y $0$. Este automorfismo es de
orden tres. La curva $X_{ns}^{+}(7)$ est\'{a} definida sobre $\bb{Q}$,
tiene g\'{e}nero cero y cuenta con tres puntos $F$-racionales$ que
corresponden a las c\'{u}spides. Si $s\in\Gal(F,\bb{Q})$ es el automorfismo
de $F$ que act'\'{u}a como $\sigma$ sobre las c\'{u}spides, existe un
uniformizador ($K$-isomorfismo, en este caso)

\begin{align*}
 f & \,:\,X_{ns}^{+}(7)\rightarrow\bb{P}^{1}
\end{align*}
que hace corresponder las c\'{u}spides con $\infty$, $1$ y $0$, y tal que
$f^{s}=1-\frac{1}{f}$.

Sea $m\in\bb{Z}$ un entero, y sea $Q_{m}$ el polinomio

\begin{align*}
Q_{m}(X) & \,=\,X^{3}\,-\,mX^{2}\,+\,(m-3)X\,+\,1
\text{ .}
\end{align*}
Existen exactamente $24$ unidades excepcionales en $F$ que son ceros de
$Q_{m}$ para alg\'{u}n entero $m$%
%(ver [[kenku]] para las referencias y los detalles)%
. La relaci\'{o}n con los puntos $\bb{Q}$-racionales con $j$-invariante
en $\bb{Z}[1/7]$ est\'{a} dada por el siguiente lema. La demostraci\'{o}n
se puede hallar en [[kenku]].

\begin{lemaUnoKenku}
Sea $\overline{[E,\varphi]}\in X_{ns}^{+}(7)$ un punto $\bb{Q}$-racional
tal que $j(E)\in\bb{Z}[1/7]$. Si $\varepsilon:=f(\overline{[E,\varphi]})$,
entonces $\varepsilon$ es una unidad de $F$ y
$s(\varepsilon)=1-\varepsilon^{-1}$. Adem\'{a}s, resulta ser una
unidad excepcional que es cero de alg\'{u}n polinomio $Q_{m}$.

Rec\'{\i}procamente, si $\varepsilon\in F$ satisface
$s(\varepsilon)=1-\varepsilon^{-1}$, cumple con ser una unidad excepcional
y cero de $Q_{m}$ para alg\'{u}n entero $m$, entonces corresponde v\'{\i}a
$f:\,X_{ns}^{+}(7)\rightarrow\bb{P}^{1}$ a un punto $\bb{Q}$-racional con
$j$-invariante en $\bb{Z}[1/7]$.

\end{lemaUnoKenku}

Resta encontrar una parametrizaci\'{o}n para $X_{ns}^{+}(7)$.
Recordemos que existe un morfismo $\pi:\,X_{ns}^{+}(7)\rightarrow X(1)$
de grado $7\phi(7)/2=21$. Supongamos que existe una curva $Z$ y una
factorizaci\'{o}n de $\pi$ como en el diagrama siguiente

\begin{align*}
& X_{ns}^{+}(7)\xrightarrow{\Phi_{3}}
Z\xrightarrow{\Phi_{7}} X(1)\text{ ,}
\end{align*}
con $\Phi_{3}$ de grado $3$ y $\Phi_{7}$ de grado $7$. Supongamos,
adem\'{a}s, que contamos con un uniformizador $z:\,Z\rightarrow\bb{P}^{1}$
tal que los \'{u}nicos puntos de $X_{ns}^{+}(7)$ en los cuales $z$ toma el
valor $\infty$ son aquellos en los que $f$ toma los valores $\infty$, $1$
y $0$ (es decir, las c\'{u}spides), y supongamos que hay un \'{u}nico
punto, en $X_{ns}^{+}(7)$, arriba de $z=0$ (y que, por ende, ramifica con
orden $3$). La relaci\'{o}n entre los uniformizadores $z$ y $f$ es

\begin{align*}
z &\,=\,\frac{a(f-b)^{3}}{f(f-1)}\text{ ,}
\end{align*}
para ciertas constantes $a$ y $b$ en $\bb{Q}(\zeta_{7})$.

La curva $X_{ns}^{+}(7)$ se obtiene como cociente de $X'(7)$ a partir del
subgrupo de automorfismos determinado por el subgrupo

\begin{align*}
C_{ns}^{+}(7)'/\{\pm 1\} & \,\subset\,\rm{PSL}_{2}(\bb{Z}/7\bb{Z})
\,=\,\SL{2}(\bb{Z}/7\bb{Z})/\{\pm 1\}
\text{ ,}
\end{align*}
donde $C_{ns}^{+}(7)'$ es $C_{ns}^{+}(7)\cap\SL{2}(\bb{Z}/7\bb{Z})$.
Sea $S\subset\rm{PSL}_{2}(\bb{Z}/7\bb{Z})$ un subgrupo isomorfo al
grupo de permutaciones de curatro elementos, $\bb{S}_{4}$.
Conjugando, podemos asumir que $S$ contiene a $C_{ns}^{+}(7)'/\{\pm 1\}$.
Sea $\widetilde{S}$ su preimagen por la proyecci\'{o}n
$\GL{2}(\bb{Z}/7\bb{Z})\rightarrow\rm{PGL}_{2}(\bb{Z}/7\bb{Z})$.
La acci\'{o}n de $\GL{2}(\bb{Z}/7\bb{Z})$ sobre las ra\'{\i}ces de la unidad
estaba dada por el determinante (ver [[ARRIBA, REFERENCIA, UNA DEMO]]).
Como $S$ est\'{a} contenido en $\rm{PSL}_{2}(\bb{Z}/7\bb{Z})$, si
$\kappa\in\widetilde{S}$, su determinante es un cuadrado \textit{modulo}
$7$, es decir, $1$, $2$ o $4$. Por otra parte, si

\begin{align*}
w & \,=\,2\zeta_{7}^{4}\,+\,2\zeta_{7}^{2}\,+\,2\zeta_{7}\,+\,1\text{ ,}
\end{align*}
entonces $w^{2}=-7$. En particular, $w$ queda fijo por $\widetilde{S}$,
mientras que $\zeta_{7}$ no. De esto se deduce que
$\bb{Q}(j,\{f_{0}^{\pm\overline{v}}\})\^{S}\cap\bb{Q}(\zeta_{7})$
es igual a $\bb{Q}(\sqrt{-7})$, y que existe una curva correspondiente al
grupo $S$ que da una factorizaci\'{o}n de $X_{ns}^{+}(7)\rightarrow X(1)$
y que cuenta con las propiedades de la curva $Z$ de antes. [[kenku]]
[[ligozat]]

Sea $Z$ esta curva, y sea $z:\,Z\rightarrow\bb{P}^{1}$
un uniformizador tal que su relaci\'{o}n con $j$ sea

\begin{align*}
j & \,=\,z(z^{2}+7\lambda z+7\lambda -21)^{3}\text{ .}
\end{align*}
Usando el m\'{e}todo expuesto en [[ligozat, ch. 2]], se pueden hallar los
valores de $a$ y de $b$, y obtener la siguiente tabla:




\end{subsection}