

La proposici\'{o}n que ahora enunciamos, y el argumento que le sigue, se pueden
encontrar en [[Serre, appendix]].

\begin{propoEquivsNumClasUno}
Sea $p>3$ un primo congruente a $3$ \textit{modulo} $4$. Las siguientes propiedades
son equivalentes:

\begin{itemize}
\item[i)] $h(-p)=1$;
\item[ii)] $\left( \frac{l}{p} \right)=-1$ para todo primo $l<p/4$;
\item[ii)'] $\left( \frac{l}{p} \right)=-1$ para todo primo $l<\sqrt{p/3}$;
\item[iii)] (si $p>7$) $p\equiv 3\,(8)$ y $R-N=3$, si $R$ (respectivamente, $N$)
es el n\'{u}mero de residuos (no residuos) cuadr\'{a}ticos \textit{modulo} $p$ en
$[1,(p-1)/2]$;
\item[iv)] si $x\in[\![ 0,(p-7)/4 ]\!]$, $P_{p}(x)=x^{2}+x+(p+1)/4$ es primo;
\item[iv)'] si $x\in[0,1/2(\sqrt{p/3}-1)]$, $P_{p}(x)$ es primo.
\end{itemize}
\end{propoEquivsNumClasUno}

\begin{proof}
El anillo de enteros de $\bb{Q}(\sqrt{-p})$ es $\bb{Z}[w]$, donde
$w=\frac{1+\sqrt{-p}}{2}$; un primo $l$ es inerte en $\bb{Q}(\sqrt{-p})/\bb{Q}$,
si, y s\'{o}lo si $(l/p)=-1$ (ver [[Cox]]). Dicho de otra manera, si $h(-p)=1$,
la condici\'{o}n $(l/p)=-1$ implica que $l$ es una norma: existe
$\alpha\in\bb{Z}[w]$, $\alpha=x+yw$, tal que $\Nm(\alpha)=l$. Pero $\Nm(\alpha)$
es igual a $(x+\frac{y}{2})^{2}+(\frac{y}{2})^{2}p$. As\'{\i}, $l$ es primo,
$y\not =0$ y $\Nm(\alpha)\geq p/4$. Rec\'{\i}procamente, asumamos que $(ii)'$ se
cumple. Entonces, si $\frak{l}\subset\bb{Z}[w]$ es un ideal primo, y vale
$\Nm(\frak{l})<\sqrt{p/3}$, dado $l$ primo que divide a $\Nm(\frak{l})$, es
$l<\sqrt{p/3}$. En particular, $(l/p)=-1$, por hip\'{o}tesis, y $\frak{l}=(l)$ es
un ideal principal. Todo ideal de norma menor que $\sqrt{p/3}$ es, entonces,
principal. Pero toda clase en $\Cl(\bb{Z}[w])$ contiene un elemento con esta
propiedad.

El resto de la demostraci\'{o}n (excepto las equivalencias con (iii)) es,
tambi\'{e}n, elemental.
\end{proof}

Sea $p\equiv 3\,(\rm{mod}\,4)$ un primo, $3<p<75$, y sea $m=(p+1)/4$. En este caso,
por (iv)', $h(-p)=1$ es equivalente a que $m$ y $m+2$ sean primos. Esto \'{u}ltimo
es cierto para $m$ en el conjunto $\{3,5,11,17\}$. El primo $p$ es un de
$11$, $19$, $43$ o $67$.

Sea $\Lambda=\omega_{1}\bb{Z}\oplus\omega_{2}\bb{Z}$ un ret\'{\i}culo en $\bb{C}$,
generado por $\omega_{1}$ y $\omega_{2}$ tales que
$\omega_{1}/\omega_{2}\in\frak{h}$. Los puntos de $N$-torsi\'{o}n de
$\bb{C}/\Lambda$ conforman el grupo
\begin{math}
\langle(\omega_{1}/N)+\Lambda\rangle\times
\langle(\omega_{2}/N)+\Lambda\rangle
\end{math}
isomorfo a $(\bb{Z}/N\bb{Z})^{2}$. Sea $\mu_{N}$ el grupo de ra\'{\i}ces
$N$-\'{e}simas de la unidad en $\bb{C}$, $\mu_{N}=\langle e^{2\pi i/N}\rangle$.
Dados $P$ y $Q$ en $\bb{C}/\Lambda$ de $N$-torsi\'{o}n, existe
$\gamma\in\MM{2}(\bb{Z}/N\bb{Z})$ tal que

\begin{align*}
\begin{bmatrix}
P\\Q
\end{bmatrix} & \,=\,\gamma
\begin{bmatrix}
(\omega_{1}/N)+\Lambda\\(\omega_{2}/N)+\Lambda
\end{bmatrix}\text{ .}
\end{align*}
Definimos $e_{N}(P,Q):=e^{2\pi i\det(\gamma)/N}$, el \textit{pairing} de Weil en
$P$ y $Q$. Esto define, v\'{\i}a la correspondencia con curvas el\'{\i}pticas
definidas sobre $\bb{C}$, una aplicaci\'{o}n,
$E[N]\times E[N]\rightarrow\mu_{N}$, en pares de puntos de $N$-torsi\'{o}n de una
curva el\'{\i}ptica $E$.

Sean $E$ y $E'$ dos curvas complejas, y sean $(P,Q)$ y $(P',Q')$ pares de puntos de
orden $N$ en $E$ y en $E'$, respectivamente, tales que
$e_{N}(P,Q)=e_{N}(P',Q')=e^{2\pi i/N}=\zeta_{N}$. La dupla $E$, junto con $(P,Q)$
se dice relacionada con $E'$, con $(P',Q')$, si existe un isomorfismo
$E\xrightarrow{\sim}E'$ (definido sobre $\bb{C}$) tal que $P\mapsto P'$ y
$Q\mapsto Q'$. \'{E}sta es una relaci\'{o}n de equivalencia, y denotamos $S'(N)$
al conjunto que resulta de tomar las clses de equivalencia correspondientes en

\begin{align*}
& \left\lbrace (E,(P,Q))\,:\,E\text{ curva el\'{\i}ptica sobre }\bb{C},
e_{N}(P,Q)=\zeta_{N}\right\rbrace\text{ ,}
\end{align*}
y con $[E,(P,Q)]$ a la clase de $(E,(P,Q))$.

Sea, ahora, $\Gamma(N)$ el subgrupo principal de congruencia de nivel $N$, el
n\'{u}cleo del morfismo sobreyectivo
$\SL{2}(\bb{Z})\rightarrow\SL{2}(\bb{Z}/N\bb{Z})$ dado por reducir coordenadas
\textit{modulo} $N$. Sean $Y'(N)$ la superficie de Riemann
$\Gamma(N)\backs\frak{h}$, y $X'(N)=\Gamma(N)\backs\frak{h}^{*}$ su
compactificaci\'{o}n agregando las c\'{u}spides correspondientes a $\Gamma(N)$
[[Diamond, Shurman]]. El conjunto $S'(N)$ est\'{a} dado por

\begin{align*}
& \left\lbrace [\bb{C}/\Lambda_{\tau},\,
(\tau/N+\Lambda_{\tau},1/N+\Lambda_{\tau})]\,:\,\tau\in\frak{h} \right\rbrace
\text{ ,}
\end{align*}
y dos puntos $[\bb{C}/\Lambda_{\tau},\,
(\tau/N+\Lambda_{\tau},1/N+\Lambda_{\tau})]$ y $[\bb{C}/\Lambda_{\tau'},\,
(\tau'/N+\Lambda_{\tau'},1/N+\Lambda_{\tau'})]$ son iguales, si, y s\'{o}lo si
$\Gamma(N)\tau=\Gamma(N)\tau'$. La aplicaci\'{o}n $[\bb{C}/\Lambda_{\tau},\,
(\tau/N+\Lambda_{\tau},1/N+\Lambda_{\tau})]\mapsto\Gamma(N)\tau$ da una
biyecci\'{o}n entre $S'(N)$ y la curva $Y'(N)$. Es decir, $X'(N)$ parametriza
(clases de equivalencia de) curvas el\'{\i}pticas junto con un par de generadores
de la $N$-torsi\'{o}n y con un \textit{pairing} particular.

Las curvas en las que fijaremos nestra atenci\'{o}n ser\'{a}n ciertos cocientes
de las curvas $X'(N)$. Sea $E/\bb{C}$ una curva el\'{\i}ptica, y sea $E[N]$ su
subgrupo de $N$-torsi\'{o}n. Fijar un isomorfismo
$\varphi:\,E[N]\xrightarrow{\sim}(\bb{Z}/N\bb{Z})^{2}$ equivale a dar una base de
$E[N]$, es decir, dos puntos $P$ y $Q$ de $E$ que generen $E[N]$. En este sentido,
la informaci\'{o}n relativa a la torsi\'{o}n que brinda un par $(E,\varphi)$ es
menos espec\'{\i}fica que la contenida en los pares $(E,(P,Q))$. Dados dos pares
$(E,\varphi)$ y $(E',\varphi')$, los mismos son equivalentes, si existe un
isomorfismo $E\xrightarrow{\sim}E'$ tal que $\varphi\mapsto\varphi'$, es decir,
si $\{P,Q\}$ es la base de $E[N]$ y $\{P',Q'\}$ la de $E'[N]$, $P\mapsto P'$ y
$Q\mapsto Q'$. La relaci\'{o}n es la misma; simplemente estamos permiti\'{e}ndonos
ver otros puntos que antes, en $X'(N)$, no consideramos. \'{E}sta es una
relaci\'{o}n de equivalencia, y denotamos $S(N)$ al conjunto de clases de pares
$(E,\varphi)$ (denotadas $[E,\varphi]$) por dicha relaci\'{o}n.

Sea $E/\bb{C}$ una curva el\'{\i}ptica, y sea
$\varphi:\,E[N]\rightarrow(\bb{Z}/N\bb{Z})^{2}$ una estructura de nivel. Sea
$e_{N}:\,E[N]\times E[N]\rightarrow\mu_{N}$ una forma alternada, no degenerada,
el \textit{pairing} de Weil, sea $\zeta_{N}:=e^{2\pi i/N}$ y $\zeta(\varphi)$ la
ra\'{\i}z de la unidad

\begin{align*}
\zeta(\varphi) & \,:=\,e_{N}\left(
\varphi^{-1}\left(\begin{bmatrix}1\\0\end{bmatrix}\right),\,
\varphi^{-1}\left(\begin{bmatrix}0\\1\end{bmatrix}\right)\right)
\text{ .}
\end{align*}

El par $(E,\varphi)$ es equivalente a uno de la forma
$(\bb{C}/\Lambda,\,(\tau/N+\Lambda_{\tau},1/N+\Lambda))$, $\tau\in\frak{h}$, si,
y s\'{o}lo si $\zeta(\varphi)=\zeta_{N}$.

En general, la forma alternada $e_{N}$ determina una funci\'{o}n
$S(N)\rightarrow\mu_{N}$. Haciendo de \'{e}sta una funci\'{o}n continua, vemos
que se puede obtener una superficie de Riemann compacta $X(N)$ a partir de $S(N)$,
y, como $e_{N}(aP+bQ,cP+dQ)=e_{N}(P,Q)^{\det(\gamma)}$, para
\begin{math}\gamma=
\left[\begin{smallmatrix}a&b\\c&d\end{smallmatrix}\right]\in\GL{2}(\bb{Z}/N\bb{Z})
\end{math}, la curva $X(N)$ es disconexa.
M\'{a}s precisamente, es la uni\'{o}n disjunta de $\phi(N)$ componentes. Cada una
de estas componentes es una copia de $X'(N)$ y est\'{a}n en correspondencia con
las ra\'{\i}ces primitivas $N$-\'{e}simas de $1$: si $\zeta_{N}^{k}$ es una de
ellas, la componente correspondiente es la que contiene las clases $[E,\varphi]$
con $\zeta(\varphi)=\zeta_{N}^{k}$. [[Booher, sec. 7]] [[Deligne, Rapoport]].
Para ser m\'{a}s claros, el grupo $\GL{2}(\bb{Z}/N\bb{Z})$ act\'{u}a sobre los
puntos no cuspidales de la curva compactificada $X(N)$. Esta acci\'{o}n est\'{a}
determinada por $(E,\phi)\mapsto(E,\gamma\circ\varphi)$.

Sea $H$ un subgrupo de $\GL{2}(\bb{Z}/N\bb{Z})$ tal que
$\det(H)=(\bb{Z}/N\bb{Z})^{\times}$. A trav\'{e}s del determinante,
$\GL{2}(\bb{Z}/N\bb{Z})$ permuta las componentes de la curva $X(N)$. Como la
imagen de $H$ por $\det(\cdot)$ es todo $(\bb{Z}/N\bb{Z})^{\times}$, el cociente
$H\backs X(N)$ es conexo. En el caso en que $H$ es igual a todo el grupo general
lineal, obtenemos $X(1)=\SL{2}(\bb{Z})\backs\frak{h}^{*}$.

Dado un subgrupo arbitrario $H$ de $\GL{2}(\bb{Z}/N\bb{Z})$, llamamos $H'$ a la
intersecci\'{o}n $H\cap\SL{2}(\bb{Z}/N\bb{Z})$, y $\Gamma_{H}$ a la preimagen de
$H'$ por $\SL{2}(\bb{Z})\rightarrow\SL{2}(\bb{Z}/N\bb{Z})$. Notando que
$\Gamma_{H}$ es un subgrupo de congruencia (de nivel $N$), definimos

\begin{align*}
X_{H} & \,:=\,\Gamma_{H}\backs\frak{h}^{*}\text{ .}
\end{align*}
Esta curva se identifica con el cociente $H'\backs X'(N)$, y tambi\'{e}n con
$H\backs X(N)$, si $\det(H)=(\bb{Z}/N\bb{Z})^{\times}$. En otras palabras, $X_{H}$
es el cociente de $X'(N)$ por la acci\'{o}n del subgrupo de automorfismos
determinado por $H'$.
%En particular, $H'$ act\'{u}a sobre el conjunto de c\'{u}spides de $X'(N)$,
%de manera que las c\'{u}spides de $X_{H}$ se identifican con las \'{o}rbitas
%de  esta acci\'{o}n.

Necesitaremos saber en qu\'{e} casos estas curvas est\'{a}n definidas sobre
$\bb{Q}$.

\begin{propoDefinidasSobreQ}
Sea $H$ un subgrupo de $\GL{2}(\bb{Z}/N\bb{Z})$ tal que su imagen por el
determinante cubra $(\bb{Z}/N\bb{Z})^{\times}$. Entonces $X_{H}=H\backs X(N)$ es
una variedad algebraica conexa definida sobre $\bb{Q}$.
\end{propoDefinidasSobreQ}
El enunciado de esta proposici\'{o}n, como su demostraci\'{o}n se pueden encontrar
en [[Booher, thm. 43]].

\begin{proof}
Primero se definen ciertas funciones modulares para el grupo $\Gamma(N)$
[[Diamond, Shurman]]: sea $v=(c_{v},d_{v})$ un par donde $c_{v}$ y $d_{v}$ son
enteros y sus reducciones no son ambas divisibles por $N$, es decir,
$\overline{v}(\overline{c_{v}},\overline{d_{v}})\not =0$ en $(\bb{Z}/N\bb{Z})^{2}$.
Dado $(\bb{C}/\Lambda,(P,Q))$ con $e_{N}(P.Q)=\zeta_{N}$, definimos

\begin{align*}
F_{0}^{\overline{v}}(\bb{C}/\Lambda,(P,Q)) & \,=\,
\frac{g_{2}(\Lambda)}{g_{3}(\Lambda)\wp_{\Lambda}(c_{v}P+d_{v}Q)\text{ ,}
\end{align*}
donde $g_{2}$ y $g_{3}$ son las funciones que se obtienen de las series de
Eisenstein $G_{4}$ y $G_{6}$, y $\wp_{\Lambda}$ es la funci\'{o}n de
Weierstra{\ss} del ret\'{\i}culo $\Lambda$. De manera equivalente, si
$\tau\in\frak{h}$, podemos definir

\begin{align*}
f_{0}^{\overline{v}}(\tau) & \,=\,
\frac{g_{2}(\tau)}{g_{3}(\tau)}\wp_{\tau}(\frac{c_{v}\tau+d_{v}}{N})
\text{ ,}
\end{align*}
y
\begin{math}
f_{0}^{\overline{v}}(\tau)=
F_{0}^{\overline{v}}(\bb{C}/\Lambda_{\tau},(\tau/N,1/N))
\end{math}.
Las funciones $F_{0}^{\overline{v}}$ no dependen de la clase de
$(\bb{C}/\Lambda,(P,Q))$, y las funciones $f_{0}^{\overline{v}}$ no dependen de
la \'{o}rbita $\Gamma(N)\tau\subset\frak{h}$. Se veifica que \'{e}stas son
funciones invariantes por $\Gamma(N)$ y meromorfas, tanto en $\frak{h}$, como en
las c\'{u}spides de $\Gamma(N)$. Es decir, son funciones meromorfas en $X'(N)$.

Sea $\bb{C}(X'(N))$ el cuerpo de funciones meromorfas en $X'(N)$, y sea

\begin{align*}
\theta & \,:\,\SL{2}(\bb{Z})\rightarrow\rm{Aut}(\bb{C}(X'(N)))\,|\\
 & \gamma\mapsto(
\theta(\gamma):\,f\mapsto f^{\theta(\gamma)}=f\circ\gamma)\text{ .}
\end{align*}
Esta aplicaci\'{o}n es un morfismo de grupos, y $\ker(\theta)$ contiene a
$\widetilde{\Gamma}(N):=\{\pm\}\Gamma(N)$. Notemos, por otra parte, que dos
funciones $f_{0}^{\overline{u}}$ y $f_{0}^{\overline{v}}$ son iguales, si, y
s\'{o}lo si $\overline{u}=\pm\overline{v}$, pues $\wp_{\tau}(z)=\wp_{\tau}(z')$,
si, y s\'{o}lo si $z+\Lambda_{\tau}=\pm z'+\Lambda_{\tau}$. Adem\'{a}s,
\begin{math}
(f_{0}^{\overline{v}})^{\theta(\gamma)}=
f_{0}^{\overline{v}}\circ\gamma=f_{0}^{\overline{v\gamma}}$
\end{math},
con lo que, como $f_{0}^{\overline{v}}\in\bb{C}(X'(N))$, el n\'{u}cleo
$\ker(\theta)$ est\'{a} contenido en $\widetilde{\Gamma}(N)$.

Ahora, $\theta(\SL{2}(\bb{Z}))$ es un subgrupo del grupo de automorfismos del
cuerpo $\bb{C}(X'(N))$, y su cuerpo fijo es $\bb{C}(X(1))=\bb{C}(j)$. En
definitiva, la extensi\'{o}n $\bb{C}(X'(N))/\bb{C}(X(1))$ es galoisiana con
grupo de Galois

\begin{align*}
\theta(\SL{2}(\bb{Z}))& \,\simeq\,\SL{2}(\bb{Z})/\widetilde{\Gamma}(N)
\,\simeq\,\SL{2}(\bb{Z}/N\bb{Z})/\{\pm 1\}\text{ .}
\end{align*}
El mismo argumento que los \'{u}nicos elementos de $\SL{2}(\bb{Z})$ que act\'{u}an
trivialmente sobre
\begin{math}
\bb{C}(j,\{f_{0}^{\pm\overline{v}}\,:\,
\pm\overline{v}\in((\bb{Z}/N\bb{Z})^{2}\smallsetminus\{(0,0)\})/\{\pm 1\}\})
\end{math}
son los pertenecientes a $\widetilde{\Gamma}(N)$. Lo mismo es cierto para el
cuerpo $\bb{C}(j,f_{0}^{\pm\overline{(0,1)}},f_{0}^{\pm\overline{(1,0)}})$, que
coincide, entonces, con $\bb{C}(X'(N))$.

El siguiente paso consiste en demostrar que la extensi\'{o}n
$\bb{Q}(\mu_{N},j,\{f_{0}^{\pm\overline{v}}\})/\bb{Q}(j)$ tambi\'{e}n es Galois y
que su grupo se identifica con $\GL{2}(\bb{Z}/N\bb{Z})/\{\pm 1\}$.

Finalmente, si $H\subset\GL{2}(\bb{Z}/N\bb{Z})$ es un subgrupo que satisface
$\det(H)=(\bb{Z}/N\bb{Z})^{\times}$, definimos los cuerpos

\begin{align*}
K_{1} & \,=\,\bb{Q}(\mu_{N},j,\{f_{0}^{\pm\overline{v}}\})^{\Gamma_{H}}\\
K_{2} & \,=\,\bb{C}(j,\{f_{0}^{\pm\overline{v}}\})^{H'/\{\pm 1\}}\text{ .}
\end{align*}
Un poco de Teor\'{\i}a de Galois muestra que
$K_{1}=\bb{Q}(j,\{f_{0}^{\pm\overline{v}}\})^{H'/\{\pm 1\}}$ y que
$K_{1}\cap\overline{\bb{Q}}=\bb{Q}$, correspondi\'{e}ndose con una curva
poryectiva, no singular, y esta curva est\'{a} definida sobre $\bb{Q}$.
Con respecto a $K_{2}$, este cuerpo tiene que ser el cuerpo de funciones del
cociente $H'\backs X'(N)$, ya que, para $f\in\bb{C}(X'(N))$,
$f^{\theta(\gamma)}=f\circ\gamma$. Ahora bien, el cuerpo de funciones de la
curva definida sobre los n\'{u}meros complejos que se obtiene a partir de las
ecuaciones que definen la curva correspondiente al cuerpo funcional $K_{1}$,
coincide con $K_{2}$. Es decir, $H\backs X(N)$ es isomorfa sobre $\bb{C}$ a una
curva definida sobre $\bb{Q}$. Usaremos la misma notaci\'{o}n para referirnos
tanto a $H\backs X(N)$ como a su modelo sobre $\bb{Q}$.

\end{proof}