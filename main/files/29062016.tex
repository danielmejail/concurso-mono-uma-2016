\begin{subsection}{Nivel $5$}
Hemos visto que la curva $X_{ns}^{+}(3)$ admite un uniformizador $t$
[[REVISAR EL S\'{\I}MBOLO]] que satisface $j=t^{3}$. En particular,
se deduce que toda curva el\'{\i}ptica $E$ definida sobre
$\overline{\bb{Q}}$ da lugar a un punto $\bb{Q}$-racional en
$X_{ns}^{+}(3)$, si, y s\'{o}lo si $j(E)$ es un cubo en $\bb{Q}$.
Si $d$ es igual a $-7$, $-8$, $-28$, $-43$, $-67$ o a $-163$, entonces, en
el orden cuad\'{a}tico imaginario $I$ de discriminante $d$, el primo $3$ es
no ramificado y $5$ es inerte. Se obtienen, as\'{\i}, al menos seis
puntos $\bb{Q}$-racionales en $X_{ns}^{+}(3)$ y en $X_{ns}^{+}(5)$ cuyos
$j$-invariantes son cubos en $\bb{Z}$. Si $d$ es menor a $-163$, tanto
$3$, como $5$, es inerte en el orden de discriminante $d$. Por lo tanto,
a partir de un orden cuadr\'{a}tico imaginario $I$ de discriminante $d<-163$
con n\'{u}mero de clases igual a $1$, se obtiene un punto $\bb{Q}$-racional
en $X_{ns}^{+}(5)$ con $j$ un cubo entero. Pero, por medio de una
parametrizaci\'{o}n de dicha curva, se obtiene la lista completa de los
posibles puntos con los que se debe corresponder. Como los seis \'{o}rdenes
mencionados y aquel de discriminante $-3$ dan cuenta de todos estos puntos,
$I$ debe ser uno de ellos. En [[chen]], el autor obtiene una
parametrizaci\'{o}n de $X_{ns}^{+}(5)$, permiti\'{e}ndole dar una
soluci\'{o}n al problema del n\'{u}mero de clases igual a $1$, como
tambi\'{e}n interpretar la soluci\'{o}n por Siegel del problema
[[REFERENCIA]] en t\'{e}rminos de $X_{ns}^{+}(5)$. A continuaci\'{o}n
resumimos el proceso que conduce a dicha parametrizaci\'{o}n.





\end{subsection}