

%\begin{subsection}{Nivel $5$}
Hemos visto que la curva $X_{ns}^{+}(3)$ admite un uniformizador $s$
que satisface $j=s^{3}$. En particular,
se deduce que toda curva el\'{\i}ptica $E$ definida sobre
$\overline{\bb{Q}}$ da lugar a un punto $\bb{Q}$-racional en
$X_{ns}^{+}(3)$, si, y s\'{o}lo si $j(E)$ es un cubo en $\bb{Q}$.
Si $d$ es igual a $-7$, $-8$, $-28$, $-43$, $-67$ o a $-163$, entonces, en
el orden cuad\'{a}tico imaginario $I$ de discriminante $d$, el primo $3$ es
no ramificado y $5$ es inerte. Se obtienen, as\'{\i}, al menos seis
puntos $\bb{Q}$-racionales en $X_{ns}^{+}(3)$ y en $X_{ns}^{+}(5)$ cuyos
$j$-invariantes son cubos en $\bb{Z}$. Si $d$ es menor a $-163$, tanto
$3$, como $5$, es inerte en el orden de discriminante $d$. Por lo tanto,
a partir de un orden cuadr\'{a}tico imaginario $I$ de discriminante $d<-163$
con n\'{u}mero de clases igual a $1$, se obtiene un punto $\bb{Q}$-racional
en $X_{ns}^{+}(5)$ con $j$ un cubo entero. Pero, por medio de una
parametrizaci\'{o}n de dicha curva, se obtiene la lista completa de los
posibles puntos con los que se debe corresponder. Como los seis \'{o}rdenes
mencionados y aquel de discriminante $-3$ dan cuenta de todos estos puntos,
$I$ debe ser uno de ellos. En \cite{chenLevelFive}, el autor obtiene una
parametrizaci\'{o}n de $X_{ns}^{+}(5)$, permiti\'{e}ndole dar una
soluci\'{o}n al problema del n\'{u}mero de clases igual a $1$, como
tambi\'{e}n interpretar la soluci\'{o}n por Siegel del problema
(en un trabajo titulado \textit{Zum Beweise des Starkschen Satzes})
en t\'{e}rminos de $X_{ns}^{+}(5)$. A continuaci\'{o}n
resumimos el proceso que conduce a dicha parametrizaci\'{o}n.

Como en los dos ejemplos anteriores, consideremos el cubrimiento de
$X_{ns}^{+}(5)$ sobre $X(1)$. Recordemos que los puntos el\'{\i}pticos de
orden dos o tres en $X_{ns}^{+}(5)$ son puntos que se proyectan, respectivamente,
sobre $i$ o $\rho$ en $X(1)$ y cuyo \'{\i}ndice de ramificaci\'{o}n es igual a $1$.
Por otra parte, las f\'{o}rmulas al final de \ref{subsec:ram} indican que, para
esta curva, hay un \'{u}nico punto el\'{\i}ptico de orden tres (y, por lo tanto,
otros tres puntos arriba de $\rho$ cuyo \'{\i}ndice de ramificaci\'{o}n es $3$),
y dos de orden dos (y cuatro puntos arriba de $i$ cuyo \'{\i}ndice es $2$). Con
respecto a las c\'{u}spides, las mismas f\'{o}rmulas nos muestran que
$X_{ns}^{+}(5)$ cuenta con dos c\'{u}spides, y la ramificaci\'{o}n es, en ambas,
de \'{\i}ndice $5$.
%
Como el grupo de Galois act\'{u}a por permutaciones sobre el conjunto de
c\'{u}spides, y dado que las c\'{u}sides de $X_{ns}^{+}(5)$ est\'{a}n
definidas sobre $\bb{Q}(\zeta_{5})$, donde $\zeta_{5}:=e^{2\pi i/5}$,
si $\sigma\in\Gal(\bb{Q}(\zeta_{5})/\bb{Q})$, entonces $\sigma^{2}$
act\'{u}a trivialmente sobre dicho conjunto. En particular, las c\'{u}spides
est\'{a}n definidas sobre la subextensi\'{o}n cuadr\'{a}tica
$\bb{Q}(\sqrt{5})/\bb{Q}$.

Sea $\eta:\,X_{ns}^{+}(5)\rightarrow\bb{P}^{1}$ un uniformizador definido
sobre $\bb{Q}$. Componiendo $\eta$ con un $\bb{Q}$-automorfismo de
$\bb{P}^{1}$, podemos suponer que las c\'{u}spides de $X_{ns}^{+}(5)$ son
las ra\'{\i}ces de $X^{2}-5$. Por otra parte, como la curva tiene un
\'{u}nico punto el\'{\i}ptico de orden tres, el mismo tiene que ser un punto
$\bb{Q}$-racional de $X_{ns}^{+}(5)$, tiene que quedar fijo por la
acci\'{o}n del grupo de Galois. Podemos asumir tambi\'{e}n que
$\eta$ en este punto toma el valor $0$.
Entonces, la parametrizaci\'{o}n de la curva va a estar dada por una
relaci\'{o}n de la forma:

\begin{align*}
j & \,=\,\lambda
\frac{\eta(\eta-A)^{3}(\eta^{2}-B\eta+C)^{3}}{(\eta^{2}-5)^{5}}\text{ .}
\end{align*}
Las constantes $A$, $B$ y $C$ est\'{a}n determinadas por el valor de $\eta$
en los puntos de $X_{ns}^{+}(5)$ arriba de $\rho$ cuyo \'{\i}ndice de
ramificaci\'{o}n es $3$. Usando un cubrimiento intermedio de manera
similar a lo explicado en \ref{subsec:nivelSiete} es posible calcular los
valores de estas constantes. A trav\'{e}s de la transformaci\'{o}n
$z\mapsto 2z/(z+5)$, se llega a la relaci\'{o}n siguiente \cite{chenLevelFive}

\begin{align*}
j & \,=\,5^{3}
\frac{\eta(2\eta+1)^{3}(2\eta^{2}+7\eta+8)^{3}}{(\eta^{2}+\eta-1)^{5}}
\text{ .}
\end{align*}
%
Una vez hallada esta parametrizaci\'{o}n de $X_{ns}^{+}(5)$, si llamamos
$t$ al uniformizador de $X_{ns}^{+}(3)$ que es la ra\'{\i}z c\'{u}bica
de $j$, sabemos que, dado un orden cuadr\'{a}tico imaginario $I$
cuyo n\'{u}mero de clases es $1$, y elegida una curva el\'{\i}ptica $E$
(definida sobre $\bb{Q}$) con multiplicaci\'{o}n compleja por $I$,
si el primo $3$ no ramifica en $I$ y $5$ es inerte, entonces, por medio
de $t$ y $\eta$, obtenemos enteros $m$, $x$ e $y$ tales que
$j(E)=m^{3}$, $\eta=x/y$ y $(x,y)=1$. Adem\'{a}s, por la relaci\'{o}n
con $j$, sabemos que la terna $(x,y,m)$ tiene que ser una soluci\'{o}n
en $\bb{Z}$ a la ecuaci\'{o}n

\begin{align*}
m^{3} & \,=\,u(x,y)\,:=\,5^{3}
\frac{x(2x+y)^{3}(2x^{2}+7xy+8y^{2})^{3}}{(x^{2}+xy-y^{2})^{5}}
\text{ .}
\end{align*}
Las soluciones con $m=0$ son $(0,1,0)$, $(0,-1,0)$, $(-1,2,0)$ y
$(1,-2,0)$. 

En general, si $(x,y,m)$ es soluci\'{o}n a la ecuaci\'{o}n, tambi\'{e}n
lo es $(-x,-y,m)$. Sea $m\not =0$ y $(x,y,m)$ una
soluci\'{o}n. Si $l$ es un primo racional que divide a $x^{2}+xy-y^{2}$,
entonces $(x,y)=1$ implica que $l$ no puede dividir ni a $x$, ni a $y$.
Dado que $u(x,y)$ es un entero y $l$ divide al denominador en la
expresi\'{o}n para $u(x,y)$, el primo $l$ tiene que ser $5$. Esto se
deduce de que el sistema de ecuaciones \textit{modulo} $l$ (con $l$
primo)

\begin{align*}
z^{2}\,+\,z\,-\,1 & \,\equiv\, 0\,(\rm{mod}\,l)\\
2z\,+\,1 & \,\equiv\, 0\,(\rm{mod}\,l)
\end{align*}
tiene soluciones en enteras s\'{o}lo si $l=5$, y de que lo mismo es
cierto para

\begin{align*}
z^{2}\,+\,z\,-\,1 & \,\equiv\, 0\,(\rm{mod}\,l)\\
2z^{2}\,+\,7z\,+\,8 & \,\equiv\, 0\,(\rm{mod}\,l)\text{ .}
\end{align*}
Pero la ecuaci\'{o}n $z^{2}+z-1\equiv 0\,(5^{2})$ no admite soluciones
en $\bb{Z}$. En definitiva, si $(x,y,m)$ es una soluci\'{o}n
para $m^{3}=u(x,y)$ con $x$, $y$ y $m$ en $\bb{Z}$ y $(x,y)=1$,
la expresi\'{o}n $x^{2}+xy-y^{2}$ es igual a $\pm 5$ o $\pm 1$. En el
primer caso, $u(x,y)$ no es un cubo en $\bb{Z}$. Por esta raz\'{o}n,
$x^{2}+xy-y^{2}=\pm 1$.

Si denotamos con $\epsilon$ al elemento $(-1+\sqrt{5})/2$ de
$F:=\bb{Q}(\sqrt{5})$, y $\cal{O}_{F}$ al anillo de enteros de este
cuerpo, la condici\'{o}n sobre $x^{2}+xy-y^{2}$ equivale a que
la norma de $x+y\epsilon\in\cal{O}_{F}$ sea igual a $1$ o a $-1$, a que
$x+y\epsilon$ sea una unidad en este anillo.
Pero las unidades de $\cal{O}_{F}=\bb{Z}[\epsilon]$ son de la forma

\begin{align*}
\pm\epsilon^{n} & \,=\,\pm(x_{n}\,+\,y_{n}\epsilon)\text{ ,}
\end{align*}
donde $x_{n}:=(-1)^{n+1}F_{n}$ e $y_{n}:=(-1)^{n}F_{n+1}$
($F_{n}$ es el $n$-\'{e}simo n\'{u}mero de Fibonacci). En particular,
Dada la soluci\'{o}n $(x,y,m)$, tenemos $x=\pm x_{n}$ para alg\'{u}n
$n$. Pero, de la expresi\'{o}n para $u(x,y)$, deducimos que $x$ es un
cubo en $\bb{Z}$, y que, entonces $x_{n}$ tambi\'{e}n lo es.

Los \'{u}nicos n\'{u}meros de Fibonacci que son cubos son $F_{1}=1$, $F_{2}=1$
y $F_{6}=8$ (ver \cite{chenLevelFive}). Si definimos

\begin{align*}
 & L_{1}\,:=\,1\text{ , }\,L_{2}\,:=\,3\text{ , }
 \,L_{n+1}\,=\,L_{n}\,+\,L_{n-1}\text{ ,}\\
 & a\,=\,\frac{1\,+\,\sqrt{5}}{2}\text{ , }
 \,b\,=\,\frac{1\,-\,\sqrt{5}}{2}\text{ ,}
\end{align*}
inductivamente,

\begin{align*}
 & F_{n}\,=\,\frac{a^{n}\,-\,b^{n}}{\sqrt{5}}\text{ , }
 \,L_{n}\,=\,a^{n}\,+\,b^{n}\text{ y}\\
 & L_{n}^{2}\,-\,5F_{n}^{2}\,=\,4(-1)^{n}\text{ .}
\end{align*}
Sobre la curva de nivel $24$ el problema de hallar los puntos enteros se
reduc\'{\i}a a hallar soluciones a ciertas ecuaciones diof\'{a}nticas. Sobre la
curva de nivel $5$, la ecuaci\'{o}n diof\'{a}ntica es
%el problema pasa a ser la resoluci\'{o}n de

\begin{align*}
 Y^{2} & \,=\,5Z^{6}\,\pm\,4\text{ .}
\end{align*}

En resumen, las posibles soluciones $(x,y,m)$ a $m^{3}=u(x,y)$ en $\bb{Z}$
con $(x,y)=1$, cumplen con $x=0$, $\pm 1$ o $\pm 8$. Requiriendo que $y$ en
$(x,y,m)$ sea positivo, las posibles soluciones son las ternas en la siguiente
tabla.

\begin{tabular}{r|l|l}
 $d$ & $j=t^{3}$ & $(x,y,m)$\\
\hline
 $-3$ & $0$ & $(0,1,0)$\\
 $-3$ & $0$ & $(-1,2,0)$\\
 $-7$ & $-3^{3}\cdot 5^{3}$ & $(-1,1,-15)$\\
 $-8$ & $2^{6}\cdot 5^{3}$%
	& $(1,0,20)$\\
 $-28$ & $3^{3}\cdot 5^{3}\cdot 7^{3}$%
	& $(1,1,255)$\\
 $-43$ & $-2^{15}\cdot 3^{3}$%
	& $(1,2,-96)$\\
 $-67$ & $-2^{15}\cdot 3^{3}\cdot 5^{3}\cdot 11^{3}$%
	& $(-8,5,-5280)$\\
 $-163$ & $-2^{18}\cdot 3^{3}\cdot 5^{3}\cdot 23^{3}\cdot 29^{3}$%
	& $(8,13,-640320)$
\end{tabular}



%\end{subsection}