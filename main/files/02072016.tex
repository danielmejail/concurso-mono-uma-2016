Como el grupo de Galois act\'{u}a por permutaciones sobre el conjunto de
c\'{u}spides, y dado que las c\'{u}sides de $X_{ns}^{+}(5)$ est\'{a}n
definidas sobre $\bb{Q}(\zeta_{5})$, donde $\zeta_{5}:=e^{2\pi i/5}$,
si $\sigma\in\Gal(\bb{Q}(\zeta_{5})/\bb{Q})$, entonces $\sigma^{2}$
act\'{u}a trivialmente sobre dicho conjunto. En particular, las c\'{u}spides
est\'{a}n definidas sobre la subextensi\'{o}n cuadr\'{a}tica
$\bb{Q}(\sqrt{5})/\bb{Q}$.

Sea $\eta:\,X_{ns}^{+}(5)\rightarrow\bb{P}^{1}$ un uniformizador definido
sobre $\bb{Q}$. Componiendo $\eta$ con un $\bb{Q}$-automorfismo de
$\bb{P}^{1}$, podemos suponer que las c\'{u}spides de $X_{ns}^{+}(5)$ son
las ra\'{\i}ces de $X^{2}-5$. Por otra parte, como la curva tiene un
solo punto el\'{\i}ptico de orden tres, el mismo tiene que ser un punto
$\bb{Q}$-racional de $X_{ns}^{+}(5)$, tiene que quedar fijo por la
acci\'{o}n del grupo de Galois. Podemos asumir tambi\'{e}n que
$\eta$ en este punto toma el valor $0$.
Entonces, la parametrizaci\'{o}n de la curva va a estar dada por una
relaci\'{o}n de la forma:

\begin{align*}
j & \,=\,\lambda
\frac{\eta(\eta-A)^{3}(\eta^{2}-B\eta+C)^{3}}{(\eta^{2}-5)^{5}}\text{ .}
\end{align*}
Las constantes $A$, $B$ y $C$ est\'{a}n determinadas por el valor de $\eta$
en los puntos de $X_{ns}^{+}(5)$ arriba de $\rho$ cuyo \'{\i}ndice de
ramificaci\'{o}n es $3$. Usando un cubrimiento intermedio de manera
similar a lo explicado en [[REFERENCIA nivel $7$]] es posible calcular los
valores de estas constantes. A trav\'{e}s de la transformaci\'{o}n
$z\mapsto 2z/(z+5)$, se llega a la relaci\'{o}n siguiente [[chen]]

\begin{align*}
j & \,=\,5^{3}
\frac{\eta(2\eta+1)^{3}(2\eta^{2}+7\eta+8)^{3}}{(\eta^{2}+\eta-1)^{5}}
\text{ .}
\end{align*}


