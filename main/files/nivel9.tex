En la soluci\'{o}n al problema del n\'{u}mero de clases igual a $1$ que damos
a continuaci\'{o}n, y que se encuentra en \cite{baranLevelNine},
el m\'{e}todo es similar al que se puede encontrar en \cite{chenLevelFive}
y que est\'{a} detr\'{a}s de la descripci\'{o}n en la secci\'{o}n anterior.

Sea $\rm{r}:\,\SL{2}(\bb{Z}/9\bb{Z})\rightarrow\SL{2}(\bb{Z}/3\bb{Z})$ el  morfismo
dado por reducci\'{o}n de coordenadas \textit{modulo} $3$, y sea $N$ el
n\'{u}cleo de este morfismo. Si llamamos $H$ al subgrupo de
$\SL{2}(\bb{Z}/9\bb{Z})$ generado por las matrices
\begin{math}\left[\begin{smallmatrix}0&-1\\1&0\end{smallmatrix}\right]\end{math}
y
\begin{math}\left[\begin{smallmatrix}-1&-4\\-4&1\end{smallmatrix}\right]\end{math},
y $N'$ al generado por
\begin{math}\left[\begin{smallmatrix}1&-3\\3&1\end{smallmatrix}\right]\end{math},
entonces $H$ normaliza a $N'$
y $C_{ns}^{+}(9)'=C_{ns}^{+}(9)\cap\SL{2}(\bb{Z}/9\bb{Z})$ es el producto
semidirecto de $N'$ por $H$.
Si $N''$ es el subgrupo generado por
\begin{math}\left[\begin{smallmatrix}1&-3\\3&1\end{smallmatrix}\right]\end{math}
y por
\begin{math}\left[\begin{smallmatrix}-2&3\\3&4\end{smallmatrix}\right]\end{math},
entonces $H$ normaliza a $N''$ tambi\'{e}n, y

\begin{align*}
 & C_{ns}^{+}(9)'\,\simeq\,N'\,\rtimes\,H\text{ y }
 \,r^{-1}(C_{ns}^{+}(3))\,\simeq\,N\,\rtimes\,H\text{ .}
\end{align*}
Adem\'{a}s, valen las inclusiones
$C_{ns}^{+}(9)\subset N''\rtimes H\subset r^{-1}(C_{ns}^{+}(3))$, ambas de
\'{\i}ndice $3$.
Llamaremos $B$ al subgrupo de $r^{-1}(C_{ns}^{+}(3))$ isomorfo a $N''\rtimes H$, y
$\Gamma_{B}$ al subgrupo de $\SL{2}(\bb{Z})$ conformado por las matrices
congruentes a alg\'{u}n elemento de $B$ al reducir coordenadas \textit{modulo} $9$.

En cuanto a las inclusiones
$\Gamma_{ns}^{+}(9)\subset\Gamma_{B}\subset\Gamma_{ns}^{+}(3)$ los \'{\i}ndices
son, tambi\'{e}n, $3$. En particular, los cubrimientos $\Phi_{1}$ y
$\Phi_{2}$ en

\begin{align*} 
 & X_{ns}^{+}(9)\xrightarrow{\Phi_{1}}
 X_{B}\xrightarrow{\Phi_{2}} X_{ns}^{+}(3)\xrightarrow{\Phi_{3}}
 X(1)\text{ ,}
\end{align*}
son, ambos, de grado $3$. El grado del cubrimiento $\Phi_{3}$ tambi\'{e}n es $3$.
A continuaci\'{o}n hacemos uso de la informaci\'{o}n de ramificaci\'{o}n de
estos cubrimientos, refiriendo a \cite{baranLevelNine} con respecto a los detalles
de c\'{o}mo obtenerla.

Sabemos que las curvas $X_{ns}^{+}(3)$ y $X_{ns}^{+}(9)$ est\'{a}n definidas
sobre $\bb{Q}$ y que los morfismos $\Phi_{2}\circ\Phi_{1}$ y
$\Phi_{3}$ son $\bb{Q}$-morfismos. Si consideramos el subgrupo
$(\bb{Z}/9\bb{Z})^{\times}\cdot B$ de $\GL{2}(\bb{Z}/9\bb{Z})$, y argumentando
como con el grupo $S$ al tratar la curva de nivel $7$, vemos que $X_{B}$ est\'{a}
definida sobre $\bb{Q}(\sqrt{-3})$.

En cuanto a $X_{ns}^{+}(3)$ elgimos el uniformizador $t$ tal que $j=t^{3}$.
En esta curva hay un \'{u}nico punto $\rho'$ arriba de $\rho\in X(1)$ y
un \'{u}nico punto $\infty'$ arriba de $\infty$. Sobre el punto $i\in X(1)$, en
cambio, se proyectan tres, $i_{1}$, $i_{2}$ e $i_{3}$, eligiendo los
sub\'{\i}ndices de manera que $t(i_{1})=12$, $t(i_{2})=12\zeta_{3}^{-1}$
y $t(i_{3})=12\zeta_{3}$, donde $\zeta_{3}=e^{2\pi i/3}$.

En $X_{B}$ hay una \'{u}nica c\'{u}spide, que denotamos $\infty''$.
Hay dos puntos $i_{3,1}$ e $i_{3,2}$ en $X_{B}$ tales que
$\Phi_{2}(i_{3,k})=i_{3}$. El \'{\i}ndice de ramificaci\'{o}n de $\Phi_{2}$
en $i_{3,1}$ es $1$ y en $i_{3,2}$ es igual a $2$. \'{E}stos son los \'{u}nicos
puntos arriba de $i_{3}$; la acci\'{o}n
de $\Gal(\overline{\bb{Q}}/\bb{Q}(\sqrt{-3}))$ los permuta. Pero
uno es ramificado y el otro no, entonces los tiene que dejar fijos, es decir,
son $\bb{Q}(\sqrt{-3})$-racionales.

\begin{propoBaranNineXB}\label{thm:propoBaranNineXB}
 Existe un uniformizador $w:\,X_{B}\rightarrow\bb{P}^{1}$, definido
 sobre $\bb{Q}(\sqrt{-3})$ y tal que $w(\infty'')=\infty$, $w(i_{3,1})=2\sqrt{-3}$
 y $w(i_{3,2})=-\sqrt{-3}$. Adem\'{a}s, la relaci\'{o}n entre $w$ y $t$ es

 \begin{align*}
  t & \,=\,\zeta_{3}^{-1}(w^{3}\,+\,9w\,-\,6)\text{ .}
 \end{align*}
\end{propoBaranNineXB}

\begin{proof}[Demostraci\'{o}n]
 Supongamos que $\eta:\,X_{B}\rightarrow\bb{P}^{1}$ es el uniformizador determinado
 por $\eta(\infty'')=\infty$, $\eta(i_{3,1})=1$ y $\eta(i_{3,2})=0$. Sabemos que,
 entonces, la relaci\'{o}n con $t$ tiene que ser de la forma
 (ver la secci\'{o}n \ref{subsec:params})

 \begin{align*}
  t &\,=\,\lambda\prod_{k=1}^{3}\,(\eta\,-\,\eta(\rho_{k}))
  \,=\,\lambda(\eta^{3}\,+\,A\eta^{2}\,+\,B\eta\,+\,C)
 \end{align*}
ya que los puntos $\rho_{k}$, arriba de $\rho'$, son no ramificados.
Las constantes $\lambda\not = 0$, $A$, $B$ y $C$ pertenecen al
cuerpo $\bb{Q}(\sqrt{-3})$. Tambi\'{e}n podemos expresar $t$ en t\'{e}rminos de
los valores de $\eta$ en otros puntos: por ejemplo, teniendo en cuenta la
ramificaci\'{o}n de $\Phi_{2}$,

\begin{align*}
 t & \,=\,\lambda(\eta\,-\,\eta(i_{3,2}))^{2}(\eta\,-\,\eta(i_{3,1}))
 \,+\,t(i_{3})\\
 & \,=\,\lambda(\eta\,-\,\eta(i_{1,2}))^{2}(\eta\,-\,\eta(i_{1,1}))
 \,+\,t(i_{1})\text{ .}
\end{align*}
El valor de $t$ en $i_{3}$ es $12\zeta_{3}$ y, en $i_{1}$, $12$. Evaluando en
$i_{3,2}$, se deduce que el valor de $\lambda C$ es $12\zeta_{3}$, y, como el
\'{\i}ndice de ramificaci\'{o}n de $\Phi_{2}$ en $i_{3,2}$ es igual a $2$, que
$t-t(i_{3})$ tiene un cero doble en $i_{3,2}$. En particular, la constante $B$
tiene que ser igual a $0$. Por otra parte, evaluando en $i_{3,1}$, se ve que el
valor de $A$ es $-1$. De la misma manera, al evaluar en $i_{1,1}$ e $i_{1,2}$,
se obtiene un sistema de ecuaciones que relacionan las constantes $\lambda$ y $C$
con $\eta(i_{1,1})$ y $\eta(i_{1,2})$ de donde se puede deducir los valores de
estos cuatro elementos de $\bb{Q}(\sqrt{-3})$. La relaci\'{o}n entre los
uniformizadores es

\begin{align*}
 t & \,=\,-81(\zeta_{3}\,-\,1)\left(\eta^{3}\,-\,\eta^{2}\,+\,
 \frac{-4(\zeta_{3}-1)}{81}\right)\text{ .}
\end{align*}
El enunciado de la proposici\'{o}n se puede deducir de hacer un cambio de
variables: $w=-(\sqrt{-3})(-3\eta+1)$.
\end{proof}

\begin{propoBaranNineXNine}\label{thm:propoBaranNineXNine}
 Existe un uniformizador $y:\,X_{ns}^{+}(9)\rightarrow\bb{P}^{1}$, definido
 sobre $\bb{Q}$, tal que su relaci\'{o}n con el uniformizador $t$ est\'{a} dada por

 \begin{align}\label{eq:relUnifsTresYNueve}
  t & \,=\,\frac{-3(y^{3}+3y^{2}-6y+4)(y^{3}+3y^{2}+3y+4)%
  (5y^{3}-3y^{2}-3y+2)}{(y^{3}-3y+1)^{3}}\text{ .}
 \end{align}
\end{propoBaranNineXNine}

\begin{proof}[Demostraci\'{o}n]
 El uniformizador $y$ se define en t\'{e}rminos de $w$ y un conjugado. Veamos,
 en primer lugar, que $X_{B}$ no puede estar definida sobre $\bb{Q}$ de manera
 compatible con $X_{ns}^{+}(9)$. Es decir, sabemos que $X_{B}$ est\'{a} definida
 sobre $K:=\bb{Q}(\sqrt{-3})$, y que el cuerpo de funciones $K(X_{B})$ est\'{a}
 contenido en $K(X_{ns}^{+}(9))$. Si $\sigma$ un generador de $\Gal(K/\bb{Q})$,
 entonces, lo que queremos decir con que no es posible que $X_{B}$ est\'{e}
 definida de manera compatible sobre $X_{ns}^{+}(9)$ es que $\sigma$ no deja fijo
 al subcuerpo $K(X_{B})$ bajo la acci\'{o}n de $\Gal(K/\bb{Q})$ sobre
 $K(X_{ns}^{+}(9))$. La raz\'{o}n de esto es que, por ejemplo, los puntos
 $i_{1}$ e $i_{2}$ de $X_{ns}^{+}(3)$ son conjugados%?`en qu\'{e} sentido?
 , pero uno ramifica en $X_{B}$ y el otro no, con lo cual $\Phi_{2}$ no puede
 ser un $\bb{Q}$-morfismo y $X_{B}$ estar definida sobre $\bb{Q}$.

 La imagen del cuerpo $K(X_{B})=K(w)$ por el automorfismo $\sigma$ es
 $K(w')$, donde $w'=w^{\sigma}$. Dado el grado de $K(X_{ns}^{+}(9))/K(t)$,
 resulta que $K(X_{ns}^{+}(9))=K(w,w')$. Por otra parte, dado que $t^{\sigma}=t$,
 se deduce la igualdad

 \begin{align*}
  \zeta_{3}^{-1}(w^{3}+9w-6)&\,=\,\zeta_{3}(w'^{3}+9w'-6)\text{ .}
 \end{align*}
Esta igualdad permite escribir $w$ en t\'{e}rminos de
$u:=(w-\sqrt{-3})/(w'+\sqrt{-3})$. La relaci\'{o}n que se obtiene es

\begin{align*}
 w & \,=\,3u\sqrt{-3}\left(
 \frac{-u^{2}-\zeta_{3}^{-1}}{u^{3}-\zeta_{3}^{-1}}\right)\,+\,\sqrt{-3}
 \text{ .}
\end{align*}
En conjunto con la relaci\'{o}n entre $t$ y $w$, podemos expresar la relaci\'{o}n
entre los uniformizadores $u$ y $t$. Pero $u$ est\'{a} definido sobre
$\bb{Q}(\sqrt{-3})$. Haciendo el cambio de variables
$u=(y+\zeta_{3})/(\zeta_{3}y+1)$, $y$ cumple con que $y^{\sigma}=y$ y, adem\'{a}s,
con que la relaci\'{o}n con $t$ es la del enunciado.
\end{proof}

Una vez encontrada esta parametrizaci\'{o}n, con el fin de determinar los puntos
enteros en $X_{ns}^{+}(9)$, el paso siguiente es determinar las soluciones
$(y,t)$, con $t$ un n\'{u}mero entero e $y=m/n$ ($m$, $n$ enteros coprimos), de la
ecuaci\'{o}n \ref{eq:relUnifsTresYNueve}.
%, o, lo que es lo mismo,
%ternas $(m,n,t)$ de enteros, $(m,n)=1$, tales que
%
%\begin{align*}
% t & \,=\,\frac{-3(m^{3}+3m^{2}n-6mn^{2}+4n^{3})(m^{3}+3m^{2}n+3mn^{2}+4n^{3})%
%		  (5m^{3}-3m^{2}n-3mn^{2}+2n^{3})}{(m^{3}-3mn^{2}+n^{3})^{3}}
%		  \text{ .}
%\end{align*}

Se puede ver de manera elemental que, si $(m/n,t)$ es una soluci\'{o}n en $\bb{Z}$,
con $m$ y $n$ coprimos, entonces

\begin{align}\label{eq:denominadorTresYNueve}
 m^{3}\,-\,3mn^{2}\,+\,n^{3} & \,=\,k\text{ ,}
\end{align}
con $k\in\{\pm 1,\,\pm 3\}$.
El prolema queda reducido a hallar soluciones a la ecuaci\'{o}n
\ref{eq:denominadorTresYNueve}
con $k=1$ o $3$. Las soluciones en estos casos est\'{a}n completamente
determinadas y son nueve pares $(m,n)$. Los puntos que estos pares determinan en
$X_{ns}^{+}(9)$ son los puntos enteros de la curva, y todos, excepto uno, se
corresponden con \'{o}rdenes en cuerpos cuadr\'{a}ticos imaginarios (tabla 5.2
en \cite{baranLevelNine}). El caso excepcional es $j=3^{3}41^{3}61^{3}149^{3}$. \'{E}ste es
el $j$-invariante de una curva el\'{\i}ptica que, si bien no es CM, parece serlo
\textit{modulo} $9$. Concretamente, la acci\'{o}n de Galois sobre la
$9$-torsi\'{o}n de esta curva se realiza en el normalizador de un subgrupo de
Cartan \textit{non-split}.