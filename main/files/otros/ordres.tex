
Sea $K$ una extensi\'{o}n cuadr\'{a}tica de $\bb{Q}$. Llamamos
\emph{ideales} de $K$ a los sub-$\bb{Z}$-m\'{o}dulos de $K$ finitamente
generados $M\subset K$ tales que $M\otimes_{\bb{Z}}\bb{Q}=K$, es decir,
que contienen una base de $K$ en tanto $\bb{Q}$-espacio vectorial.
Esto es equivalente a definir los ideales de $K$ como los
sub-$\bb{Z}$-m\'{o}dulos libres de rango $[K:\bb{Q}]=2$. Un
\emph{orden} en $K$ es un ideal $I$ que es, adem\'{a}s, subanillo de
$K$.
Esta es la noci\'{o}n usual de orden en un cuerpo de n\'{u}meros.

El concepto de ideal seg\'{u}n la definic\'{o}n anterior es, sin
embargo, un poco distinta. Para ser precisos, llamaremos, dado un
ideal $M\subset K$, de $K$, \emph{orden asociado a} $M$ u
\emph{orden de} $M$ al orden $I$ del cuerpo que satisface

\begin{align*}
I & \,=\,\left\lbrace \beta\in K\,:\,\beta M\subset M \right\rbrace
\text{ .}
\end{align*}
El conjunto del lado derecho es, efectivamente, un orden en $K$ al cual
denotaremos por $I(M)$. Un ideal de $K$, $M$, se dice $I$-ideal, si
$I(M)=I$ (esto no es lo mismo que un ideal de $I$, es decir, un
sub-$I$-m\'{o}dulo de $I$).

Existen \'{o}rdenes de un cuerpo cuadr\'{a}tico $K$, y todo orden
est\'{a} contenido en un orden maximal (respecto de la inclusi\'{o}n).
De hecho, puesto que $K$ es un cuerpo de n\'{u}meros, existe un
\'{u}nico orden maximal, su anillo de enteros.
Si $I=\cal{O}_{K}$ es el anillo de enteros de $K$, se tiene, en particular, que, si $M$ es un ideal de $I$
(un sub-$I$-m\'{o}dulo de $I$), entonces $M$ es un $I$-ideal.
Si $M$ es un $I$-ideal, entonces resulta un ideal fraccionario de $K$.
Dado un orden en $K$, $I$, un sub-$I$-m\'{o}dulo de $K$ finitamente
generado se llama \emph{ideal fraccionario} de $I$ en $K$.
Diremos que un ideal fraccionario de $I$ (en este sentido) es
\emph{propio}, si su orden coincide con $I$. Siempre se tiene
la inclusi\'{o}n $I\subset I(\frak{a})$ para todo ideal
(fraccionario) $\frak{a}$ de $I$. Para todo orden maximal, $I$, la
noci\'{o}n de $I$-ideal y la de ideal fraccionario coinciden.
En general, un $I$-ideal es un ideal fraccionario \emph{propio}.

Sea $\alpha\in K$ un elemento no nulo de esta extensi\'{o}n
cuadr\'{a}tica, y sea $I\subset K$ un orden arbitrario. El
subconjunto $\alpha I$ de $K$ es un $I$-ideal. Y, si $\alpha^{-1}$ es
su inverso, $(I\alpha^{-1})(\alpha I)=II=I$. En general, si $M$ es
un ideal fraccionario de $I$, diremos que es \emph{invertible}, si
existe alg\'{u}n ideal fraccionario de $I$, $M'$, tal que
$MM'=I$. Se puede ver que, en una extensi\'{o}n cuadr\'{a}tica de
$\bb{Q}$, un ideal fraccionario de un orden es propio, si, y s\'{o}lo
si es invertible. Esto permite definir el an\'{a}logo al grupo de
clases de un cuerpo de n\'{u}meros --el grupo de clases de ideales
asociado a su anillo de enteros-- para otros \'{o}rdenes en $K$ no
necesariamente maximales: el conjunto de ideales propios de un orden
$I$ conforma un grupo, y el subconjunto de ideales principales
(que son propios) es, en realidad, un subgrupo. El cociente entre
ellos es lo que llamaremos grupo de clases de ideales de $I$. Lo
denotaremos por $\Cl(I)$. Sea $\id(I)$ el \emph{grupo} de ideales
fraccionarios propios de $I$ en $K$ (el producto de ideales determina
una estructura de grupo gracias a la equivalencia entre las nociones
de ideal propio y de ideal invertible), y sea $\rm{P}(I)$ el subgrupo
de $I$-ideales principales. El grupo de clases de $I$ es
$\Cl(I):=\id(I)/\rm{P}(I)$. En particular $\Cl(\cal{O}_{K})=\Cl(K)$.

Fijemos un cuerpo cuadr\'{a}tico $K$ y un orden $I$ en $K$.
Si $M$ es un ideal \'{\i}ntegro de $I$ (un ideal fraccionario
contenido en $I$), entonces podemos considerar el cociente
$I/M$. Al igual que en el caso particular en que $I$ es el
anillo de enteros de $K$, este cociente es finito, y su cardinal
es lo que llamaremos la \emph{norma} de $M$, y denotaremos $\nm(M)$.
%Notemos que
%toda clase en $\Cl(I)$ contiene un representante \'{\i}ntegro.

Sea $d_{K}$ el discriminante de $K$, es decir, del orden $\cal{O}_{K}$.
Como tanto $I$, como $\cal{O}_{K}$ son $\bb{Z}$-m\'{o}dulos libres
de rango $2$, el \'{\i}ndice del primero en el segundo es finito, $f$,
digamos. En particular, $f\cal{O}_{K}$ est\'{a} contenido en $I$.
Sea $\{1,w\}$ una $\bb{Z}$-base de $\cal{O}_{K}$. El orden
$\bb{Z}+f\cal{O}_{K}$, que tiene por base a $\{1,fw\}$,
tambi\'{e}n est\'{a} contenido en $I$, y es $\bb{Z}$-libre de rango
$2$. El \'{\i}ndice de $(1,fw)_{\bb{Z}}$
(el $\bb{Z}$-m\'{o}dulo generado por $1$ y $fw$) en
$\cal{O}_{K}=(1,w)_{\bb{Z}}$ es $f$, al igual que el de $I$ en
$\cal{O}_{K}$, con lo cual $I=(1,fw)_{\bb{Z}}$.

De esto se deduce, en particular, que el discriminante de $I$ es
$f^{2}d_{K}$, manteniendo la notaci\'{o}n anterior. Y, adem\'{a}s,
si $K=\bb{Q}[\sqrt{d_{K}}]$, entonces tambi\'{e}n
$K=\bb{Q}[\sqrt{D}]$, donde $D=f^{2}d_{K}$. De hecho, el discriminante
de $I$ lo determina (a $I$) un\'{\i}vocamente, y todo entero
no cuadrado congruente a $1$ o $0$ \textit{modulo} $4$ aparece de esta
manera.

Veamos algunas propiedades de la norma de ideales:

\begin{normaIdealesPropis}
Sea $I$ un orden en un cuerpo cuadr\'{a}tico imaginario. Es cierto
entonces

\begin{itemize}
\item[(i)] que la norma del ideal principal $\alpha I$ coincide con la
norma de $\alpha$, si $0\not =\alpha\in I$;
\item[(ii)] que $\nm(MM')=\nm(M)\nm(M')$ para
todo par de ideales $M,M'\subset I$ propios;
\item[(iii)] que, si $M\subset I$ es un ideal propio, entonces
$M\overline{M}=\nm(M)I$, donde $\overline{M}$
es el ideal de $K$ generado por los conjugados de los elementos
de $M$.
\end{itemize}
\end{normaIdealesPropis}

\paragraph{Relaci\'{o}n con formas cuadr\'{a}ticas}
Sea $I$ un orden en un cuerpo cuadr\'{a}tico imaginario $K$ de
discriminante $D<0$, y sea $d_{K}$ el discriminante del cuerpo.
Sea $f=ax^{2}+bxy+cy^{2}$ una forma cuadr\'{a}tica (primitiva y
definida positiva) de discriminante $D$. Como $D$ es negativo, existe
una \'{u}nica ra\'{\i}z $\tau$ del polinomio cuadr\'{a}tico $f(x,1)$
perteneciente al semiplano complejo superior, $\frak{h}$. Porque $f$
es definida positiva, $\tau$, en t\'{e}rminos del discriminante y de
los coeficientes de $f$, es igual a $(-b+\sqrt{D})/2a$.

Sea $f$ el conductor del orden $I$, de manera que $D=f^{2}d_{K}$, y sea
$w_{K}\in K$ el elemento

\begin{align*}
w_{K} & \,:=\,\frac{d_{K}+\sqrt{d_{K}}}{2}
\text{ .}
\end{align*}
Por un lado, $\cal{O}_{K}=(1,w_{K})_{\bb{Z}}$. Se puede ver que
$(1,a\tau)_{\bb{Z}}=(1,fw_{K})_{\bb{Z}}=I$, con lo que
$(a,a\tau)_{\bb{Z}}$ es un ideal propio de $I$ en $K$.

Dadas dos formas $f$ y $g$, ellas son propiamente equivalentes,
si, y s\'{o}lo si sus ra\'{\i}ces en $\frak{h}$, $\tau$ y $\tau'$,
pertenecen a la misma \'{o}rbita en el semiplano por la acci\'{o}n de
$\SL{2}(\bb{Z})$. Esto \'{u}ltimo es, a su vez, equivalente a que los
ret\'{\i}culos $[1,\tau]$ y $[1,\tau']$ en $\bb{C}$ est\'{e}n
relacionados por una homotecia definida sobre $K$: que exista $\lambda$
en $K$ tal que

\begin{align*}
[1,\tau] & \,=\,\lambda [1,\tau']
\text{ .}
\end{align*}

Todo esto muestra que, si $f$ es una forma cuadr\'{a}tica, cuya
ra\'{\i}z asociada es $\tau$, la aplicaci\'{o}n
$f\mapsto[a,a\tau]$ determina una aplicaci\'{o}n biyectiva
del grupo de clases de formas de discriminante $D$ en el grupo de
clases del orden $I$. (En principio, esta aplicaci\'{o}n es inyectiva,
pero la sobreyectividad se sigue de que todo $I$-ideal est\'{a}
generado, en tanto $\bb{Z}$-m\'{o}dulo, por dos elementos de $K$).
Esta aplicaci\'{o}n resulta ser un morfismo de grupos, con lo que
ambos son isomorfos.

Usando las propiedades de la norma de ideales de un orden en un cuerpo
cuadr\'{a}tico imaginario, podemos demostrar lo siguiente:

\begin{propoRepEnterosNormaIdeal}
Sea $f$ una forma cuadr\'{a}tica de discriminante $D$,
y sea $I$ el orden de discriminante $D$.
Sea $M$ un ideal en la clase correspondiente a la clase de $f$
en el grupo $\Cl(I)$. Dado un entero $m>0$, el mismo est\'{a}
representado por $f$, si, y s\'{o}lo si $m=\nm(M')$, para alg\'{u}n
ideal $M'$ en la misma clase que $M$.
\end{propoRepEnterosNormaIdeal}

Si $f$ es una forma primitiva, entonces $f$ representa n\'{u}meros
coprimos con $m$. Esto tiene como consecuencia el corolario
siguiente:

\begin{coroRepEnterosNormaIdeal}
Sea $I$ un \'{o}rden en un cuerpo cuadr\'{a}tico imaginaro, y sea $m$
un entero no nulo. Entonces toda clase en $\Cl(I)$ contiene un
ideal propio de $I$ cuya norma es coprima con $m$.
\end{coroRepEnterosNormaIdeal}

En lo que resta de la secci\'{o}n mostramos la relaci\'{o}n entre los
ideales de un orden en un cuerpo cuadr\'{a}tico imaginario, y los
ideles del anillo de enteros del cuerpo correspondiente. Como
consecuencia de esta relaci\'{o}n, se obtiene una f\'{o}rmula para
el n\'{u}mero de clases del orden en t\'{e}rminos del conductor y del
n\'{u}mero de clases del anillo de enteros.

\paragraph{Los ideales de un orden}

Sea $I$ un orden en un cuerpo cuadr\'{a}tico $K$, cuyo anillo de
enteros denotamos por $\cal{O}_{K}$. Sea $f$ el conductor del orden
$I$, es decir, el \'{\i}ndice de $I$ en $\cal{O}_{K}$. Si
$M\subset I$ es un ideal, decimos que $M$ es coprimo
con un entero $m$, si el $\bb{Z}$-m\'{o}dulo $M+mI$ es igual a
$I$. La norma de un ideal como $M$ la definimos como el orden
$\nm(M)$ del cociente $I/M$. En este $\bb{Z}$-m\'{o}dulo
cociente consideramos la multiplicaci\'{o}n por el conductor de $I$.
Como $I/M$ es finito, este endomorfismo es un iso, si, y
s\'{o}lo es sobreyectivo, y esto es equivalente a que el ideal
$M$ sea coprimo con el conductor de $I$. Por otro lado, por ser
un grupo abeliano finito, multiplicar por $f$ es isomorfismo, si y
s\'{o}lo si $f$ es coprimo con el orden de $I/M$. Por otra
parte, si $b$ es un elemento del cuerpo $K$ tal que
$bM\subsetM$, como el ideal es finitamente generado,
en tanto $\bb{Z}$-m\'{o}dulo, $b$ satisface un polinomio m\'{o}nico
con coeficientes enteros, es un entero en $K$. De esto y de que
$f\cal{O}_{K}\subset I$, se deduce que $b$ pertenece a $I$. Esto
implica que $M$ es un ideal propio de $I$, su orden asociado
es igual a $I$, si $M$ es coprimo con el conductor $f$.

Dado que la norma de ideales de $I$ es multiplicativa,
$\nm(MM')=\nm(M)\nm(M')$, los ideales
de $I$, coprimos con el conductor del orden forman un subconjunto
del grupo de $I$-ideales de $K$ cerrado por multiplicaci\'{o}n.
Generan un subgrupo $\id(I,f)$ de $\id(I)$ que contiene a
$\rm{P}(I,f)$, el subgrupo de ideales principales de norma
coprima con $f$. Se puede ver que existe un isomorfismo

\begin{align*}
\id(I,f)/\rm{P}(I,f) & \,\simeq\,\Cl(I)
\text{ .}
\end{align*}

Esto se aplica, en particular, al caso en que $I=\cal{O}_{K}$.

\begin{propoMaxOrd}
Sea $I$ un orden en un cuerpo cuadr\'{a}tico imaginario $K$,
de conductor $f$. Entonces, si $\frak{a}$ es un ideal de
$\cal{O}_{K}$ coprimo con $f$, el conjunto $\frak{a}\cap I$ es un
ideal en $I$ coprimo con $f$ y cuya norma coincide con la de
$\frak{a}$. En la otra direcci\'{o}n, si $M$ es un ideal en
$I$, entonces el ideal que $M$ genera en $\cal{O}_{K}$,
$M\cal{O}_{K}$, es coprimo con $f$ y su norma es igual a la de
$M$. Finalmente, $\frak{a}\mapsto\frak{a}\cap I$ da un
isomorfismo

\begin{align*}
& \id(\cal{O}_{K},f)\xrightarrow{\sim}\id(I,f)
\text{ ,}
\end{align*}
cuya inversa es $M\mapsto M\cal{O}_{K}$.
\end{propoMaxOrd}

Este resultado muestra, por un lado, que, si bien $I$ puede no
ser dominio de Dedekind, sigue existiendo descomposici\'{o}n
\'{u}nica en ideales primos (maximales) coprimos con $f$,
para todo ideal de $I$ coprimo con $f$. Por otro lado, es posible
describir el grupo de clases de $I$ en t\'{e}rminos del orden
maximal $\cal{O}_{K}$: existe un isomorfismo

\begin{align*}
\Cl(I) & \,\simeq\,\id_{K}(f)/\rm{P}_{K,\bb{Z}}(f)
\text{ ,}
\end{align*}
donde $\rm{P}_{K,\bb{Z}}(f)$ es el subgrupo de
$\id_{K}(f):=\id(\cal{O}_{K},f)$ generado por los ideales principales
$\alpha\cal{O}_{K}$, con $\alpha\in I$ y de norma coprima con $f$
(equivalentemente, $\alpha\in\cal{O}_{K}$,

\begin{align*}
\alpha & \,\equiv\, a\,\rm{mod}f\cal{O}_{K}
\text{ ,}
\end{align*}
para alg\'{u}n entero $a$ coprimo con $f$).

\paragraph{El n\'{u}mero de clases de un orden}

El n\'{u}mero de clases de orden en un cuerpo cuadr\'{a}tico imaginario
se puede expresar en t\'{e}rminos del n\'{u}mero de clases del
anillo de enteros del cuerpo y del conductor del orden.

Sea $K$ un cuerpo cuadr\'{a}tico imaginario, y sea $d_{K}$ su discriminante.
Definimos el s\'{\i}mbolo de Kronecker en $d_{K}$:

\begin{align*}
 \left(\frac{d_{K}}{2}\right) & \,:=\,
 \begin{cases}
  0 & \text{ si } 2|d_{K}\text{ ,}\\
  1 & \text{ si } d_{K}\equiv 1\,(\mod\,8)\text{ ,}\\
  -1 & \text{ si } d_{K}\equiv 5\,(\mod\,8)\text{ .}
 \end{cases}
\end{align*}
Con $(d_{K}/p)$ indicamos el s\'{\i}mbolo de Legendre, si $p\not = 2$ es
un primo, o el s\'{\i}mbolo de Kronecker, si $p$ es $2$.

\begin{teoNumDeClassRel}
 Sea $I$ un orden en un cuerpo cuadr\'{a}tico imaginario $K$, y sean
 $\cal{O}_{K}$ el anillo de enteros y $d_{K}$ el discriminante de $K$.
 Sean $h(I)$ el cardinal de $\Cl(I)$ y $h(\cal{O}_{K})$ el de $\Cl(K)$.
 Entonces, si $f=|\cal{O}_{K}:I|$,

 \begin{align*}
  h(I) & \,=\,h(\cal{O}_{K})\frac{f}{|\cal{O}_{K}^{\times}:I^{\times}|}
  \prod_{p|f}\,\left( 1\,-\,\left( \frac{d_{K}}{p} \right)\frac{1}{p} \right)
  \text{ ,}
 \end{align*}
donde $I^{\times}$ y $\cal{O}_{K}^{\times}$ son los grupos de unidades de los
\'{o}rdenes $I$ y $\cal{O}_{K}$, y el n\'{u}mero que aparece multiplicando a
$h(\cal{O}_{K})$ es un entero.
\end{teoNumDeClassRel}

Adem\'{a}s de esta f\'{o}rmula, contamos con una expresi\'{o}n para $h(\cal{O}_{K})$;

\begin{align*}
 h(\cal{O}_{K}) & \,=\,\sum_{n=1}^{|d_{K}|-1}\,\left(\frac{d_{K}}{n}\right)n
 \text{ ,}
\end{align*}
donde, si $n$ se decompone como $\prod_{i}\,p_{i}$ en producto de primos,
$(d_{K}/n)=\prod_{i}\,(d_{K}/p_{i})$.

%\paragraph{Un poco de Teor\'{\i}a de cuerpos de clases}
%
%\input{classFieldsCox}
%
%\input{complexMultiplicationCox}


