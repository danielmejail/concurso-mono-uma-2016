
Por \emph{forma cuadr\'{a}tica} nos referiremos a funciones, polinomios
homog\'{e}neos de dos variables, $x,y$, con coeficientes, $a,b,c$,
enteros:

\begin{align*}
f(x,y) & \,=\,ax^{2}\,+\,bxy\,+\,cy^{2}\text{ .}
\end{align*}
Una forma cuadr\'{a}tica se dice \emph{primitiva}, si sus coeficientes
son coprimos. Un entero $m$ se dice \emph{representado} por una forma
$f$, si existen soluciones a $m=f(x,y)$, y
\emph{representado propiamente}, si $x$ e $y$ se pueden tomar
coprimos.

El grupo $\GL{2}(\bb{Z})$ act\'{u}a sobre el conjunto de formas
cuadr\'{a}ticas binarias de la siguiente manera: si
$a$, $b$, $c$ y $d$ son enteros tales que
$ad-bc\in\bb{Z}^{\times}=\{\pm 1\}$, y si $f$ es una forma, entonces
la funci\'{o}n de dos variables

\begin{align*}
g(x,y)\,=\,
\begin{bmatrix}a&b\\c&d\end{bmatrix}\,\cdot\,f(x,y) &
\,=\,f(ax+by,cx+dy)\text{ ,}
\end{align*}
es una forma cuadr\'{a}tica. Las formas $f$ y $g$ se dicen
\emph{equivalentes}. Si la matriz que realiza la equivalencia
pertenece a $\SL{2}(\bb{Z})$, entonces $f$ y $g$ son
\emph{propiamente equivalentes}.

El discriminante de una forma $f=ax^{2}+bxy+cy^{2}$, se define como

\begin{align*}
D & \,=\,b^{2}\,-\,4ac\text{ .}
\end{align*}
Una forma cuadr\'{a}tica se dice \emph{indefinida}, si representa
tanto valores positivos, como valores negativos. En caso contrario,
se dice \emph{definida} (\emph{positivamente} o \emph{indefinidamente},
dependiendo de cu\'{a}l de estos dos conjuntos continen elemnentos
representados). Esta caracter\'{\i}stica queda determinada por el
valor del discriminante: si $D$ es negativo, la correspondiente
forma es definida, y, en ese caso, si el coeficiente de $x^{2}$ es
positivo, la forma es definida positivamente. Estas propiedades son
invariantes dentro de una misma clase de equivalencia.

\begin{definicionesLemaRepPorFormasDeDiscriminanteFijo}
Sea $D$ congruente a $0$ o a $1$ \textit{modulo} $4$, y sea $m$ un
entero impar coprimo con $D$. Entonces $m$ est\'{a} representado
propiamente por una forma de discriminante $D$, si, y s\'{o}lo si
$D$ es un cuadrado \textit{modulo} $m$.
\end{definicionesLemaRepPorFormasDeDiscriminanteFijo}

\begin{proof}
Si una forma $f$ de discriminante $D$ representa propiamente a $m$,
entonces $f$ es propiamente equivalente a una forma como

\begin{align*}
g & \,=\,mx^{2}\,+\,bxy\,+\,cy^{2}\text{ ,}
\end{align*}
con $b,c\in\bb{Z}$. Pero, entonces los discriminantes de $f$ y de $g$
difieren en un cuadrado de $\bb{Z}$, y el discriminante de $g$ es
congruente a $b^{2}\,(\rm{mod}\,m)$.

Rec\'{\i}procamente, si $D\equiv b^{2}\,(m)$, reemplazando $b$ por
$b+m$, se puede asumir que $b$ y $D$ tienen la misma paridad. As\'{\i},
existe $c\in\bb{Z}$ tal que $D\equiv b^{2}\,(4m)$. Entonces la forma
$mx^{2}+bxy+cy^{2}$ representa a $m$ propiamente, para cierto $c$
entero.
\end{proof}

Una forma primitiva definida positivamente, $f=ax^{2}+bxy+cy^{2}$,
se dice \emph{reducida}, si se satisface

\begin{itemize}
\item $|b|\leq a\leq c$, y
\item $b\geq 0$, si $|b|=a$ o si $a=c$.
\end{itemize}

\begin{definicionesTeoFormasPrimitivasEquivalenciaPropia}
Toda forma primitiva definida positivamente es propiamente
equivalente a una \'{u}nica forma reducida.
\end{definicionesTeoFormasPrimitivasEquivalenciaPropia}

Sea $f$ una forma cuadr\'{a}tica definida positiva, y sea $D<0$
su discriminante. Si $f$ es reducida, se puede ver que el coeficiente
de en $x^{2}$ est\'{a} acotado superiormente por $\sqrt{(-D)/3}$.
Siendo $f$ reducida, existen finitas elecciones posibles para los
coeficientes.

\begin{definicionesTeoNumDeClases}
Sea $D$ un entero negativo congruente a $0$ o a $1$ \textit{modulo}
$4$. Entonces el n\'{u}mero de clases de formas cuadr\'{a}ticas
primitivas definidas positivamente de discriminante $D$ es finito.
\end{definicionesTeoNumDeClases}

Al n\'{u}mero de clases de formas primitivas de discriminante $D$
lo denotaremos con $h(D)$. Esta cantidad coincide con el n\'{u}mero de
formas cuadr\'{a}ticas reducidas de discriminante $D$. Al conjunto de
clases lo denotaremos con $C(D)$.

\paragraph{El grupo de clases de formas}
De ahora en adelante las formas que se considerar\'{a}n, exceptuando
los casos en que se mencione, ser\'{a}n formas cuadr\'{a}ticas
primitivas y definidas positivamente.

Dadas dos formas $f$ y $g$, definimos su composici\'{o}n de la
siguiente manera: si $f$ y $g$ est\'{a}n dadas por

\begin{align*}
f & \,=\,ax^{2}\,+\,bxy\,+\,cy^{2}\text{ ,}\\
g & \,=\,a'x^{2}\,+\,b'xy\,+\,c'y^{2}\text{ ,}
\end{align*}
ambas de discriminante $D<0$, y son tales que el m\'{a}ximo com\'{u}n
divisor de $a$, $a'$ y $(b+b')/2$ sea igual a $1$, entonces la forma

\begin{align*}
F(x,y) & \,=\,aa'x^{2}\,+\,Bxy\,+\,\frac{B^{2}-D}{4aa'}y^{2}
\end{align*}
es una forma cuadr\'{a}tica primitiva y definida positiva denominada
\emph{composici\'{o}n de Dirichlet} de $f$ y $g$ (el entero $B$
est\'{a} determinado por ciertas ecuaciones de congruencia). Si bien
dos formas pueden no satisfacer la condici\'{o}n que se requiere de sus
coeficientes, de manera que, con esta noci\'{o}n de composici\'{o}n,
no es posible componer dos formas cualesquiera. De todas maneras,
es posible hallar otro par de formas propiamente equivalentes respectivamente para el cual esta condici\'{o}n se cumple.

\begin{definicionesCompositioGrupoDeClasesDeFormas}
Sea $D<0,D\equiv 0,1\,(4)$. Sea $C(D)$ el conjunto de clases de formas
(primitivas y definidas positivamente) de discriminante $D$. La
composici\'{o}n de Dirichlet induce un operaci\'{o}n bien definida
en $C(D)$. Si llamamos \emph{clase pricipal} a la clase que contiene
la forma --denominada \emph{forma principal}--

\begin{align*}
& x^{2}\,-\,\frac{D}{4}y^{2}\text{ o}\\
& x^{2}\,+\,xy\,+\,\frac{1-D}{4}y^{2}\text{ ,}
\end{align*}
si $D\equiv 0$ (en el primer caso) o si $D\equiv 1$ (en el segundo),
entonces esta operaci\'{o}n y la clase principal determinan una
estructura de grupo abeliano finito en $C(D)$; el inverso de la clase
de $ax^{2}+bxy+cy^{2}$ es la clase de la forma opuesta:
$ax^{2}-bxy+cy^{2}$.
\end{definicionesCompositioGrupoDeClasesDeFormas}

Una de las propiedades de la composici\'{o}n, $F$, de dos formas,
$f$ y $g$, es: existen formas bilineales $B_{1}$y $B_{2}$:

\begin{align*}
B_{i}(x,y;z,w) & \,=\, a_{i}xz\,+\,b_{i}xw\,+\,
c_{i}yz\,+\,d_{i}yw
\end{align*}
($i$ en $\{1,2\}$), tales que

\begin{align*}
f(x,y)g(z,w) & \,=\,F(B_{1}(x,y;z,w),B_{2}(x,y;z,w))
\text{ .}
\end{align*}

Particularmente importantes son los elementos de orden $\leq 2$ en
el grupo de clases de formas de discriminante $D$.

\begin{definicionesLemaFormasDeOrdenDos}
Una forma de discriminante $D$, $f=ax^{2}+bxy+cy^{2}$, reducida
tiene orden menor o igual a $2$ en $C(D)$, si, y s\'{o}lo si,
$b=0$, $a=b$ o $a=c$.
\end{definicionesLemaFormasDeOrdenDos}

\begin{proof}
Sea $g$ la forma opuesta de $f$. El orden de la clase de $f$ es
$\leq 2$, si, y s\'{o}lo si $f$ y $g$ son propiamente equivalentes.
Como $f$ es reducida, hay dos casos a considerar: o bien
$|b|<a<c$, o bien vale alguna de $b=a$ y $a=c$. En el primer caso,
$g$ tambi\'{e}n es una forma reducida, y es propiamente equivalente a
$f$, si, y s\'{o}lo si $b=0$ (unicidad).
En el segundo caso, usando las transformaciones $S$ (si $a=c$)
y $T$ (si $b=a$), donde

\begin{align*}
S & \begin{bmatrix}0&-1\\1&0\end{bmatrix}\,=\,
\text{ y}\\
T & \,=\, \begin{bmatrix}1&1\\0&1\end{bmatrix}
\text{ ,}
\end{align*}
se puede ver que $f$ y $g$ son propiamente equivalentes.
\end{proof}

\begin{definicionesPropoFormasDeOrdenDos}
Sea $D<0,D\equiv 0,1\,(4)$, y sea $r$ la cantidad de primos impares
que dividen a $D$. Si $D\equiv 1$, sea $\mu:=r$. Si $D=-4n$, definimos $\mu$ de la siguiente manera: si $n\equiv 3\,(4)$, entonces $\mu:=r$;
si $n\equiv 1,2\,(4)$, se toma $\mu:=r+1$; si $n\equiv 4\,(8)$,
$\mu:=r+1$; si $n\equiv 0\,(8)$, $\mu:=r+2$. Entonces, loos
elementos de orden, a lo sumo, $2$ en $C(D)$ son $2^{\mu-1}$.
\end{definicionesPropoFormasDeOrdenDos}

\paragraph{G\'{e}nero}
Sea $D$, un entero negativo, el discriminante de una forma
cuadr\'{a}tica. Se puede ver que existe un (\'{u}nico)
morfismo de grupos

\begin{align*}
\chi & \,:\,(\bb{Z}/D\bb{Z})^{\times}\rightarrow
\left\lbrace\pm 1\right\rbrace
\end{align*}
tal que $\chi([p])=(D/p)$ para todo primo impar $p$ que no divide a
$D$, donde $[p]$ indica la clase de $p$ en $\bb{Z}/D\bb{Z}$, y
$(D/p)$ es el s\'{\i}mbolo de Legendre. Usando los resultados
ya enunciados, se puede deducir la siguiente proposici\'{o}n:

\begin{definicionesPropoRepDePrimosPorFormas}
Sea $D\equiv 0,1\,(\rm{mod}\,4)$ y negativo. Sea $\chi$ el morfismo
mencionado. Si $p$ es un primo impar que no divide a $D$, entonces
$p$ est\'{a} representado por una de las formas
(primitivas, definidas positivamente y) reducidas de discriminante $D$,
si, y s\'{o}lo si la clase $[p]$ pertenece al n\'{u}cleo de $\chi$.
\end{definicionesPropoRepDePrimosPorFormas}

Si $D$ es de la forma $-4n$ y $p$ es un primo impar, se tiene la
igualdad

\begin{align*}
\left( \frac{D}{p} \right) & \,=\,\left( \frac{-n}{p} \right)
\text{ .}
\end{align*}
Por ejemplo, para $n=14$, $D=-56$, por reciprocidad cuadr\'{a}tica,
est\'{a}n caracterizados los primos $p$ para los que $(-14/p)=1$
de acuerdo a su clase \textit{modulo} $56$. Hay cuatro formas
reducidas de discriminante $-56$. Por el resultado anterior, para $p$,
$(-14/p)=1$ es equivalente a ser representado por una de estas formas.
Se puede verificar que las formas $x^{2}+5y^{2}$ y $2x^{2}+7y^{2}$
representan las clases en $\{1,9,15,23,25,29\}$, mientras que
$3x^{2}+2xy+5y^{2}$ y $3x^{2}-2xy+5y^{2}$ representan su complemento
en $(\bb{Z}/56\bb{Z})^{\times}$.

Fijado el discriminante $D$, decimos que dos formas pertenecen al
mismo \emph{g\'{e}nero}, si representan los mismos valores en
$(\bb{Z}/D\bb{Z})^{\times}$.

\begin{definicionesLemaClasesRepresentadas}
Dado $D<0,D\equiv 0,1\,(4)$, sea $\chi$ el morfismo definido arriba.
Sea $f$ una forma de discriminante $D$. Las clases en
$(\bb{Z}/D\bb{Z})^{\times}$ representadas por la forma principal
consforma un subgrupo, $H$, de $\ker(\chi)$. Las clases representadas
por $f$ conforman una coclase de $H$ en $\ker(\chi)$.
\end{definicionesLemaClasesRepresentadas}

Podemos dar, as\'{\i}, la siguiente definici\'{o}n: si $H'$ es una
coclase de $H$ en $\ker(\chi)$, definimos el g\'{e}nero de $H'$ como
el conjunto de aquellas formas que representan los elementos en $H'$.
De esta manera, queda definida una aplicaci\'{o}n,

\begin{align*}
\Phi & \,:\,C(D)\rightarrow\ker(\chi)/H\text{ ,}
\end{align*}
del grupo de clases de formas en el conjunto de coclases de $H$ en
$\ker(\chi)$. Recordemos que, si $f$ y $g$ son dos formas
cadr\'{a}ticas para las cuales su composici\'{o}n de Dirichlet $F$
est\'{a} definida, existen entonces formas bilineales $B_{1}$ y $B_{2}$
tales que

\begin{align*}
F(B_{1}(x,y;z,w),B_{2}(x,y;z,w)) & \,=\,f(x,y)g(z,w)\text{ .}
\end{align*}
En particular, dadas dos clases de formas en $C(D)$, y respectivos representantes $f$ y $g$, pertenecientes a g\'{e}neros
$\Phi^{-1}(H')$ y $\Phi^{-1}(H'')$, respectivamente, para las cuales
la composici\'{o}n $F$ est\'{a} definida, $f$ representa elementos en
$H'$, $g$ en $H''$ y $F$ en $H'H''$. De esto se deduce que $\Phi$ es
un morfismo de grupos.

En primera instancia, esto implica que todos los g\'{e}neros en que
$C(D)$ se subdivide contienen la misma cantidad de clases, y que,
como el subgrupo $H$ contiene todos los cuadrados de
$(\bb{Z}/D\bb{Z})^{\times}$, %evaluando la forma principal en $(x,0)$
el cociente $\ker(\chi)/H$ es isomorfo $\{\pm 1\}^{m}$ para
alg\'{u}n entero $m$, y, por lo tanto, el n\'{u}mero de g\'{e}neros es
una potencia de $2$.
%La imagen de $\Phi$ es un subgrupo de este cociente
Espec\'{\i}ficamente se tiene el siguiente resultado:

\begin{definicionesGeneraYCuadrados}
Sea $D<0,D\equiv 0,1\,(4)$, y sea $\mu$ el n\'{u}mero definido en
\ref{definicionesPropoFormasDeOrdenDos}. Entonces hay $2^{\mu-1}$
g\'{e}neros de formas de discriminante $D$, y el g\'{e}nero principal
coincide con el subgrupo $C(D)^{2}$ de cuadrados en el grupo de
clases de formas.
\end{definicionesGeneraYCuadrados}