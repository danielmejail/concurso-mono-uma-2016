A diferencia de las curvas consideradas en las soluciones anteriores
($N=3$, $5$, $7$, $8$, $9$), la curva de nivel $11$, $X_{ns}^{+}(11)$
es $\bb{Q}$-isomorfa a la curva de g\'{e}nero $1$ dada por la primero
de las siguientes ecuaciones de Weierstra{\ss}:

\begin{align*}
y^{2}\,+\,y & \,=\,x^{3}\,-\,x^{2}\,-\,7x\,+\,10\text{ ,}\\
y^{2}\,+\,11y & \,=\,x^{3}\,+\,11x^{2}\,+\,33x\text{ .}
\end{align*}
La segunda de estas ecuaciones se obtiene tras reemplzar $x$ e $y$ por
$x+4$ e $y+5$, y determina una curva isomorfa, que denotaremos $C$.
Entonces, al igual que $X_{ns}^{+}(11)$, la curva el\'{\i}ptica $C$
parametriza clases de isomorfismo de curvas el\'{\i}pticas con cierta
estructura de nivel. En [[schoof, tzanakis]] se demuestra que un punto
$\bb{Q}$-racional de $C$, $P$, corresponde a una curva el\'{\i}ptica
cuyo $j$-invariante es un entero (racional), si, y s\'{o}lo si
$P=(x,y)$ tiene la propiedad de que $x/(xy-11)$ pertenece a $\bb{Z}$.
El problema de contar los puntos enteros de $X_{ns}^{+}(11)$ se
convierte en el problema de contar los puntos racionales de $C$ cuyas
coordenadas tienen esta propiedad. El resultado central en
[[schoof, tzanakis]] es:

\begin{teoSchoofTzanakis}
Sea $C$ la curva el\'{\i}ptica dada por la ecuaci\'{o}n de
Weierstra{\ss}

\begin{align*}
y^{2}\,+\,11y & \,=\,x^{3}\,+\,11x^{2}\,+\,33x\text{ .}
\end{align*}
Existen \'{u}nicamente siete puntos $P=(x,y)$ en $C(\bb{Q})$ tales que
$x/(xy-11)$ sea un n\'{u}mero entero.
\end{teoSchoofTzanakis}
Por lo tanto, $X_{ns}^{+}(11)$ cuenta s\'{o}lo con siete puntos
enteros. Por otro lado, estos puntos enteros vienen exclusivamente
de \'{o}rdenes en cuerpos cuadr\'{a}ticos imaginarios: el primo $11$ es
inerte en los \'{o}rdenes cuadr\'{a}ticos con n\'{u}mero de clases
$1$ y de discriminante $-3$, $-4$, $-12$, $-16$, $-27$, $-67$ y $-163$.
En particular, si
$|d|$ es suficientemente grande, si $d<-44$ por ejemplo, como $11$ es
inerte en un orden de discriminante $d$, su n\'{u}mero de clases no
podr\'{a} ser igual a $1$, excepto que $d$ sea uno de los ya
mencionados. Esta soluci\'{o}n del problema difiere del resto en que
el g\'{e}nero de la curva modular es $1$ (si bien para $N=24$ el
g\'{e}nero tambi\'{e}n es $1$, la soluci\'{o}n, en ese caso, viene de
considerar parametrizaciones de las curvas de g\'{e}nero $0$
$X_{ns}^{+}(3)$ y $X_{ns}^{+}(8)$ [[booher]]).
Resumimos, a continuaci\'{o}n, la demostraci\'{o}n del teorema
[[teoSchoofTzanakis]].

En la ecuaci\'{o}n que define a $C$, reemplazando $y$ por
$(y-11)/2$ y luego $y/2$ por $y$, obtenemos

\begin{align*}
y^{2} & \,=\,x^{3}\,+\,11x^{2}\,+\,33x\,+\,\frac{121}{4}
\,=:\,q(x)\text{ .}
\end{align*}
Llamemos $\widetilde{C}$ a la curva el\'{\i}ptica que esta ecuaci\'{o}n
determina.
El polinomio $q$ tiene un \'{u}nico cero en $\bb{R}$. En particular,
dado que, si $(x,y)$ es un punto de orden $2$ de $\widetilde{C}$,
$y$ debe ser igual a $0$ y $x$ una ra\'{\i}z de $q$, el subgrupo
$\widetilde{C}[2]$ no puede estar contenido en $\widetilde{C}(\bb{R})$.
Esto implica que
$\widetilde{C}(\bb{R})$ es isomorfo a $\bb{R}/\bb{Z}$ y, en particular,
tiene una \'{u}nica componente conexa. Todo esto es cierto, tambi\'{e}n,
para la curva $C$.

Sea $t$ la funci\'{o}n en $C$ definida por la expresi\'{o}n
$t=y-(11/x)$. Si $(x,y)$ es un cero de $t$, entonces $x$ es un cero del
polinomio $x^{5}+11x^{4}+33x^{3}-121x-121$. Las ra\'{\i}ces de este
polinomio son reales, con lo cual, los (cinco) puntos $(x,y)$ que son
ceros de la funci\'{o}n $t$ son todos reales. El morfismo
$j:\,X_{ns}^{+}(11)\rightarrow\bb{P}^{1}$ determina un morfismo de la
curva el\'{\i}ptica $C$ en $\bb{P}^{1}$ a trav\'{e}s de un
isomorfismo definido sobre $\bb{Q}$ entre $X_{ns}^{+}(11)$ y $C$.
Este isomorfismo es elegido de manera que las c\'{u}spides de
$X_{ns}^{+}(11)$, que son cinco, se coprrespondan con los ceros de $t$
(ver [[schoof, tzanakis]], y las referencias que all\'{\i} se
encuentran).

Sea $\omega=\omega_{C}=dx/(2y+11)$ el diferencial invariante de $C$.
La integral $\int_{\cal{O}}^{P}\,\omega$ (donde $\cal{O}$ es el
punto neutro de la curva el\'{\i}ptica y $P$ un punto arbitrario)
no est\'{a} bien definida como elemento de $\bb{C}$, pero las
posibles ambig\"{u}edades surgen de la elecci\'{o}n del camino entre
los puntos. As\'{\i}, \textit{modulo} el ret\'{\i}culo que se le asocia
a $C$ v\'{\i}a sus per\'{\i}odos, pdemos definir $\lambda(P)$ como el
valor de esta integral \textit{modulo} el ret\'{\i}culo. Ahora, si
$P$ es un punto real de $C$, entonces existe un camino que lo une con
$\cal{O}$ y que est\'{a} contenido en $C(\bb{R})$. Esto muestra que
$\lambda(P)$ es un n\'{u}mero real, si $P$ es un punto real. El grupo
de puntos reales de $C$ es isomorfo a $\bb{R}/\bb{Z}$ v\'{\i}a la
aplicaci\'{o}n $\lambda$ y la elecci\'{o}n de un per\'{\i}odo real
para $C$. Por ejemplo, (identificando las curvas $C$ y $\widetilde{C}$)
se puede tomar
$\Omega:=\int_{r}^{\infty}\,dx/\sqrt{q(x)}$, donde $r$ es la
ra\'{\i}z real del polinomio $q$ definido antes. 

Antes de pasar a la demostraci\'{o}n, es necesaria una \'{u}ltima
definici\'{o}n. Si $f$ es una funci\'{o}n en la curva $C$ y si no es
constante, dado un punto racional, $P\in C(\bb{Q})$, definimos la
altura de $P$ (respecto de $f$) como el producto
$H_{f}(P):=\prod_{v}\,\rm{max}\{1,|f(P)|_{v}\}$, donde $v$ recorre
los lugares de $\bb{Q}$. La altura logar\'{\i}tmica asociada es
$h_{f}(P):=\rm{log}(H_{f}(P))$ y, finalmente, si $f$ es, adem\'{a}s,
una funci\'{o}n par, la altura can\'{o}nica se define como

\begin{align*}
\widehat{h}(P) & \,:=\,\frac{1}{\rm{deg}(f)}
\rm{lim}_{n\rightarrow\infty}\,\frac{h_{f}([2^{n}]P)}{4^{n}}\text{ .}
\end{align*}
La funci\'{o}n $t=y-(11/x)$ no es par, pero se la puede relacionar con
la altura can\'{o}nica: si $P\in C(\bb{Q})$, entonces
$\widehat{h}(P)$ est\'{a} acotada por $(1/3)h_{t}(P)+4,52$.

La demostraci\'{o}n del teorema [[teoSchoofTzanakis]] se puede dividir
en una serie de pasos.



