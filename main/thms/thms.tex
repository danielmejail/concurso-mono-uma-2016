
%losNueveCuerpos
\theoremstyle{plain}
\newtheorem{propoEquivsNumClasUno}{Proposici\'{o}n}[section]
\newtheorem{teoStarkHeegner}[propoEquivsNumClasUno]{Teorema}

%-------------------------
%generalidades
\theoremstyle{plain}
\newtheorem{propoDefinidasSobreQ}{Proposici\'{o}n}[section]
\newtheorem{propoPuntosEnterosCuerpoQuad}[propoDefinidasSobreQ]{Proposici\'{o}n}
\theoremstyle{remark}
\newtheorem*{defSubgrupoDeCartan}{Definici\'{o}n}
\newtheorem*{ejemploSubgrupoDeCartan}{Ejemplo}
\newtheorem*{obsDetSobreUnidades}{Observaci\'{o}n}

%ramBaran
\theoremstyle{plain}
\newtheorem{lemaRepsCoclasesDelNormalizador}[propoDefinidasSobreQ]{Lema}

%--------------------------
%nivel24
\theoremstyle{plain}
\newtheorem{teoUnifsOchoYTres}{Teorema}[section]
\theoremstyle{remark}
\newtheorem*{obsSerre}{Observaci\'{o}n}

%nivel7
\theoremstyle{plain}
\newtheorem{lemaUnoKenku}[teoUnifsOchoYTres]{Lema}

%nivel5

%nivel11
\theoremstyle{plain}
\newtheorem{teoSchoofTzanakis}[teoUnifsOchoYTres]{Teorema}
\theoremstyle{remark}
\newtheorem*{obsGenUno}{Observaci\'{o}n}

%nivel9
\theoremstyle{plain}
\newtheorem{propoBaranNineXB}[teoUnifsOchoYTres]{Proposici\'{o}n}
\newtheorem{propoBaranNineXNine}[teoUnifsOchoYTres]{Proposici\'{o}n}